%%%%%%%%%%%%%%%%%%%%%%%%%%%%%%%%%%%%%%%%%%%%%%%%%%%%%%%%%%%%%%%
% sample preface
%
% Copy it to a new file with a new name and use it as the basis
% for your article
%
%%%%%%%%%%%%%%%%%%%%%%%% Springer-Verlag %%%%%%%%%%%%%%%%%%%%%%%%%%

\preface

\emph{PicoLisp Works} is a compilation of (almost) all available
information about the technological gem \emph{PicoLisp} - a
programming language and environment that definitely deserves wider
attention. 

Built on the unique characteristics of Lisp (almost no syntax, code is
equivalent to data), PicoLisp combines powerful abstractions with
simplicity and purity.

In a software world that is driven by hypes and desillusions, a
language like PicoLisp almost appears as timeless as mathematics. With
its roots in the very beginning of programming language development
(Lisp was, together with Fortran, among the very first of its kind),
PicoLisp may well represent the future too -- as a candidate for being
the "hundred-year language", that all programming languages finally
converge into. 

As Paul Graham puts it in his famous essay~\footnote{Paul Graham: ``The Hundred-Year Language'' \url{http://paulgraham.com/hundred.html}, 2003}:

\begin{verse}
  \textit{The hundred-year language could, in principle, be
    designed today, and such a language, if it existed, might be
    good to program in today.}  
\end{verse}

This book consists of references, tutorials, articles and essays about
PicoLisp. The reader should consider all documents written by
\emph{Alexander Burger}, the creator of PicoLisp, as the ``official''
references for the language. While the community tutorials and
articles might be very helpful and a great source of information,
they are just that - documents written by members of the PicoLisp
community at various stages of ``PicoLisp Enlightment''. When in doubt
about substantive or style questions, always refer to the offcial docs
as last instance. 

One of the official articles, \emph{A Unifying Language for Database
  And User Interface Development}, has been written as early as 2002
and is therefore slightly out of date in some technical details. It
describes the old \emph{Java-Applet GUI} that is not supported
anymore. However, since the conceptual reasoning in this article is
still valid and of fundamental importance, it was included in this
book anyway.

\emph{PicoLisp Works} is accompanied by a second volume,
\emph{PicoLisp by Example}, with more than 600 PicoLisp solutions to a
wide range of programming tasks as well as the full PicoLisp function
reference. Both volumes are freely available as \emph{pdf} files, e.g.
on \emph{Scribd}~\footnote{scribd.com}. They are published under the
\emph{GNU Free Documentation Licence}, their source code is available
in public \emph{Github}
repositories~\footnote{https://github.com/tj64/picolisp-works}~\footnote{https://github.com/tj64/picolisp-by-example}.



\vspace{1cm}
\begin{flushright}\noindent
Berlin, August 2012 \hfill {\it Thorsten Jolitz}\\
\end{flushright}

\vfill


% \vspace{1cm}
% \begin{flushright}\noindent
% Berlin, Langweid\hfill {\it Thorsten Jolitz}\\
% August 2012\hfill {\it Alexander Burger}\\
% \end{flushright}


