\title{Some technical questions and answers}
\author{Alexander Burger}
% Use \authorrunning{Short Title} for an abbreviated version of
% your contribution title if the original one is too long
\institute{\texttt{abu@software-lab.de}}
%
% Use the package "url.sty" to avoid
% problems with special characters
% used in your e-mail or web address
%


\maketitle

\begin{abstract}
  These are some technical questions about PicoLisp with answers,
  additional to the official FAQ.
\end{abstract}


\section{Can there be more than one copy of the symbol \textbf{T}?}
\label{sec:tj-fix-labels-some-technical-questions-and-answers}

\paragraph{Question}
\label{sec:dfs}

\textbf{NIL} is a special symbol which exists exactly once in the whole system.
Can there be more than one copy of the symbol \textbf{T}?


\paragraph{Answer}
\label{sec:dfs}

In this sense, \textit{any} internal symbol is unique, and exists exactly once
in the system. And any internal symbol (also 'NIL') could exist a second time,
but it cannot be interned at the same time (and would thus be a transient
symbol).

``interned'' means no more or less that an entry in the ``internal'' symbol table
points to that symbol.

'NIL' is special, however, because even if you would succeed to intern a new
(transient) symbol ``NIL'' (it isn't possible, as the existing 'NIL' cannot be
uninterned), it would not have the speciality of the old 'NIL', e.g. being
returned from boolean functions, because these functions return a hard-compiled
pointer to the old 'NIL', and conditionals check for this pointer.

Take a normal symbol, say ``abc''. You can create several transient symbols ``abc''
in the system, for example by loading them from different source files, or
separating the input with (====), by 'pack'ing them and so on.

Now you could 'intern' one of those ``abc'' symbols. It will appear as 'abc'. The
reader will always return that interned symbol when it sees 'abc'. A call to
(intern ``abc'') also would return that already-interned symbol 'abc'. To change
another one of the above ``abc'' symbols to 'abc', you'll first have to unintern
(with 'zap') the existing 'abc'.


\section{Why is the symbol \textbf{T} not protected like \textbf{NIL}?}
\label{sec:sfs}

\paragraph{Question}
\label{sec:dfs}

(Related to the one above) If one tries to put a property on the
\textbf{NIL} symbol, one gets the message ``NIL -- Protected symbol``.
Why is the symbol \textbf{T} not protected likewise?

\paragraph{Answer}
\label{sec:dfs}

Perhaps is this error message not necessary, at least not in the 64-bit version.
For a background, see the structure of 'NIL' in ``doc/structures``:

\begin{wideverbatim}
      NIL:  /
            |
            V
      +-----+-----+-----+-----+
      |  /  |  /  |  /  |  /  |
      +-----+--+--+-----+-----+
\end{wideverbatim}

'NIL' is the only symbol that has a double nature: It is a symbol \textit{and} a cons pair.
It doesn't even have a name in the 32-bit version, the reader and printer simply
know about it, and read and print 'N', 'I', and 'L' by themselves.

In the 64-bit version the situation is slightly different:
\begin{wideverbatim}
      NIL:  /
            |
            V
      +-----+-----+-----+-----+
      |'LIN'|  /  |  /  |  /  |
      +-----+--+--+-----+-----+
\end{wideverbatim}

Here NIL \textit{does} have a name. This field where you see the letters 'N', 'I' and
'L' here, is called the symbol's tail. Besides the name, it can also hold a
property list.

Technically, there is perhaps no problem when storing also properties in that
tail of 'NIL', and it might work (at least on the 64-bit version) if we removed
the above error message.

However, I think it is wise to prohibit properties in 'NIL', as ``NIL'' also means
``nothing'', and storing properties in ``nothing'' sounds a bit sick.

Any opinions?

\section{Why does the REPL exit when \textbf{NIL} is typed?}
\label{sec:fsl}


\paragraph{Question}
\label{sec:dfs}

If you type just ``NIL'' or ``()'' on the command line in the REPL,
then the REPL exits. Why is that?


\paragraph{Answer}
\label{sec:dfs}

The REPL is basically

\begin{wideverbatim}
   (while (read)
      (eval @) )
\end{wideverbatim}

'read' returns 'NIL' upon end of file, but also when it indeed \textit{reads} 'NIL'.

The same happens btw if you write the atom 'NIL' somewhere in a 'load'ed file.
This is a convenient way to ``comment'' the rest of the file.

Even better is \textit{conditional commenting} of the rest of a file, by using
a backquote read macro. When `(condition) is read, the rest of the file
will be ignored if (condition) evaluates to 'NIL'. There are examples
for that in the sources, e.g. at the end of ``lib/form.l``, where `*Dbg causes
the rest of the file to be loaded only if in debugging mode.

\emph{Note:} As of picoLisp-3.0.6, the interpreter does no longer exit
when the top level REPL reads NIL. This is a bit inconsistent, but
more what seems to be expected by most people.


\section{PicoLisp indicated that 'be' was undefined - why?}
\label{sec:dsfs}

\paragraph{Question}
\label{sec:dfs}

At this URL
(\underline{http://rosettacode.org/wiki/Pascal\%27s\_triangle/Puzzle\#PicoLisp}\footnote{http://rosettacode.org/wiki/Pascal\%27s\_triangle/Puzzle\#PicoLisp})
there is some PicoLisp code for solving Pascal's triangle. I tried it
out on my machine and PicoLisp indicated that 'be' was undefined.
Where would I find it? I'm running version 3.0.4 on a Windows 7 Home
Premium 64-bit system.

\paragraph{Answer}
\label{sec:dfs}

This is most probably because you didn't load the full system, but
just the 'picolisp' binary perhaps.

Normally, PicoLisp is either started locally with the 'p' or 'dbg' scripts:
\begin{wideverbatim}
   $ ./p +
   :
\end{wideverbatim}

or globally (e.g. when installed via some package) with the 'pil' command:
\begin{wideverbatim}
   $ pil +
   :
\end{wideverbatim}

In both cases, the full system is loaded. The trailing '+' indicates
debug mode.
\begin{wideverbatim}
   : (pp 'be)
   (de be CL
      (with (car CL)
         (if (== *Rule This)
            (=: T (conc (: T) (cons (cdr CL))))
            (=: T (cons (cdr CL)))
            (setq *Rule This) )
         This ) )
   -> be
\end{wideverbatim}

