\title{Frequently Asked Questions (FAQ)}
\author{Alexander Burger}
% Use \authorrunning{Short Title} for an abbreviated version of
% your contribution title if the original one is too long
\institute{\texttt{abu@software-lab.de}}
%
% Use the package "url.sty" to avoid
% problems with special characters
% used in your e-mail or web address
%


\maketitle

\begin{quote}
  \begin{flushleft}
   Monk: ``If I have nothing in my mind, what shall I do?''\\
   Joshu: ``Throw it out.''\\
    \emph{Monk: ``But if there is nothing, how can I throw it out?''\\
    Joshu: ``Well, then carry it out.''\\ (Zen koan)}
  \end{flushleft}
\end{quote}

\begin{abstract}
These are the official FAQ for PicoLisp. 
\end{abstract}

\section{Why did you write yet another Lisp?}
\label{sec:faq-why-did-you-write-yet-another-lisp?}


Because other Lisps are not the way I'd like them to be. They
concentrate on efficient compilation, and lost the one-to-one
relationship of language and virtual machine of an interpreted system,
gave up power and flexibility, and impose unnecessary limitations on the
freedom of the programmer. Other reasons are the case-insensitivity and
complexity of current Lisp systems.

 
\section{Who can use PicoLisp?}
\label{sec:faq-who-can-use-picolisp?}


PicoLisp is for programmers who want to control their programming
environment, at all levels, from the application domain down to the bare
metal. Who want use a transparent and simple - yet universal -
programming model, and want to know exactly what is going on. This is an
aspect influenced by \texttt{Forth}.

It does \emph{not} pretend to be easy to learn. There are already plenty of
languages that do so. It is not for people who don't care what's under
the hood, who just want to get their application running. They are
better served with some standard, ``safe'' black-box, which may be easier
to learn, and which allegedly better protects them from their own
mistakes.

 
\section{What are the advantages over other Lisp systems?}
\label{sec:faq-what-are-the-advantages-over-other-lisp-systems?}
\subsection{Simplicity}
\label{sec:faq-simplicity}


PicoLisp is easy to understand and adapt. There is no compiler enforcing
special rules, and the interpreter is simple and straightforward. There
are only three data types: Numbers, symbols and lists (``LISP'' means
``List-, Integer- and Symbol Processing'' after all ;-). The memory
footprint is minimal, and the tarball size of the whole system is just a
few hundred kilobytes.
\subsection{A Clear Model}
\label{sec:faq-a-clear-model}


Most other systems define the language, and leave it up to the
implementation to follow the specifications. Therefore, language
designers try to be as abstract and general as possible, leaving many
questions and ambiguities to the users of the language.

PicoLisp does the opposite. Initially, only the single-cell data
structure was defined, and then the structure of numbers, symbols and
lists as they are composed of these cells. Everything else in the whole
system follows from these axioms. This is documented in the chapter
about the \hyperref[ref.html-vm]{The PicoLisp Machine} in the reference manual.
\subsection{Orthogonality}
\label{sec:faq-orthogonality}


There is only one symbolic data type, no distinction (confusion) between
symbols, strings, variables, special variables and identifiers.

Most data-manipulation functions operate on the value cells of symbols
as well as the CARs of list cells:


\begin{wideverbatim}
: (let (N 7  L (7 7 7)) (inc 'N) (inc (cdr L)) (cons N L))
-> (8 7 8 7)
\end{wideverbatim}

There is only a single functional type, no ``special forms''. As there is
no compiler, functions can be used instead of macros. No special
``syntax'' constructs are needed. This allows a completely orthogonal use
of functions. For example, most other Lisps do not allow calls like


\begin{wideverbatim}
: (mapcar if '(T NIL T NIL) '(1 2 3 4) '(5 6 7 8))
-> (1 6 3 8)
\end{wideverbatim}

PicoLisp has no such restrictions. It favors the principle of ``Least
Astonishment''.
\subsection{Object System}
\label{sec:faq-object-system}


The OOP system is very powerful, because it is fully dynamic, yet
extremely simple:

\begin{itemize}
\item In other systems you have to statically declare ``slots''. In PicoLisp,
   classes and objects are completely dynamic, they are created and
   extended at runtime. ``Slots'' don't even exist at creation time. They
   spring into existence purely dynamically. You can add any new
   property or any new method to any single object, at any time,
   regardless of its class.
\item The multiple inheritance is such that not only classes can have
   several superclasses, but each individual object can be of more than
   one class.
\item Prefix classes can surgically change the inheritance tree for any
   class or object. They behave like Mixins in this regard.
\item Fine-control of inheritance in methods with \texttt{super} and \texttt{extra}.
\end{itemize}
\subsection{Pragmatism}
\label{sec:faq-pragmatism}


PicoLisp has many practical features not found in other Lisp dialects.
Among them are:

\begin{itemize}
\item Auto-quoting of lists when the CAR is a number. Instead of \texttt{'(1 2 3)}    you can just write \texttt{(1 2 3)}. This is possible because a number never
   makes sense as a function name, and has to be checked at runtime
   anyway.
\item The \texttt{quote} function returns all unevaluated arguments, instead of
   just the first one. This is both faster (\texttt{quote} does not have to
   take the CAR of its argument list) and smaller (a single cell instead
   of two). For example, \texttt{'A} expands to \texttt{(quote . A)} and \texttt{'(A B C)}    expands to \texttt{(quote A B C)}.
\item The symbol \texttt{@} is automatically maintained as a local variable, and
   set implicitly in certain flow- and logic-functions. This makes it
   often unnecessary to allocate and assign local variables.
\item \hyperref[tut.html-funio]{Functional I/O} is more convenient than explicitly
   passing around file descriptors.
\item A well-defined \hyperref[ref.html-cmp]{ordinal relationship} between
   arbitrary data types facilitates generalized comparing and sorting.
\item Uniform handling of \texttt{var} locations (i.e. values of symbols and CARs
   of list cells).
\item The universality and usefulness of symbol properties is enforced and
   extended with implicit and explicit bindings of the symbol \texttt{This} in
   combination with the access functions \texttt{=:}  \texttt{:} and \texttt{::}.
\item A very convenient list-building machinery, using the \texttt{link}, \texttt{yoke},
   \texttt{chain} and \texttt{made} functions in the \texttt{make} environment.
\item The syntax of often-used functions is kept non-verbose. For example,
   instead of \texttt{(let ((A 1) (B 2) C 3) ..)} you write
   \texttt{(let (A 1 B 2 C 3) ..)}, or just \texttt{(let A 1 ..)} if there is only a
   single variable.
\item The use of the hash (\texttt{\#}) as a comment character is more adequate
   today, and allows a clean hash-bang (\texttt{\#!}) syntax for stand-alone
   scripts.
\item The interpreter is \hyperref[ref.html-invoc]{invoked} with a simple and
   flexible syntax, where command line arguments are either files to be
   interpreted or functions to be directly executed. With that, many
   tasks can be performed without writing a separate
   \hyperref[tut.html-script]{script}.
\item A sophisticated system of interprocess communication, file locking
   and synchronization allows multi-user access to database
   applications.
\item A Prolog interpreter is tightly integrated into the language. Prolog
   clauses can call Lisp expressions and vice versa, and a
   self-adjusting depth-first search predicate \texttt{select} can be used in
   database queries.
\end{itemize}

\subsection{Persistent Symbols}
\label{sec:faq-persistent-symbols}


Database objects (``external'' symbols) are a primary data type in
PicoLisp. They look like normal symbols to the programmer, but are
managed in the database (fetched from, and stored to) automatically by
the system. Symbol manipulation functions like \texttt{set}, \texttt{put} or \texttt{get},
the garbage collector, and other parts of the interpreter know about
them.
\subsection{Application Server}
\label{sec:faq-application-server}


It is a stand-alone system (it does not depend on external programs like
Apache or MySQL) and it provides a ``live'' user interface on the client
side, with an application server session for each connected client. The
GUI layout and behavior are described with S-expressions, generated
dynamically at runtime, and interact directly with the database
structures.
\subsection{Localization}
\label{sec:faq-localization}


Internal exclusive and full use of UTF--8 encoding, and self-translating
\hyperref[ref.html-transient-io]{transient symbols} (strings), make it easy to
write country- and language-independent applications.

 
\section{How is the performance compared to other Lisp systems?}
\label{sec:faq-how-is-the-performance-compared-to-other-lisp-systems?}


Despite the fact that PicoLisp is an interpreted-only system, the
performance is quite good. Typical Lisp programs operating on list data
structures are executed in (interpreted) PicoLisp at about the same
speed as in (compiled) CMUCL, and about two or three times faster than
in CLisp or Scheme48. Programs with lots of numeric calculations,
however, may be slower on a 32-bit system, due to PicoLisp's somewhat
inefficient implementation of numbers. The 64-bit version improved on
that.

But in practice, speed was never a problem, even with the first versions
of PicoLisp in 1988 on a Mac II with a 12 MHz CPU. And certain things
are cleaner and easier to do in plain \texttt{C} or \texttt{asm} anyway. It is very
easy to write \texttt{C} functions in PicoLisp, either in the kernel, as shared
object libraries, or even inline in the Lisp code.

PicoLisp is very space-effective. Other Lisp systems reserve heap space
twice as much as needed, or use rather large internal structures to
store cells and symbols. Each cell or minimal symbol in PicoLisp
consists of only two pointers. No additional tags are stored, because
they are implied in the pointer encodings. No gaps remain in the heap
during allocation, as there are only objects of a single size. As a
result, consing and garbage collection are very fast, and overall
performance benefits from a better cache efficiency. Heap and stack grow
automatically, and are limited only by hardware and operating system
constraints.

 
\section{What means ``interpreted''?}
\label{sec:faq-what-means-interpreted?}


It means to directly execute Lisp data as program code. No
transformation to another representation of the code (e.g. compilation),
and no structural modifications of these data, takes place.

Lisp data are the ``real'' things, like numbers, symbols and lists, which
can be directly handled by the system. They are \emph{not} the textual
representation of these structures (which is outside the Lisp realm and
taken care of the \texttt{read} ng and \texttt{print} ng interfaces).

The following example builds a function and immediately calls it with
two arguments:


\begin{wideverbatim}
: ((list (list 'X 'Y) (list '* 'X 'Y)) 3 4)
-> 12
\end{wideverbatim}

Note that no time is wasted to build up a lexical environment. Variable
bindings take place dynamically during interpretation.

A PicoLisp function is able to inspect or modify itself while it is
running (though this is rarely done in application programming). The
following function modifies itself by incrementing the `0' in its body:


\begin{wideverbatim}
(de incMe ()
   (do 8
      (printsp 0)
      (inc (cdadr (cdadr incMe))) ) )

: (incMe)
0 1 2 3 4 5 6 7 -> 8
: (incMe)
8 9 10 11 12 13 14 15 -> 16
\end{wideverbatim}

Only an interpreted Lisp can fully support such ``Equivalence of Code and
Data''. If executable pieces of data are used frequently, like in
PicoLisp's dynamically generated GUI, a fast interpreter is preferable
over any compiler.

 
\section{Is there (or will be in the future) a compiler available?}
\label{sec:faq-is-there-(or-will-be-in-the-future)-a-compiler-available?}


No. That would contradict the idea of PicoLisp's simple virtual machine
structure. A compiler transforms it to another (physical) machine, with
the result that many assumptions about the machine's behavior won't hold
any more. Besides that, PicoLisp primitive functions evaluate their
arguments independently and are not suited for being called from
compiled code. Finally, the gain in execution speed would probably not
be worth the effort. Typical PicoLisp applications often use single-pass
code which is loaded, executed and thrown away; a process that would be
considerably slowed down by compilation.

 
\section{Is it portable?}
\label{sec:faq-is-it-portable?}


Yes and No. Though we wrote and tested PicoLisp originally only on
Linux, it now also runs on FreeBSD, Mac OS X (Darwin), Cygwin/Win32, and
probably other POSIX systems. The first versions were even fully
portable between DOS, SCO-Unix and Macintosh systems. But today we have
Linux. Linux itself is very portable, and you can get access to a Linux
system almost everywhere. So why bother?

The GUI is completely platform independent (Browser), and in the times
of Internet an application server does not really need to be portable.

 
\section{Is PicoLisp a web server?}
\label{sec:faq-is-picolisp-a-web-server?}


Not really, but it evolved a great deal into that direction.

Historically it was the other way round: We had a plain X11 GUI for our
applications, and needed something platform independent. The solution
was obvious: Browsers are installed virtually everywhere. So we
developed a protocol which persuades a browser to function as a GUI
front-end to our applications. This is much simpler than to develop a
full-blown web server.

 
\section{I cannot find the LAMBDA keyword in PicoLisp}
\label{sec:faq-i-cannot-find-the-lambda-keyword-in-picolisp}


Because it isn't there. The reason is that it is redundant; it is
equivalent to the \texttt{quote} function in any aspect, because there's no
distinction between code and data in PicoLisp, and \texttt{quote} returns the
whole (unevaluated) argument list. If you insist on it, you can define
your own \texttt{lambda}:


\begin{wideverbatim}
: (def 'lambda quote)
-> lambda
: ((lambda (X Y) (+ X Y)) 3 4)
-> 7
: (mapcar (lambda (X) (+ 1 X)) '(1 2 3 4 5))
-> (2 3 4 5 6)
\end{wideverbatim}

 
\section{Why do you use dynamic variable binding?}
\label{sec:faq-why-do-you-use-dynamic-variable-binding?}


Dynamic binding is very powerful, because there is only one single,
dynamically changing environment active all the time. This makes it
possible (e.g. for program snippets, interspersed with application data
and/or passed over the network) to access the whole application context,
freely, yet in a dynamically controlled manner. And (shallow) dynamic
binding is the fastest method for a Lisp interpreter.

Lexical binding is more limited by definition, because each environment
is deliberately restricted to the visible (textual) static scope within
its establishing form. Therefore, most Lisps with lexical binding
introduce ``special variables'' to support dynamic binding as well, and
constructs like \texttt{labels} to extend the scope of variables beyond a
single function.

In PicoLisp, function definitions are normal symbol values. They can be
dynamically rebound like other variables. As a useful real-world
example, take this little gem:


\begin{wideverbatim}
(de recur recurse
   (run (cdr recurse)) )
\end{wideverbatim}

It implements anonymous recursion, by defining \texttt{recur} statically and
\texttt{recurse} dynamically. Usually it is very cumbersome to think up a name
for a function (like the following one) which is used only in a single
place. But with \texttt{recur} and \texttt{recurse} you can simply write:


\begin{wideverbatim}
: (mapcar
   '((N)
      (recur (N)
         (if (=0 N)
            1
            (* N (recurse (- N 1))) ) ) )
   (1 2 3 4 5 6 7 8) )
-> (1 2 6 24 120 720 5040 40320)
\end{wideverbatim}

Needless to say, the call to \texttt{recurse} does not have to reside in the
same function as the corresponding \texttt{recur}. Can you implement anonymous
recursion so elegantly with lexical binding?

 
\section{Are there no problems caused by dynamic binding?}
\label{sec:faq-are-there-no-problems-caused-by-dynamic-binding?}


You mean the \emph{funarg} problem, or problems that arise when a variable
might be bound to \emph{itself}? For that reason we have a convention in
PicoLisp to use \hyperref[ref.html-transient-io]{transient symbols} (instead of
internal symbols)

\begin{enumerate}
\item for all parameters and locals, when functional arguments or
   executable lists are passed through the current dynamic bindings
\item for a parameter or local, when that symbol might possibly be
   (directly or indirectly) bound to itself, and the bound symbol's
   value is accessed in the dynamic context
\end{enumerate}

This is a form of lexical \emph{scoping} - though we still have dynamic
\emph{binding} - of symbols, similar to the \texttt{static} keyword in \texttt{C}.

In fact, these problems are a real threat, and may lead to mysterious
bugs (other Lisps have similar problems, e.g. with symbol capture in
macros). They can be avoided, however, when the above conventions are
observed. As an example, consider a function which doubles the value in
a variable:


\begin{wideverbatim}
(de double (Var)
   (set Var (* 2 (val Var))) )
\end{wideverbatim}

This works fine, as long as we call it as \texttt{(double 'X)}, but will break
if we call it as \texttt{(double 'Var)}. Therefore, the correct implementation
of \texttt{double} should be:


\begin{wideverbatim}
(de double ("Var")
   (set "Var" (* 2 (val "Var"))) )
\end{wideverbatim}

If \texttt{double} is defined that way in a separate source file,
and/or isolated via the \texttt{====} function, then the symbol
\texttt{Var} is locked into a private lexical context and cannot
conflict with other symbols.

Admittedly, there are two disadvantages with this solution:

\begin{enumerate}
\item The rules for when to use transient symbols are a bit complicated.
   Though it is safe to use them even when not necessary, it will take
   more space then and be more difficult to debug.
\item The string-like syntax of transient symbols as variables may look
   strange to alumni of other languages.
\end{enumerate}

Fortunately, these pitfalls do not occur so very often, and seem more
likely in utilities than in production code, so that they can be easily
encapsulated.

 
\section{But with dynamic binding I cannot implement closures!}
\label{sec:faq-but-with-dynamic-binding-i-cannot-implement-closures!}


This is not true. Closures are a matter of scope, not of binding.

For a closure it is necessary to build and maintain a separate
environment. In a system with lexical bindings, this has to be done at
\emph{each} function call, and for compiled code it is the most efficient
strategy anyway, because it is done once by the compiler, and can then
be accessed as stack frames at runtime.

For an interpreter, however, this is quite an overhead. So it should not
be done automatically at each and every function invocation, but only if
needed.

You have several options in PicoLisp. For simple cases, you can take
advantage of the static scope of \hyperref[ref.html-transient-io]{transient symbols}. For the general case, PicoLisp has built-in functions like
\texttt{bind} or \texttt{job}, which dynamically manage statically scoped
environments.

Environments are first-class objects in PicoLisp, more flexible than
hard-coded closures, because they can be created and manipulated
independently from the code.

As an example, consider a currying function:


\begin{wideverbatim}
(de curry Args
   (list (car Args)
      (list 'list
         (lit (cadr Args))
         (list 'cons ''job
            (list 'cons
               (list 'lit (list 'env (lit (car Args))))
               (lit (cddr Args)) ) ) ) ) )
\end{wideverbatim}

When called, it returns a function-building function which may be
applied to some argument:


\begin{wideverbatim}
: ((curry (X) (N) (* X N)) 3)
-> ((N) (job '((X . 3)) (* X N)))
\end{wideverbatim}

or used as:


\begin{wideverbatim}
: (((curry (X) (N) (* X N)) 3) 4)
-> 12
\end{wideverbatim}

In other cases, you are free to choose a shorter and faster solution. If
(as in the example above) the curried argument is known to be immutable:


\begin{wideverbatim}
(de curry Args
   (list
      (cadr Args)
      (list 'fill
         (lit (cons (car Args) (cddr Args)))
         (lit (cadr Args)) ) ) )
\end{wideverbatim}

Then the function built above will just be:


\begin{wideverbatim}
: ((curry (X) (N) (* X N)) 3)
-> ((X) (* X 3))
\end{wideverbatim}

In that case, the ``environment build-up'' is reduced by a simple
(lexical) constant substitution with zero runtime overhead.

Note that the actual \texttt{curry} function is simpler and more pragmatic. It
combines both strategies (to use \texttt{job}, or to substitute), deciding at
runtime what kind of function to build.

 
\section{Do you have macros?}
\label{sec:faq-do-you-have-macros?}


Yes, there is a macro mechanism in PicoLisp, to build and immediately
execute a list of expressions. But it is seldom used. Macros are a
kludge. Most things where you need macros in other Lisps are directly
expressible as functions in PicoLisp, which (as opposed to macros) can
be applied, passed around, and debugged.

 
\section{Why are there no strings?}
\label{sec:faq-why-are-there-no-strings?}


Because PicoLisp has something better:
\hyperref[ref.html-transient-io]{Transient symbols}. They look and behave like
strings in any respect, but are nevertheless true symbols, with a value
cell and a property list.

This leads to interesting opportunities. The value cell, for example,
can point to other data that represent the string's translation. This is
used extensively for localization. When a program calls


\begin{wideverbatim}
(prinl "Good morning!")
\end{wideverbatim}

then changing the value of the symbol \texttt{''Good morning!''} to its
translation will change the program's output at runtime.

Transient symbols are also quite memory-conservative. As they are stored
in normal heap cells, no additional overhead for memory management is
induced. The cell holds the symbol's value in its CDR, and the tail in
its CAR. If the string is not longer than 7 bytes, it fits (on the
64-bit version) completely into the tail, and a single cell suffices. Up
to 15 bytes take up two cells, 23 bytes three etc., so that long strings
are not very efficient (needing twice the memory on the average), but
this disadvantage is made up by simplicity and uniformity. And lots of
extremely long strings are not the common case, as they are split up
anyway during processing, and stored as plain byte sequences in external
files and databases.

Because transient symbols are temporarily interned (while \texttt{load} ng the
current source file), they are shared within the same source and occupy
that space only once, even if they occur multiple times within the same
file.

 
\section{What about arrays?}
\label{sec:faq-what-about-arrays?}


PicoLisp has no array or vector data type. Instead, lists must be used
for any type of sequentially arranged data.

We believe that arrays are usually overrated. Textbook wisdom tells that
they have a constant access time O(1) when the index is known. Many
other operations like splits or insertions are rather expensive. Access
with a known (numeric) index is not really typical for Lisp, and even
then the advantage of an array is significant only if it is relatively
long. Holding lots of data in long arrays, however, smells quite like a
program design error, and we suspect that often more structured
representations like trees or interconnected objects would be better.

In practice, most arrays are rather short, or the program can be
designed in such a way that long arrays (or at least an indexed access)
are avoided.

Using lists, on the other hand, has advantages. We have so many
concerted functions that uniformly operate on lists. There is no
separate data type that has to be handled by the interpreter, garbage
collector, I/O, database and so on. Lists can be made circular. And
lists don't cause memory fragmentation.

 
\section{How to do floating point arithmetics?}
\label{sec:faq-how-to-do-floating-point-arithmetics?}


PicoLisp does not support real floating point numbers. You can do all
kinds of floating point calculations by calling existing library
functions via \texttt{native}, inline-C code, and/or by loading the
``@lib/math.l'' library.

But PicoLisp has something even (arguably) better: Scaled
\hyperref[ref.html-num-io]{fixpoint numbers}, with unlimited precision.

The reasons for this design decision are manifold. Floating point
numbers smack of imperfection, they don't give ``exact'' results, have
limited precision and range, and require an extra data type. It is hard
to understand what really goes on (How many digits of precision do we
have today? Are perhaps 10-byte floats used for intermediate results?
How does rounding behave?).

For fixpoint support, the system must handle just integer arithmetics,
I/O and string conversions. The rest is under programmer's control and
responsibility (the essence of PicoLisp).

Carefully scaled fixpoint calculations can do anything floating points
can do.

 
\section{What happens when I locally bind a symbol which has a function
definition?} 
\label{sec:faq-what-happens-when-i-locally-bind-a-symbol-which-has-a-function}

That's not a good idea. The next time that function gets executed within
the dynamic context the system may crash. Therefore we have a convention
to use an upper case first letter for locally bound symbols:


\begin{wideverbatim}
(de findCar (Car List)
   (when (member Car (cdr List))
      (list Car (car List)) ) )
\end{wideverbatim}

;-)

 
\section{Would it make sense to build PicoLisp in hardware?}
\label{sec:faq-would-it-make-sense-to-build-picolisp-in-hardware?}


At least it should be interesting. It would be a machine executing list
(tree) structures instead of linear instruction sequences. ``Instruction
prefetch'' would look down the CAR- and CDR-chains, and perhaps need only
a single cache for both data and instructions.

Primitive functions like \texttt{set}, \texttt{val}, \texttt{if} and \texttt{while}, which are
written in \texttt{C} or assembly language now, would be implemented in
microcode. Plus a few I/O functions for hardware access. \texttt{EVAL} itself
would be a microcode subroutine.

Only a single heap and a single stack is needed. They grow towards each
other, and cause garbage collection if they get too close. Heap
compaction is trivial due to the single cell size.

There would be no assembly-language. The lowest level (above the
hardware and microcode levels) are s-expressions: The machine language
is \emph{Lisp}.

 
\section{I get a segfault if I \ldots{}}
\label{sec:faq-i-get-a-segfault-if-i}


PicoLisp is a pragmatic language. It doesn't check at runtime for all
possible error conditions which won't occur during normal usage. Such
errors are usually detected quickly at the first test run, and checking
for them after that would just produce runtime overhead.

Catching the segfault signals is also not a good idea, because the Lisp
heap is most probably be damanged afterwards.

It is recommended, though, to inspect the code periodically with \texttt{lint}.

  
\section{Where can I ask questions?}
\label{sec:faq-where-can-i-ask-questions?}


The best place is the
\href{mailto:picolisp@software-lab.de?subject=Subscribe}{PicoLisp Mailing List} (see also
\href{http://www.mail-archive.com/picolisp@software-lab.de/}{The Mail Archive}), or the IRC \texttt{\#picolisp}
channel on FreeNode.net.

