\title{Pilog Solve and the +Aux Relation}
\author{Henrik Sarvell}
% Use \authorrunning{Short Title} for an abbreviated version of
% your contribution title if the original one is too long
\institute{\texttt{hsarvell@gmail.com}}
%
% Use the package "url.sty" to avoid
% problems with special characters
% used in your e-mail or web address
%


\maketitle


\begin{abstract}
  This article uses the 'Doctrine' example (known from the PHP world)
  to demonstrate the usage (and advantages) of Pilog \texttt{solve} and
  the \texttt{+Aux} relation. 
\end{abstract}

\section{'Doctrine for dummies' example}
\label{sec:pilog-solve-'doctrine-for-dummies'-example}


I just set out to duplicate the
\href{http://www.prodevtips.com/2008/08/05/doctrine-for-dummies/}{Doctrine
  for dummies example}, but this time for real, with a real OODB
system, not some silly ORM. Thanks goes to Alex for helping me out with
the queries.


\begin{wideverbatim}
(class +Member +Entity)
(rel uname  (+Need +Key +String))
(rel pwd    (+Need +String))
(rel email  (+String))
(rel city   (+Ref +Link) NIL (+City))

(class +City +Entity)
(rel name   (+Key +String))

(class +Message +Entity)
(rel subject (+Idx +String))
(rel body    (+String))
(rel from    (+Ref +Link) NIL (+Member))
(rel to      (+Aux +Ref +Link) (from) NIL (+Member))

(pool "dbtest.db")
\end{wideverbatim}

So we create the relations, and the database file. Take special note of
the \textbf{+Aux} relation from \textbf{to} to \textbf{from}. It will play a central role
later.


\begin{wideverbatim}
(setq Mbrs 
  (list
     (list "member1" "password1" "member1@members.com" "Berlin")
     (list "member2" "password2" "member2@members.com" "Stuttgart")
     (list "member3" "password3" "member3@members.com" "Hamburg")))
  
(for Mbr Mbrs
  (request '(+Member) 
     'uname (get Mbr 1) 
     'pwd   (get Mbr 2) 
     'email (get Mbr 3) 
     'city  (request '(+City) 'name (get Mbr 4))))
\end{wideverbatim}

We input some members in the database, note the use of \textbf{request}. The
major thing is that it will create an object if there is none with the
given key(s) already, however if there is an object with the given keys
it will return that object, quite handy.


\begin{wideverbatim}
(setq Mbr1 (db 'uname '+Member "member1"))
(setq Mbr2 (db 'uname '+Member "member2"))

(new T '(+Message) 
  'subject "from mbr1 to mbr2"
  'body "Hello mbr2 this is mbr1"
  'to Mbr2
  'from Mbr1)
  
(new T '(+Message) 
  'subject "from mbr2 to mbr1"
  'body "Hello mbr1 this is mbr2"
  'to Mbr1
  'from Mbr2)
  
(new T '(+Message) 
  'subject "from mbr2 to mbr1, again"
  'body "Hello mbr1 this is mbr2, again"
  'to Mbr1
  'from Mbr2)
  
\end{wideverbatim}

\begin{wideverbatim}

(new T '(+Message) 
  'subject "from mbr1 to mbr3"
  'body "Hello mbr3 this is mbr1"
  'to (db 'uname '+Member "member3")
  'from Mbr1)
  
(commit)
\end{wideverbatim}

Note the use of \textbf{(new T\dots{})} without the \texttt{T} we won't
get database objects.


\section{Querying}
\label{sec:pilog-solve-querying}

\subsection{Simple queries}
\label{sec:pilog-solve-simple-queries}

\begin{wideverbatim}
(mapc show (collect 'to '+Message Mbr1 Mbr1 'from))
(mapc show (collect 'from '+Message Mbr1 Mbr1 'to))
(mapc show (collect 'from '+Message Mbr1 Mbr1))
(mapc show (collect 'to '+Message Mbr1 Mbr1))
\end{wideverbatim}

Test that everything is there by calling the above statements, the first
one will get all people who sent a message to member 1. The second one
will instead get all people who member 1 sent messages to. The two last
ones will do the same but will get messages, not people. And that was
that, this is as complex as the PHP Doctrine example got, let's move on
to more complex stuff relating to the +Aux relation:

\subsection{Using the \texttt{+Aux} relation}
\label{sec:pilog-solve-using-the-+aux}

\begin{wideverbatim}
(scan (tree 'to '+Message))
(scan (tree 'from '+Message))
\end{wideverbatim}

Output:

\begin{wideverbatim}
({5} {:} . {H}) {H}
({5} {:} . {I}) {I}
({:} {5} . {B}) {B}
({A} {5} . {L}) {L}

({5} . {B}) {B}
({5} . {L}) {L}
({:} . {H}) {H}
({:} . {I}) {I}
-> {F}
\end{wideverbatim}

Notice the difference, the \textbf{to} relation is utilizing \textbf{+Aux} so we have
extra references there between the message receiver and sender. Together
the combination can be used as a key to speed up things.


\begin{wideverbatim}
(mapc show (collect 'to '+Message (list Mbr1 Mbr2)))
\end{wideverbatim}

This is an example of how to use the \texttt{+Aux} relation to get all messages
to member 1 from member 2 with \textbf{collect}.


\subsection{Pilog \texttt{solve} with parallel scanning}
\label{sec:pilog-solve-using-the-+aux}

\begin{wideverbatim}
(mapc show
   (solve  
     (quote @M1 Mbr1 @M2 Mbr2 @S "mbr2" 
        (select (@Msgs) 
           ((subject +Message @S) (to +Message @M1) (from +Message @M2))
           (same @M1 @Msgs to)
           (same @M2 @Msgs from)
           (tolr @S @Msgs subject)))  
     @Msgs ))
\end{wideverbatim}

The above code will retrieve all messages from member 2 to member 1
that also contain the word “\texttt{mbr2}″ in the subject. It is
however not using the \texttt{+Aux} relation to make the lookup of who
is sending a message to who, we scan in parallel here.


\subsection{Pilog \texttt{solve} using the \texttt{+Aux} relation}
\label{sec:pilog-solve-using-the-+aux}


\begin{wideverbatim}
(mapc show 
   (solve
      (quote
         @M1 Mbr1
         @M2 Mbr2
         @M (list Mbr1 Mbr2)
         @S "mbr2"
         (select (@Msgs)
            ((subject +Message @S) (to +Message @M))
            (tolr @S @Msgs subject)
            (same @M1 @Msgs to)
            (same @M2 @Msgs from))) 
      @Msgs ))
\end{wideverbatim}

This example is however using the \texttt{+Aux} relation. The
\textbf{@M} list will look for for example the \textbf{\{5\} \{:\}}
combo above in the \textbf{(to +Message @M)} clause. Next we filter as
usual.

After benchmarking the above examples I got a 3\% speed increase in favor
of the second one using 6000 messages to test on, when using 11000 the
difference increased to 11\% so careful planning will pay off more and
more the bigger the database becomes.
