\title{PicoLisp and JSON}
\author{Henrik Sarvell}
% Use \authorrunning{Short Title} for an abbreviated version of
% your contribution title if the original one is too long
\institute{\texttt{hsarvell@gmail.com}}
%
% Use the package "url.sty" to avoid
% problems with special characters
% used in your e-mail or web address
%


\maketitle


\begin{abstract}
This article discusses libraries and tests for converting PicoLisp to
JSON and vice versa.   
\end{abstract}

\section{Introduction}
\label{sec:pl-json-introduction}

Yet again I have to do some documenting so I know what the hell I'm
doing since I'm all over the place at the moment. Doing something here
and then moving over to do something over there and then coming back to
coding this and that. If you're this unstructured you need crutches and
this documentation is that, it will enable me to get back to this and do
easy debugging in the future, it will trigger memories.

Therefore this code is very rough and untested, a work in progress, you
have been warned\dots{}

If you somehow landed on this page without any background on PicoLisp
or Lisp you probably need to start
\href{http://www.prodevtips.com/2008/03/28/pico-lisp/}{from the beginning}.

\section{The tests}
\label{sec:pl-json-the-tests}

\subsection{PicoLisp to JSON}
\label{sec:pl-json-picolisp-to-json}

Let's start with the tests for a change. This is an example of
converting a proper database object to JSON. The code will determine the
relations and use them to build the JSON, nothing else that might be in
there:


\begin{wideverbatim}
(load "lib/str.l")
(load "lib/json.l")

(class +Product +Entity)
(rel name (+Need +String))
(rel id (+Need +Number))
(rel descr (+String))
(rel attributes (+List +String))

(setq Product
   (new '(+Product) 'name
      "A \"PC\"" 'id 123 'attributes '("black" "laptop") ) )

(println (to> '+Json 'encObj> Product))
\end{wideverbatim}

Output:

\begin{wideverbatim}
  "{\"attributes\": [\"black\", \"laptop\"], \"id\": 123,
    \"name\": \"A \\\"PC\\\"\", \"descr\": false}"
\end{wideverbatim}

\textbf{Associative structure}, it will also be encoded as object:

\begin{wideverbatim}
(setq Pairs '((key1 . hello) (key2 . world) (false . NIL)
              (someArr . (1 2 "hello quote: \"quote\"" 4)) ) )
(println (to> '+Json 'encPair> Pairs))
\end{wideverbatim}


\begin{wideverbatim}
  "{\"key1\": \"hello\", \"key2\": \"world\", \"false\": false,
    \"someArr\": [1, 2, \"hello quote: \\\"quote\\\"\", 4]}"
\end{wideverbatim}

\textbf{2D structure} will be an array of objects:

\begin{wideverbatim}
(setq Pair1 '((key1 . hello) (key2 . world) (false . NIL) (someArr . (1 2 "hello" 4))))
(setq Pair2 '((key1 . hello) (key2 . world) (false . NIL) (someArr . (1 2 "world" 4))))
(setq Tst (list Pair1 Pair2))
(println (to> '+Json 'encTable> Tst))
\end{wideverbatim}


\begin{wideverbatim}
  "[{\"key1\": \"hello\", \"key2\": \"world\", \"false\": false,
    \"someArr\": [1, 2, \"hello\", 4]}, 
  {\"key1\": \"hello\", \"key2\": \"world\", \"false\": false,
    \"someArr\": [1, 2, \"world\", 4]}]"
\end{wideverbatim}

Simple case of \textbf{nested list}:

\begin{wideverbatim}
(setq Tst (list 1 2 "hello" (1 2 3) 3 4 "world"))
(println (to> '+Json 'encArr> Tst))
\end{wideverbatim}


\begin{wideverbatim}
"[1, 2, \"hello\", [1, 2, 3], 3, 4, \"world\"]"
\end{wideverbatim}

\subsection{JSON to PicoLisp}
\label{sec:pl-json-json-to-picolisp}

\begin{wideverbatim}
(setq Json "{\"hello1\": {\"subObj\": [123, 456, true, NIL]}, \"b\": true}")
(setq Result (from> '+Json Json))
(show Result)
(show (get Result 'hello1))
\end{wideverbatim}


\begin{wideverbatim}
$385543015 NIL
   b
   hello1 $385543062
$385543062 NIL
   subObj (123 456 T NIL)
\end{wideverbatim}

\section{The library}
\label{sec:pl-json-the-library}


\subsection{JSON to PicoLisp}
\label{sec:pl-json-json-to-picolisp}



\begin{wideverbatim}
(class +Json)

(dm from> (J)
  (=: L (chop J))
  (let C (pop (:: L))
     (let R (if (= C "[") (pre> This 'pArr>) (pre> This 'pObj>))        
        (parse> This R))))
\end{wideverbatim}

This is where the coding from JSON to PicoLisp structure begins.


\begin{wideverbatim}
(dm pre> (Type)
  (let (R (list Type) InStr NIL)
     (catch NIL 
        (while (: L)
           (let C (pop (:: L))
              (cond
                 ((= C "[") (queue 'R (pre> This 'pArr>)))
                 ((= C "{") (queue 'R (pre> This 'pObj>)))
                 ((and (or (= C "]") (= C "}")) (nT InStr)) (throw))
                 (T (when (= C "\"") 
                          (setq InStr (not InStr))) 
                       (queue 'R C)))))) R ))
\end{wideverbatim}

Here we are creating an intermediary list that will be easy to
execute. We do this by inserting the names of functions to use in
later steps into this list. But what is really happening? Well as you
saw from the prior listing we begin with either
\textbf{pArr\textgreater{}} or \textbf{pObj\textgreater{}} depending
on if we are to begin parsing to an object or an array. \textbf{InStr}
will keep track of whether the characters \textbf{\{, \}, [, ]} are
inside a string or not, if they are they should not count of course.

So while we still have characters in our chopped up list we will loop
through them by destructively popping, if we have a ``[`` we will put
\textbf{pArr\textgreater{}} on the list instead of the caracter, if we
have ``\{'' we put \textbf{pObj\textgreater{}}. If we have the
respective closing character, and it is not inside a string, we exit
by throwing \texttt{NIL}.


\begin{wideverbatim}
(dm any> (L)
   (let R (any (pack L))      
      (if (= R "true") T (if (= R "false") NIL R))))

(dm parse> (L)  
  (apply (car L) (list This (cdr L))))
    
(dm pObj> (L)
  (let (R (new) L (split L ",")) 
     (for El L
        (let Pair (split El ":")
           (put R 
              (any (any (pack (car Pair)))) 
              (let Value (cdadr Pair)                 
                 (if (lst? (car Value))
                    (parse> This (car Value))
                    (any> This Value)))))) R ))
\end{wideverbatim}

We begin by applying either \textbf{pObj\textgreater{}} or
\textbf{pArr\textgreater{}} in \textbf{parse\textgreater{}}.

The \texttt{pObj>} method will begin with creating the empty result
object, \textbf{R} and a list of sublists looking something like this:
(``k'' ``e'' ``y'' ``:'' ``v'' ``a'' ``l'' ``u'' ``e'') in \textbf{L} by
splitting by ``,''.

We continue by splitting by ``:'' to get the key and the value. The key is
then retrieved, the value will be further examined to determine if we
should apply recursion to get a sub-object/array or simply return the
result through \textbf{any\textgreater{}}.

\begin{wideverbatim}
(dm pArr> (L)
   (make       
      (for El (mapcar 'clip (split L ","))         
         (if (lst? (car El))
            (link (parse> This (car El)))
            (link (any> This El))))))
\end{wideverbatim}


\subsection{PicoLisp to JSON}
\label{sec:pl-json-picolisp-to-json}


\begin{wideverbatim}
(dm to> (F L)
   (pack (make (apply F (list This L)))))
\end{wideverbatim}

As you know from the tests above the behavior has to be explicitly set
by passing the function name to be used to generate the result when
going from PicoLisp to Javascript. It's pretty obvious actually since
it's impossible to determine from the structure of various types of
lists how to treat them. We can't infer whether a list is a normal
nested list or a paired list, for all intents and purposes they are
identical. However the output will be radically different. Note
\textbf{make}, that is why we are able to use \textbf{link} all over
the place below.


\begin{wideverbatim}
(dm encTable> (Tbl)
  (link "[")
  (let F T 
     (mapc 
        '((L) 
            (link (comma> This F)) 
            (encPair> This L) 
            (setq F NIL)) Tbl))
  (link "]"))

(dm encPair> (L)    
  (link "{")
  (let F T 
     (mapc 
     '((El) 
         (link (pack (comma> This F) "\"" (car El) "\"" ":" " "))
         (setq F NIL)                  
         (enc> This (if (pair El) (cdr El) NIL))) L ))
  (link "}"))

(dm comma> (First)
   (unless First ", "))
\end{wideverbatim}

Some redundant code here, it might benefit from refactoring, or we
could just leave it like it is and call it a day, yeah let's do that.
When encoding a table each element will in turn be encoded with
\textbf{encPair\textgreater{}}, if we have the first element we do not
prepend the ``,''.

A paired list will be encoded as an object with
\textbf{encPair\textgreater{}}.


\begin{wideverbatim}
(dm encArr> (L)  
  (link "[")
  (let F T
     (mapc 
     '((El) 
         (link (pack (comma> This F)))
         (setq F NIL)
         (enc> This El)) L ))
  (link "]"))
\end{wideverbatim}

Redundancy again! \texttt{List -> array} is easier though than
\texttt{paired list -> object}.


\begin{wideverbatim}
(dm encObj> (O)
  (encPair> This 
     (make 
        (mapc 
           '((Prop)
               (when (isa '+Relation (car Prop)) 
                  (let Key (cdr Prop) 
                     (link (cons Key (get O Key)))))) (getl (car (type O)))))))
\end{wideverbatim}

Finally something clever, in case of object we will get all the
properties of the object through the \texttt{getl} function. Every
property that is a relation will get the ``treatment''. We get the
name of the relation as \textbf{Key} and use that name on the original
object to retrieve the value. The resultant array is now a paired list
that can be encoded with \textbf{encObj\textgreater{}}.


\begin{wideverbatim}
(dm enc> (L)   
   (cond 
      ((=T L)    (link "true"))
      ((not L)   (link "false"))

      ((num? L)  (link L))
      ((lst? L)  (encArr> This L))
      ((or (box? L) (ext? L)) (encObj> This L))))
\end{wideverbatim}

accordingly, notice the escaping with \textbf{+Str}. It's the genesis of some
kind of general string library, not much in there yet though:

\begin{wideverbatim}

(dm fChr> (Lst Chr)
  (find '((C)(= C Chr)) Lst))
   
#S = hello, Lst = '("\"")
(dm esc> (S Lst)
  (pack 
     (mapcar 
        '((C)
            (if (fChr> This Lst C) (pack "\\" C) C )) (chop S))))
\end{wideverbatim}

Every character in the passed list (in this case only one) will be
escaped.

