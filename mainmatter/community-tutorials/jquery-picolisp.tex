\title{jQuery and PicoLisp}
\author{Henrik Sarvell}
% Use \authorrunning{Short Title} for an abbreviated version of
% your contribution title if the original one is too long
\institute{\texttt{hsarvell@gmail.com}}
%
% Use the package "url.sty" to avoid
% problems with special characters
% used in your e-mail or web address
%


\maketitle

\begin{abstract}
  This article describes problems arising when trying to make
  \texttt{jQuery} and \texttt{PicoLisp} work together, as well as
  possible solutions and their implementation. 
\end{abstract}

\section{Problem}
\label{sec:jquery-picolisp}

The heart of the problem with making \texttt{jQuery.post} work with
the PicoLisp server is simple once found (thanks to Alex for helping
me find it). Apache seems to use the Content-Length header to
determine the length of the argument sent by
\texttt{XMLHttpRequest.send()}, the PicoLisp server doesn't bother with
determining the the length. It looks for a \texttt{newline} at the end instead.

\section{Solution}
\label{sec:dsf}

\subsection{Description}
\label{sec:ds}

How do we solve this problem considering that \texttt{\$.ajax} does
not append a newline? With my abysmal knowledge of OO programming in
Javascript (all that prototyping makes my head hurt) I've opted for
the simplest and dirtiest solution. A simple copy paste of
\texttt{\$.post} and \texttt{\$.ajax} implemented as a separate plugin
to enable effortless upgrades of the core, the new stuff can be called
with \texttt{\$.pico.post} and works just like the original.

The only change in the plugin compared with the original is line 2806
in \texttt{jquery.js}, it looks like this in the original:
\textbf{xhr.send(s.data);}, I've changed that to
\textbf{xhr.send(s.data + `\r\n');}.

\subsection{Implementation}
\label{sec:dfs}

My current application renders the HTML like this:

\begin{wideverbatim}
(de js (JS)
   (prinl "<script type=\"text/javascript\" src=\""
      (pack *BP "js/" JS) "\"></script>" ) )

(de rss-html Prg
   (html 0 "RSS Reader" *Css NIL
      (js "jquery.js")
      (js "jquery.pico.js")
      (js "rss-reader.js")
      (<div> 'page_margins
         (<div> 'page 
            (<div> 'header "header")
            (<div> 'main
               (<div> 'left 
                  (<div> 'left_content (leftContent)))
               (<div> 'middle 
                  (<div> 'middle_content (run Prg 1)))
               (<div> 'right 
                  (<div> 'right_content "right"))
               (<div> 'clear)))
         (<div> 'footer "footer"))))
\end{wideverbatim}

So we include the new plugin in the form of \textbf{(js ``jquery.pico.js'')},
after the base library.

The rss-reader.js file now contains this:


\begin{wideverbatim}
$(document).ready(function(){
  $(".articles_link").css('cursor', 'pointer').click(function(){
    $.pico.post("@ajaxTest", {jquerytest: "test"}, function(res){
      $(".middle_content").html(res);
    });
  });
});
\end{wideverbatim}

When one of the links are clicked we call \textbf{@ajaxTest:}


\begin{wideverbatim}
(de ajaxTest ()
  (httpHead "text/plain; charset=utf-8")
  (ht:Out T
     (ht:Prin (pack "Result: " (get 'jquerytest 'http)))))
\end{wideverbatim}

And the contents of the div with the class attribute of
``middle\_content'' changes to \textbf{Result: test}, great.

Source:
\href{http://www.prodevtips.com/wp-content/uploads/2008/10/jquerypico.js}{jquerypico.js}

Note that we make use of the httpGate here, the base url in this example
is \href{http://localhost}{http://localhost}, not \href{http://localhost:8080}{http://localhost:8080}.
