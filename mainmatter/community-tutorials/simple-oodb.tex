\title{Simple OODB in PicoLisp}
\author{Henrik Sarvell}
% Use \authorrunning{Short Title} for an abbreviated version of
% your contribution title if the original one is too long
\institute{\texttt{hsarvell@gmail.com}}
%
% Use the package "url.sty" to avoid
% problems with special characters
% used in your e-mail or web address
%


\maketitle


\begin{abstract}
This article first walks through a simple example that shows how a
table in a relational database system like MySQL would be implemented
in PicoLisp's build-in \emph{object-oriented database}. Then it
discusses \emph{external symbols} in PicoLisp.     
\end{abstract}

\section{Walk through a simple  example}
\label{sec:simple-oodb-walk-through-a-simple-example}

Let's walk through a simple and usual example, what would constitute a
user table in MySQL:

\begin{wideverbatim}
(class +User +Entity)
(rel username (+Need +Key +String))
(rel password (+Need +String))

(pool "users.db")

(new! '(+User) 'username "sam" 'password "parrotno5")
(new! '(+User) 'username "fred" 'password "MegaPizza")
(new! '(+User) 'username "anna" 'password "swooosh")
(new! '(+User) 'username "fred" 'password "yiiihaaa")
(new! '(+User) 'username NIL 'password "asdf")

(mapcar show (collect 'username '+User))
\end{wideverbatim}

Output:

\begin{wideverbatim}
{6} (+User)
   password "swooosh"
   username "anna"
{5} (+User)
   password "MegaPizza"
   username "fred"

\end{wideverbatim}

\begin{wideverbatim}

{2} (+User)
   password "parrotno5"
   username "sam"
-> ({6} {5} {2})

\end{wideverbatim}

That was quite a lot at the same time. All classes that are to
generate objects stored in the database needs to be children of
\texttt{+Entity}. Furthermore some relations are needed in the form of
prefix classes, \textbf{rel} will in this case take the name of the
relation, username, and the list of classes that will define the
behavior of the relation. Prefix classes has been explained in the
prior tutorial.

In the case of the username we have \texttt{+Need} which denotes that
this relation is needed for successful creation of the persistent
object. As you can see in the output the last call to \textbf{new!}
(only difference from \textbf{new} is that we create the object in a
file instead of in the RAM) never resulted in an object on disc since
no key was created. In our case the username will be used as key and
needs to be unique, hence no second ``fred'' in the output. Of course
both username and password will both be strings. There is a short
description of more relations in the
\href{http://www.software-lab.de/ref.html#dbase}{reference}.

\section{External symbols}
\label{sec:simple-oodb-external-symbols}

Instead of the above run this (don't delete \texttt{users.db}!):

\begin{wideverbatim}
(class +User +Entity)
(rel username (+Need +Key +String))
(rel password (+Need +String))

(pool "users.db")

(show '{2})
\end{wideverbatim}


This will give us ``sam'' which means that he is directly accessible
through \texttt{\{2\}} which
\href{http://www.software-lab.de/ref.html#external-io}{Alex} explains
better than me:

\begin{quote}
  External symbol names are surrounded by braces (`\texttt{\{'} and
  \texttt{`\}}'). The characters of the symbol's name itself identify
  the physical location of the external object. This is currently the
  number of the starting block in the database file, encoded in
  base--64 notation (characters `\texttt{0}′ through `\texttt{9}',
  `\texttt{:}' through `\texttt{;}', `\texttt{A}' through `\texttt{Z}'
  and `\texttt{a}' through `\texttt{z}').
\end{quote}

Instead of \texttt{(show '\{2\})} try:


\begin{wideverbatim}
(show (db 'username '+User "sam"))
\end{wideverbatim}

This is basically the equivalent of:

\begin{wideverbatim}
SELECT * FROM `user` WHERE BINARY username = 'sam'
\end{wideverbatim}

And

\begin{wideverbatim}
(show (db 'username '+User "fred" 'password "MegaPizza"))
\end{wideverbatim}

is of course the same as the login SQL:

\begin{wideverbatim}
SELECT * FROM `user` WHERE BINARY username = 'fred' 
AND BINARY password = 'MegaPizza'
\end{wideverbatim}
