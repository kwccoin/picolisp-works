\title{Simple OO in PicoLisp}
\author{Henrik Sarvell}
% Use \authorrunning{Short Title} for an abbreviated version of
% your contribution title if the original one is too long
\institute{\texttt{hsarvell@gmail.com}}
%
% Use the package "url.sty" to avoid
% problems with special characters
% used in your e-mail or web address
%


\maketitle

\begin{abstract}
  This article describes the fundamentals of \emph{object-oriented
  programming} in PicoLisp: defining classes, creating instances, as
  well as fetching from and sorting a list of objects. 
\end{abstract}

\section{Defining classes}
\label{sec:simple-oo-defining-classes}

PicoLisp has a very nice object system which will take some time to
explore, let's begin with simple examples and work towards more complex
scenarios.


\begin{wideverbatim}
(class +Person)

(dm hello> ()
    (prin "hello"))
 
(hello> '+Person)
\end{wideverbatim}

In this example we do not instantiate an object, in PHP the last call
would correspond to \texttt{Person::hello()}.


\begin{wideverbatim}
(class +Person)

(dm setHello> (HelloPhrase)
    (=: hello HelloPhrase))

(dm hello> ()
    (prin (: hello)))
    
(setHello> '+Person "Hello how are you?")

(hello> '+Person)
\end{wideverbatim}

There are two shortcuts at work here, \textbf{=:} will \textbf{put} the contents of
HelloPhrase into the member variable hello, \textbf{:} will in turn \textbf{get} it.


\section{Creating instances}
\label{sec:simple-oo-creating-instances}

Let's start creating instances:

\begin{wideverbatim}
(class +Person)

(dm T (Age Name)
    (=: age Age)
    (=: name Name))

(setq *John (new '(+Person) 65 'John))

(show *John)
\end{wideverbatim}

So the first argument to \textbf{new} is a quoted list with the class
we want to use. The constructor is the \texttt{T} method, that is a
requirement, it always has to be called \texttt{T} for PicoLisp to
notice it correctly.

Add the above hello functions to the John code above and try:

\begin{wideverbatim}
(setHello> *John "Hello how are you?")

(hello> *John)
\end{wideverbatim}

As you can see the result is the same, in this regard PicoLisp is similar to
PHP:

\begin{wideverbatim}
class Person{

  static $hello;
  
  function __construct($age, $name){
    $this->age = $age;
    $this->name = $name;
  }
  
  static function setHello($HelloPhrase){
    self::$hello = $HelloPhrase;
  }
  
  static function hello(){
    echo self::$hello;
  }
}

\end{wideverbatim}

\begin{wideverbatim}


$john = new Person(65, "John");

$john->setHello("Hello how are you?");

$john->hello();

Person::hello();
\end{wideverbatim}

The only difference is that PHP requires us to declare the
\texttt{\$hello} variable in order for it to be used in subsequent
functions. A requirement which makes static usage less useful in PHP
than in PicoLisp.

\section{Fetch from and sort a list of objects}
\label{sec:simple-oo-fetch-from-and-sort-a-list-of-objects}


Let's see how we can fetch from and sort a list of persons:

\begin{wideverbatim}
(setq *Persons (list 
               (new '(+Person) 65 'John) 
               (new '(+Person) 38 'Fred)
               (new '(+Person) 41 'Annie) 
               (new '(+Person) 42 'Sam)))
\end{wideverbatim}

This is actually how it could look after you get a list of people from a
database which makes this example more useful than the stuff we did in
the prior tutorial in this series.


\begin{wideverbatim}
(de assoc2d (Lst Key Value)
    (filter '((Sub)(= (get Sub Key) Value)) Lst))

(show (car (assoc2d *Persons 'name 'John)))
\end{wideverbatim}

As you can see \textbf{get} can be used to get member variables in objects as
well as by index as in the prior tutorial.

Sorting is similar to what we have already done in the prior part, in
fact \texttt{getSorting} doesn't have to be changed at all, neither
does \texttt{sort2d}:


\begin{wideverbatim}
(de getCol (Lst K) 
    (mapcar '((Sub)(get Sub K)) Lst ) ) 

(de getSorting...

(de sort2d...

(show (car (sort2d *Persons 'name)))
\end{wideverbatim}

Try removing the car in the last call, you will get a list looking
something like \texttt{(\$34567855 \$68904356 \$21345679 \$56854378)}.
Show will only print their addresses when accessing objects indirectly
as is the case here when they are in a list. I leave it as an exercise
to write a function that loops through the list and shows each person.

