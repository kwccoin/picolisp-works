\title*{More OODB in PicoLisp}
\author{Henrik Sarvell}
% Use \authorrunning{Short Title} for an abbreviated version of
% your contribution title if the original one is too long
\institute{\texttt{hsarvell@gmail.com}}
%
% Use the package "url.sty" to avoid
% problems with special characters
% used in your e-mail or web address
%


\maketitle

\section{More OODB in Pico Lisp}
\label{sec:more-oodb}

This time we will examine some more complex relationships with a common
scenario, the CMS.

Without deleting the file (users.db, yes the name is no longer
representative but we don't care) created in the prior article do the
following:


\begin{wideverbatim}
(class +User +Entity)
(rel username (+Need +Key +String))
(rel password (+Need +String))

(class +Article +Entity)
(rel slug     (+Need +Key +String))
(rel headline (+Need +Idx +String))
(rel body     (+Need +String))
(rel author   (+Link)(+User))

(pool "users.db")

(obj ((+Article) slug "valves-new-strategy") 
     headline "Valve's new strategy" 
     body "An article about a computer game company." 
     author `(db 'username '+User "sam"))

(obj ((+Article) slug "diving-palau") 
     headline "Diving Palau" 
     body "An article about the diving in Palau." 
     author `(db 'username '+User "fred"))

(obj ((+Article) slug "the-coolest-gadgets-in-2008") 
     headline "The coolest gadgets in 2008" 
     body "An article about all the new gadgets unleashed in 2008." 
     author `(db 'username '+User "fred"))

(commit)

(mapcar show (collect 'slug '+Article))
\end{wideverbatim}

Obviously we are fetching objects from the database with our \textbf{db} calls
above, the \textbf{`} read macro will evaluate its list and return the result,
in this case the objects \{5\} and \{2\} (Fred and Sam). \textbf{commit} will save
our newly created objects to the database file.

The new thing here is +Link which will function as a reference to an
object, in our case a user, the author of the article.

Output:


\begin{wideverbatim}
{;} (+Article)
   author {5}
   body "An article about the diving in Palau."
   headline "Diving Palau"
   slug "diving-palau"
{C} (+Article)
   author {5}
   body "An article about all the new gadgets unleashed in 2008."
   headline "The coolest gadgets in 2008"
   slug "the-coolest-gadgets-in-2008"
{7} (+Article)
   author {2}
   body "An article about a computer game company."
   headline "Valve's new strategy"
   slug "valves-new-strategy"
-> ({;} {C} {7})
\end{wideverbatim}

This setup will make it easy for an article to retrieve its author, a
common operation in a CMS. Remove the object creation part and replace
the mapcar in the above listing with:


\begin{wideverbatim}
(setq *CurArticle (db 'slug '+Article "diving-palau"))
(show (get *CurArticle 'author))
\end{wideverbatim}

And we get Fred. However in order for Fred to be able to fetch all
articles written by him we need to change the model (you can delete
user.db now):


\begin{wideverbatim}
(class +User +Entity)
(rel username (+Need +Key +String))
(rel password (+Need +String))

(class +Article +Entity)
(rel slug     (+Need +Key +String))
(rel headline (+Need +Idx +String))
(rel body     (+Need +String))
(rel author   (+Ref +Link) NIL (+User))

(pool "cms.db")

#The new! lines here (new! '(+User) 'username "sam" 'password "parrotno5")...

#The obj lines here (obj ((+Article) slug "valves-new-strategy")...

(commit)

(setq *CurUser (db 'username '+User "fred"))
(mapcar show (collect 'author '+Article *CurUser))
\end{wideverbatim}

Don't forget to paste the lines I left out! So the only difference here
is: (rel author (+Ref +Link) NIL (+User)) where +Ref is responsible for
our new ability to fetch articles per user through a new global index.
Without +Ref we just had a reference to a single authour inside each
article. The above collect statement would not have worked with that
setup.

Let's go for a folder based structure a la MODx and TYPO3 by running
this (without deleting cms.db):


\begin{wideverbatim}
(class +User +Entity)
(rel username (+Need +Key +String))
(rel password (+Need +String))

(class +Article +Entity)
(rel slug     (+Need +Key +String))
(rel headline (+Need +Idx +String))
(rel body     (+Need +String))
(rel author   (+Ref +Link) NIL (+User))
(rel folder   (+Ref +Link) NIL (+Article))

(pool "cms.db")

(new! '(+Article) 'slug "tech" 'headline "Technology" 'body "The tech area." 'author (db 'username '+User "sam"))

(mapc '((Slug)
         (let Folder (db 'slug '+Article "tech")
           (put> (db 'slug '+Article Slug) 'folder Folder)))
       '("the-coolest-gadgets-in-2008" "valves-new-strategy"))

(commit)

(mapcar show (collect 'slug '+Article))
\end{wideverbatim}

Notice the use of \textbf{put>}, just using \textbf{put} would not give us the global
reference index, only the single reference as would have been the case
if we weren't using +Ref. Running


\begin{wideverbatim}
(mapcar show (collect 'folder '+Article (db 'slug '+Article "tech")))
\end{wideverbatim}

would have resulted in NIL if you had used only \textbf{put}. However running


\begin{wideverbatim}
(show (get (db 'slug '+Article "the-coolest-gadgets-in-2008") 'folder))
\end{wideverbatim}

would still have showed the tech article. Let's implement tags:


\begin{wideverbatim}
(class +User +Entity)
(rel username (+Need +Key +String))
(rel password (+Need +String))

(class +Article +Entity)
(rel slug     (+Need +Key +String))
(rel headline (+Need +Idx +String))
(rel body     (+Need +String))
(rel author   (+Ref +Link) NIL (+User))
(rel folder   (+Ref +Link) NIL (+Article))
(rel tags     (+List +Ref +Link) NIL (+Tag))

(class +Tag +Entity)
(rel tag (+Need +Key +String))

(pool "cms.db")

(new! '(+Tag) 'tag "scuba")
(new! '(+Tag) 'tag "gadgets")
(new! '(+Tag) 'tag "gaming")
(new! '(+Tag) 'tag "sports")
(new! '(+Tag) 'tag "pc")
(new! '(+Tag) 'tag "iphone")
(new! '(+Tag) 'tag "fun")

(setq *Relations '(("valves-new-strategy" "pc" "fun" "gaming")
                   ("diving-palau" "fun" "scuba" "sports")
                   ("the-coolest-gadgets-in-2008" "gadgets" "fun" "iphone")
                   ("tech" "gadgets" "fun" "iphone" "pc" "gaming")))

(mapc
 '((Rel)
   (let Article (db 'slug '+Article (car Rel))
     (mapc 
      '((Tag)
        (put> Article 'tags (db 'tag '+Tag Tag))) 
      (cdr Rel)))) 
 *Relations)

(commit)

(mapcar show (get (db 'slug '+Article "tech") 'tags))
\end{wideverbatim}

The new thing here is (rel tags (+List +Ref +Link) NIL (+Tag)) which
will create the references, the +List prefix is responsible for handling
the fact that we now have many tags belonging to the same article. The
last line above will show all tags that apply to the article with the
slug ``tech''.


\begin{wideverbatim}
(scan (tree 'tags '+Article))
\end{wideverbatim}

Gives me:


\begin{wideverbatim}
({H} . {A}) {A}
({K} . {D}) {D}
({K} . {F}) {F}
({L} . {7}) {7}
({L} . {F}) {F}
({M} . {A}) {A}
({N} . {7}) {7}
({N} . {F}) {F}
({O} . {D}) {D}
({O} . {F}) {F}
({P} . {7}) {7}
({P} . {A}) {A}
({P} . {D}) {D}
({P} . {F}) {F}
-> {R}
\end{wideverbatim}

And


\begin{wideverbatim}
(mapcar show (collect 'slug '+Article))
\end{wideverbatim}

results in:


\begin{wideverbatim}
{A} (+Article)
   tags ({M} {H} {P})
   author {5}
   body "An article about the diving in Palau."
   headline "Diving Palau"
   slug "diving-palau"
{F} (+Article)
   tags ({L} {N} {O} {P} {K})
   author {2}
   body "The tech area."
   headline "Technology"
   slug "tech"
{D} (+Article)
   tags ({O} {P} {K})
   folder {F}
   author {5}
   body "An article about all the new gadgets unleashed in 2008."
   headline "The coolest gadgets in 2008"
   slug "the-coolest-gadgets-in-2008"
{7} (+Article)
   tags ({L} {P} {N})
   folder {F}
   author {2}
   body "An article about a computer game company."
   headline "Valve's new strategy"
   slug "valves-new-strategy"
-> ({A} {F} {D} {7})
\end{wideverbatim}

As you can see in the above scan of the index tree we have the tag in
the car of the key and the article in the cdr. This will enable us to
get all the articles belonging to a specific tag:


\begin{wideverbatim}
(setq *CurTag (fetch (tree 'tag '+Tag) "pc"))

(mapcar show 
        (make 
         (scan 
          (tree 'tags '+Article) 
          '((Key Value)
            (when (= *CurTag (car Key)) (link Value))))))
\end{wideverbatim}

That conlcudes this piece and we still havent touched
\href{http://www.software-lab.de/ref.html#pilog}{Pilog}. Feel free to go
ahead and study the doc/family.l example and the
\href{http://www.software-lab.de/tut.html#pilog}{Pilog Tutorial}.

This entry was posted on Thursday, April 24th, 2008 at 6:55 am 



% Local variables:
% mode: latex
% TeX-master: "../../editor.tex"
% End:
