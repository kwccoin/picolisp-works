\title{Working with tables in PicoLisp}
\author{Henrik Sarvell}
% Use \authorrunning{Short Title} for an abbreviated version of
% your contribution title if the original one is too long
\institute{\texttt{hsarvell@gmail.com}}
%
% Use the package "url.sty" to avoid
% problems with special characters
% used in your e-mail or web address
%


\maketitle

\begin{abstract}
  This article demonstrates how to construct a \emph{table} in
  PicoLisp, retrieve data from it and sort it. 
\end{abstract}

\section{Example Data}
\label{sec:work-with-tables-example-data}

Let's examine a common example, a list of people forming a table without
key access to each person could look like this:

\begin{wideverbatim}
(setq *People 
      '(((name . John) (phone . 123456) (age . 56))
        ((name . Fred) (phone . 654321) (age . 35))
        ((name . Fred) (phone . 236597) (age . 38))
        ((name . Hank) (phone . 078965) (age . 23))))
\end{wideverbatim}


\section{Retrieving data from the table}
\label{sec:work-with-tables-retrieving-data-from-the-table}


What you see is they way to specify pairs, each pair is a \texttt{car}
and a \texttt{cdr}. If you recall the way \texttt{assoc} worked in the
prior tutorial you realize that each sublist can be accessed with that
function - since each key is a \texttt{car}. With that in mind a solution for
retrieving a person by key and value could look like this:


\begin{wideverbatim}
(de assoc2d (Lst K V)
    (filter '((Sub)
              (let CurValue (cdr (assoc K Sub))
                (= V CurValue))) Lst ) )
                
(println (assoc2d *People 'name 'Fred))
\end{wideverbatim}

The above example will retrieve a new list with all persons called Fred.
The way this works is through \textbf{filter} which is a member of the same
family as \textbf{mapcar}. Filter will return a new list with all items that
the callback/literal function returned true for. As in the example in
the prior tutorial Sub will be each element, in this case each person.

Next we initiate a temporary variable with \texttt{(let CurValue (cdr
  (assoc K Sub)} \emph{logic using CurValue here} ), \textbf{let} is a
cousin of \texttt{setq} but will create it's own little space. Had we
already had a variable called \texttt{CurValue} with some value in it in the
above example - that variable would've gotten a new value inside the
let expression. However after the let expression has finished
executing, \texttt{CurValue} would revert back to it's original value. It's a
convenient way of using a variable name temporarily without having to
worry about wrecking something else.

Next we use the equal (\texttt{=}) function to test if our current
value is equal to the passed value in V, since \texttt{=} will return
\texttt{T} (Pico Lisp's equivalent to true) if they are equal, or \texttt{NIL}
(the equivalent of false) we are done. All sublists without the wanted
key/value combination will return \texttt{NIL} and will therefore not get a
place in the return array.

What if you want to create your own table system, maybe you think the
above way of defining a table is too verbose, you want it to look like
this instead:

\begin{wideverbatim}
(setq *People 
      '((name John phone 123456 age 56)
        (name Fred phone 654321 age 35)
        (name Fred phone 236597 age 38)
        (name Hank phone 078965 age 23)))
\end{wideverbatim}

No problem, then you could define the lookup logic like this instead:

\begin{wideverbatim}
(de assoc1d (Lst Key) 
    (loop
     (NIL Lst NIL)
     (T (= Key (pop 'Lst)) (pop 'Lst))
     (pop 'Lst)))

(de assoc2d (Lst K V)
    (filter '((Sub)
              (= V (assoc1d Sub K))) Lst ) )

(println (assoc2d *People 'name 'Fred))
\end{wideverbatim}

Our custom \texttt{assoc1d} function is basically a replacement of
\texttt{assoc} tailored to our custom table system. It will return the
value of the passed key or \texttt{NIL} if the key can't be found. You
could try it out in isolation:

\begin{wideverbatim}
(println (assoc1d '(phone 123456 name John age 56) 'name))
\end{wideverbatim}

The \textbf{loop} function will let us loop infinitely and have an
arbitrary amount of conditional exits at the same time. We begin by
checking if the list (\texttt{Lst}) is \texttt{NIL} (all elements have
been popped off), if that is the case we return \texttt{NIL}. If not then we
pop an element off and check if it matches the passed key, if yes then
we return the next element (the value) by popping it off. If the key
didn't match we will continue down the loop conditionals and simply
pop the value off without returning it. The equivalent would look
something like this in PHP:

\begin{wideverbatim}
function assoc1d($lst, $key){
  while(!empty($lst)){
    if($key == array_shift($lst))
      return array_shift($lst);
    else
      array_shift($lst);
  }
  return false;
}
\end{wideverbatim}

In this case the Lisp equivalent isn't really any shorter. However, the
alternative to loop would be the use of temporary variables and so on
and that is a road we don't want to start walking down.

The \texttt{assoc2d} function doesn't present us with any surprises, we simply
check each returned value from the \texttt{assoc1d} function with the passed
value (V).


\section{Sort the table}
\label{sec:work-with-tables-sort-the-table}

So what if we want to sort our table of persons? The solution is similar
to sorting
\href{http://www.prodevtips.com/2008/01/06/sorting-2d-arrays-in-php-anectodes-and-reflections/}{tables in PHP}. We will extract the values (column) we want to sort by - with
the help of the key:

\begin{wideverbatim}
(de getCol (Lst K)
    (mapcar '((Sub)(assoc1d Sub K)) Lst ) )
    
(println (getCol *People 'phone))
\end{wideverbatim}

Let's put it all together:


\begin{wideverbatim}
(de getSorting (Sorted Original)
    (make
     (while Original
            (let Value (pop 'Original)
              (setq Sorted 
                    (place 
                     (link (index Value Sorted)) Sorted NIL ))))))

(de sort2d (L Key)  
     (let Col (getCol L Key)
       (mapcar '((Pos)(get L Pos)) (getSorting (sort (copy Col)) Col) )))

(println (sort2d *People 'name))
\end{wideverbatim}

When contemplating problems in Lisp it helps to think in terms of how
to generate intermediate lists that are needed to solve the problem.
In this case we generate a list describing how to sort our table
through the \texttt{getSorting} function. That list will be used in
the \texttt{Sort2d} function to do the actual sorting.

In \texttt{Sort2d} we begin by fetching the column represented by the
key, in this case we get a flat list of names as they appear in the
table (a column). Note that we use \textbf{copy} before we sort the
first argument to \texttt{getSorting}, the reason being that sort will
also sort in place, therefore to get two distinct columns in
\texttt{getSorting} we have to copy one of them, in this case the
sorted version. Otherwise they would both be sorted and that would
ruin everything.

Right at the beginning of \texttt{getSorting} we start with
\textbf{make} which will initiate a make environment. Inside a make
environment there are a few special functions that can be used, the
most common one is \textbf{link} which we use here. In this case make
will return a list created with all arguments to link. It doesn't
matter how or where the link is called, whenever and wherever it is
called within the make environment will cause it to append another
item to the list being made.

Here we will loop through the Original column with unsorted names and
pop a name off each time until it's empty. Each name is stored in
Value followed by a call to \texttt{setq} to change the Sorted column
but why? The reason is duplicate names, we need some way of marking
already fetched names by setting them to \texttt{NIL}, otherwise the
\textbf{index} function would return the same position for Fred. This
would later result in us getting a copy of the first Fred in the
sorted table instead of two unique Freds, that is really, really
unwanted behavior.

This way of preventing duplicate Freds feels a little bit ugly. In Lisp
there is not one way of doing something, in fact there are probably an
infinite number of ways of doing something, one better than the other,
only the imagination and cleverness of the programmer sets the limits.
And unfortunately my limit was reached here, but somehow I realize that
there probably is a better way \dots{}

Anyway, it's the \textbf{place} function that is responsible for replacing
names whose positions we have already linked with \texttt{NIL}, the above way of
doing this is possible because link returns the value it links as well
as linking it, thus it can be used with place at the same time. Check
out both index, place, make and link in the reference.

The main point with \texttt{getSorting} is that we compare the sorted
names with the unsorted names and return the position of the original
name in the sorted column. These are then used in \texttt{'((Pos)(get
  L Pos))} as each \texttt{Pos} with \textbf{get}. \texttt{Get} can be
used in a variety of situations, in this case it's simply used as an
index lookup, in PHP it might have looked like \$L[\$Pos].

Don't like the way the table got sorted? No problem, try calling
\texttt{Sort2d} like this instead:


\begin{wideverbatim}
(println (flip (sort2d *People 'name)))
\end{wideverbatim}

Yep, \textbf{flip} will reverse the list.

Maybe a bit late but: This is probably not the way you would be working
with records in a ``sharp'' situation. Every person would then be a
database object and sorted upon retrieval, don't get mad though. The
above was a good exercise in any case.

Note also the use of \textbf{name}, as you can see it seems like it's a
reserved word, still we are able to use it by quoting it.

