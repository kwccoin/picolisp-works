\title{Factorials, Permutations and Recursion in PicoLisp}
\author{Henrik Sarvell}
% Use \authorrunning{Short Title} for an abbreviated version of
% your contribution title if the original one is too long
\institute{\texttt{hsarvell@gmail.com}}
%
% Use the package "url.sty" to avoid
% problems with special characters
% used in your e-mail or web address
%


\maketitle

\begin{abstract}
  This article describes \emph{factorials}, \emph{permutations} and
  \emph{recusrion} can be used in the simulation of stock trading
  strategies.  
\end{abstract}

\section{Simulating stock trading strategies}
\label{sec:fact-perm-recur-simulating-stock-trading-strategies}

Currently I'm simulating trading strategies on historical stock data.
Yes I know according to
\href{http://en.wikipedia.org/wiki/Nassim_taleb}{Nassim} this is
complete bullshit but I might beg to differ. At least I feel the need
to determine if it's bullshit on my own than just take his word for it.

I have bought trading data from the SET which includes the years
1975--2008, the period 1997--2001 could possibly closely resemble what
we are up against at the moment so any simulated strategy that returns
more than simply sitting on the sidelines and doing nothing is a winner
and might be worth testing at the moment.

\section{Factorials and Permutation}
\label{sec:fact-perm-recur-factorials-and-permutation}


\subsection{First try}
\label{sec:fact-perm-recur-first-try}

However in order to simulate these strategies we need to be able to do
permutations and I couldn't find anything already created to this
effect, neither could I find a core factorial function, so here they
are:

\begin{wideverbatim}
(de fac (Num)
   (let Res 1
      (for (N 1 (>= Num N) (inc N))
         (setq Res (* Res N)))))
\end{wideverbatim}

So this one is basically the same as ye old ``N!''.

\begin{wideverbatim}
(de switch (Lst P1 P2)
  (let (V1 (get Lst P1) V2 (get Lst P2))      
      (place P1 (place P2 Lst V1) V2)))
\end{wideverbatim}

This one is used in the permutate function below, however it might be of
use in a standalone fashion, hence it having its own definition. Anyway
the end result is a switch of the values indicated by the numbers in P1
and P2.

\subsection{Using \texttt{recur} and \texttt{recurse}}
\label{sec:fact-perm-recur-using-recur}

\begin{wideverbatim}
(de permutate (Lst)      
   (let (Result (list) Count 1 Start 1) 
      (recur (Lst Start Result Count) 
         (when (>= (fac (length Lst)) Count)
            (push Result Lst)
            (when (= Start (length Lst)) (setq Start 1))
            (recurse 
               (switch Lst Start (inc Start)) 
               (inc Start) 
               Result 
               (inc Count)))
         (car Result))))
\end{wideverbatim}

Note \textbf{recur} and \textbf{recurse} here, we might just have
created a different non-recursive entry function instead but using
these two is a more lazy approach that lets us dispense with the need
to create two different definitions.

The end result is a list of lists with all different permutations.

Anyway, I will put this stuff up for inspection on the Pico Lisp mailing
list and let more knowledgeable people give feedback, updates with new
code and comments will most likely appear here in the near future. 

\textbf{Update}: OK so it didn't work, I created the above based on an
(1 2 3) example list, however in my sharp application I work with 4
numbers and it didn't manage that, I'll leave it though as an example
of how recursion can be done in PicoLisp.

\subsection{Second try}
\label{sec:fact-perm-recur-second-try}

I ended up stealing one of the algorithms from the
\href{http://en.wikipedia.org/wiki/Permutation}{permutation Wikipedia}
article, and this is the result:


\begin{wideverbatim}
(de permutation (N Lst)
   (for (J 2 (>= (length Lst) J) (inc J))
      (setq N (/ N (- J 1)))
      (setq Lst (switch Lst (inc (% N J)) J))))
   
(de permutate (Lst)
   (let Rslt (list)
      (for (N 1 (>= (fac (length Lst)) N) (inc N))
         (push Rslt (permutation N Lst)))
      (uniq (car Rslt))))
\end{wideverbatim}

Everything else equal.

\subsection{Using \texttt{permute}}
\label{sec:fact-perm-recur-using-permute}

\textbf{Update}: So I got my answer from Alex on the mailing list:

\begin{quote}
Well, there is the `permute' function in ``lib/simul.l''. Does it what
you intend?
\end{quote}


\begin{wideverbatim}
(de permute (Lst)
   (ifn (cdr Lst)
      (cons Lst)
      (mapcan
         '((X)
            (mapcar
               '((Y) (cons X Y))
               (permute (delete X Lst)) ) )
         Lst ) ) )
\end{wideverbatim}

Indeed, and very nice, excellent solution, I wish my mind was lispy
enough to come up with these things myself. If you encapsulate the
recursive permute call in a println you will get a feeling for how it
works.
