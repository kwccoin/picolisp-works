\title{Registers and Quoting in PicoLisp}
\author{Henrik Sarvell}
% Use \authorrunning{Short Title} for an abbreviated version of
% your contribution title if the original one is too long
\institute{\texttt{hsarvell@gmail.com}}
%
% Use the package "url.sty" to avoid
% problems with special characters
% used in your e-mail or web address
%


\maketitle

\begin{abstract}
  This article explains how to use the fundamental Lisp functions
  \texttt{car} and \texttt{cdr} in PicoLisp. 
\end{abstract}

\section{Install and Start}
\label{sec:reg-and-quot-install-and-start}

By now you should have compiled and installed PicoLisp as per the
\href{http://www.software-lab.de/INSTALL}{install instructions}. We will
start the \href{http://www.software-lab.de/tut.html#ledit}{interpreter}
with \textbf{./dbg}. You can create a file and just copy paste the tutorial
snippets and run them in the interpreter with : (load ``tutorial.l''). To
rerun you hit ``Esc'' followed by ``k'' to step through prior commands, hit
``Enter'' when you see the load command.

\section{The car and the cdr}
\label{sec:reg-and-quot-the-car-and-the-cdr}

As in other Lisps the \texttt{Contents of Address Register (CAR)} and
\texttt{Contents of Decrement Register (CDR)} is at the
\href{http://www.software-lab.de/ref.html#cell}{heart of the
  language}. You might also want to check out
\href{http://en.wikipedia.org/wiki/Lisp_%28programming_language%29}{Lisp on Wikipedia} before you continue. Also be sure to check out the
  \href{http://www.software-lab.de/ref.html#conv}{naming conventions}.

Let's try it out:

\begin{wideverbatim}
(setq *Greeting (list "Hello" "how" "are" "you" "doing?"))
(prin (car *Greeting))
\end{wideverbatim}

\texttt{Setq} will put the list created with list in the variable \texttt{Greeting}. Now
run the above code again but substitute \texttt{car Greeting} with \texttt{cdr Greeting}.
As you see this is basically the \emph{key =\textgreater  value} system of PHP.

Let's create a simple table/2D array:

\begin{wideverbatim}
(setq *Fruits 
      (list 
       (list "green" "apple" "guava" "avocado") 
       (list "red" "cherry" "apple")))

(prin (car *Fruits))
\end{wideverbatim}

As you can see you will get the whole first list from the last \texttt{car}
command, not really surprising if we follow the logic from the first
example, this whole list will basically be the key that was represented
by ``Hello'' above. Try writing


\begin{wideverbatim}
(prin (car (car *Fruits)))
\end{wideverbatim}

on the last line instead. As you see this will return ``green'' which is
the \texttt{car} of the first list which in turn is the car of the whole table.
These double, triple and quadruple car/cdr calls have shortcuts. Try


\begin{wideverbatim}
(prin (caar *Fruits))
\end{wideverbatim}

instead and you will get the same result. See what you get when you
try \texttt{cdr}, \texttt{cdar}, \texttt{cadr}, \texttt{caadr} and
\texttt{cdadr}. Which combinations are they shortcuts for? Take your
time to learn how these register functions applies to various list
configurations, the time will be well spent because these things are
used all the time. They're everywhere.

As it happens there is a shortcut for retrieving the contents of any key
even if it's in place 100 down the line. Try substituting the last line
with

\begin{wideverbatim}
(prin (assoc "red" *Fruits))
\end{wideverbatim}

instead. As you can see the whole list with red fruits and the key
(car) is returned. This is normally not what you want when you make a
call like that, you don't want the key too, only the fruits. But armed
with our knowledge of how the whole \texttt{car} and \texttt{cdr}
thing works we quickly do the following:


\begin{wideverbatim}
(prin (cdr (assoc "red" *Fruits)))
\end{wideverbatim}

That's better, we now get all red fruits, and only the fruits.

\section{Quoting}
\label{sec:reg-and-quot-quoting}

Try this:

\begin{wideverbatim}
(set '*Greeting '("Hello" "how" "are" "you" "doing?"))
(prin (car *Greeting))
\end{wideverbatim}

Different than above but the result is the same, it's easy to see that
\texttt{setq var} is just a shortcut for \texttt{set 'var}. The
\texttt{'} is a shortcut for \textbf{quote}, everything quoted is
taken literally. Variable names inside a quoted list will of course
not expand into their values - except when when evaluated or passed to
a family of special functions, \textbf{mapcar} (described below) is
one of them.

Evaluation example:

\begin{wideverbatim}
(setq *Prin 'prin)
(eval '(*Prin "hello"))
\end{wideverbatim}

However, when not evaluated:

\begin{wideverbatim}
(setq *Prin 'prin)
(prin '(*Prin "hello"))
\end{wideverbatim}

When thinking about quoting it helps - at least for me - to think about
the turing machine that accepts instructions in the form of these long
paper strips that you feed into it. On the other side you get the result
of the computation. When feeding the machine a variable it will of
course expand into the strip it contains. When quoting though you simply
feed the machine the raw/literal strip, telling it to treat it as such.

There are however exceptions, some functions work on data by reference,
just like with the \texttt{\&} in PHP. Let's look at such an example:

\begin{wideverbatim}
(setq *Greeting '("Hello" "how" "are" "you" "doing?"))
(prinl (pop '*Greeting))
(prinl *Greeting)
\end{wideverbatim}

In this case, as you can see the \texttt{'} will make the pop function
treat Greeting as something passed by reference. Quoted lists can of
course be created dynamically and then be passed around and executed
to create yet other lists that will be executed in other places in
absurdity. If done properly you can in this way utilize the power of
Lisp.

As it happens there is a whole category of functions that will accept a
function literal (quoted list) and use that function on another list.
Let's look at a simple example:

\begin{wideverbatim}
(de getSomething (Lst R)
    (mapcar '((Element) (R Element)) Lst) )

(setq *Fruits 
      '(("green" "apple" "guava" "avocado") 
        ("red" "cherry" "apple")))

(prin (getSomething *Fruits 'car))
\end{wideverbatim}

In this case we will retrieve all keys, basically the same functionality
as the PHP function array\_keys. If you do 'cdr you will instead get the
values which corresponds to array\_values(). What if you want to get the
first fruit in each sub-list? Yep you guessed it, just pass 'cadr
instead. Pretty dynamic isn't it? If we return to the Turing machine
analogy, this would constitute using a placeholder on a raw strip that
will get it's value when it's time to execute the strip/list.

What's happening here is that the function mapcar takes a function
literal as it's first parameter, the second parameter is the list the
function literal will operate on, actually mapcar can accept several
lists and use their contents in the function literal, however in this
case we just keep it simple. Element will be each element in the list,
in our case the two sub-lists. The result will be a new list whose
elements are the results of each operation our function literal
performs. That is why we get a list with ``green'' and ``red'' in it if we
pass 'car to getSomething.

Notice also that there is no return keyword, in PicoLisp (and all Lisps
I think) all expressions return something, hence no need for a return
keyword. And the (de… keyword is the same as function in PHP, we simply
define a new function to be used later. That's all for this time, the
next tutorial will probably cover more advanced list manipulation
examples.

