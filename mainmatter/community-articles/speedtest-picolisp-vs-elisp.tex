\title{Speedtest PicoLisp vs Elisp}
% Use \titlerunning{Short Title} for an abbreviated version of
% your contribution title if the original one is too long
\author{Thorsten Jolitz\inst{1}\and
Jos\'{e} Romero\inst{2}}
% Use \authorrunning{Short Title} for an abbreviated version of
% your contribution title if the original one is too long
\institute{
\texttt{tjolitz@gmail.com}
\and \texttt{jir@2.71828.com.ar}}
%
% Use the package "url.sty" to avoid
% problems with special characters
% used in your e-mail or web address
%

\maketitle

% 16jul12
% Thorsten Jolitz

% \documentclass[10pt,a4paper]{article}
% \usepackage{graphicx}

% \textwidth 1.4\textwidth
% \textheight 1.125\textheight
% \oddsidemargin 0em
% \evensidemargin 0em
% \headsep 0em
% \parindent 0em
% \parskip 6pt

% \title{Speedtest PicoLisp vs Emacs Lisp}
% \author{Thorsten Jolitz\\tj@data-driven.de}
% \date{2012-04-23}

% \begin{document}
% \maketitle

% \section{Speedtest PicoLisp vs Emacs Lisp}

\begin{abstract}
  This article compares the speed of (compiled and interpreted)
  \emph{Emacs Lisp} with (always interpreted) \emph{PicoLisp}, testing
  the costs of \emph{function calls/arithmetic} as well as \emph{list
    manipulation}. 
\end{abstract}

\section{The Tests}
\label{sec:pl-vs-elisp-speedtest-picolisp-vs-emacs-lisp}

\subsection{Function Call/Arithmetic Cost}
\label{sec:pl-vs-elisp-function-call/arithmetic-cost}

\subsubsection{Shell Script Approach}
\label{sec:pl-vs-elisp-shell-script-approach}

The classic \emph{Fibonacci function} was used for measuring function
call/arithmetic cost.

Here is the PicoLisp script:

\begin{wideverbatim}
#!/usr/bin/picolisp
(de fibo (N)
   (if (> 2 N)
      1
      (+ (fibo (dec N)) (fibo (- N 2))) ) )
(fibo 35)
(bye)
\end{wideverbatim}

Here is the (uncompiled) Emacs Lisp script:

\begin{wideverbatim}
#!/usr/bin/emacs --script

(defun fibo (N)
   (if (> 2 N)
      1
      (+ (fibo (1- N)) (fibo (- N 2))) ) )

(fibo 35)
\end{wideverbatim}


Here is the script that calls a byte-compiled Emacs Lisp file
with the above function definition and call:

\begin{wideverbatim}

#! /bin/sh
":"; exec emacs --no-site-file --script
"/home/tj/shellscripts/tj-fibo-compiled.elc" # -*-emacs-lisp-*-

\end{wideverbatim}


The following shell command was used to measure the performance:


\texttt{[tj@arch ~]\$ time script}

with \textit{script} being one of the three scripts above.


\subsubsection{Command Line Approach}
\label{sec:pl-vs-elisp-command-line-approach}

This is an alternative, more elegant and efficient way to run the
tests. Just produce these two files:

\begin{wideverbatim}
$ cat > fibo.el << .
 (defun fibo (N)
    (if (> 2 N)
       1
       (+ (fibo (1- N)) (fibo (- N 2))) ) )

 (fibo 35)
 .
\end{wideverbatim}

\begin{wideverbatim}
$ cat > fibo.l << .
 (de fibo (N)
    (if (> 2 N)
       1
       (+ (fibo (dec N)) (fibo (- N 2))) ) )
 (fibo 35)
 .
\end{wideverbatim}


Then byte-compile fibo.el and run the following commands:

\begin{wideverbatim}
$ time emacs --no-site-file --script fibo.el
$ time emacs --no-site-file --script fibo.elc
$ time pil fibo.l -bye
\end{wideverbatim}

As a side note: Emacs can be invoked noninteractively from the shell
to do byte compilation with the aid of the function
batch-byte-compile. In this case, the files to be compiled are
specified with command-line arguments. Use a shell command of the form

\begin{wideverbatim}
emacs -batch -f batch-byte-compile files...
\end{wideverbatim}

for example

\begin{wideverbatim}
$ emacs --no-site-file -batch -f batch-byte-compile fibo.el
\end{wideverbatim}

\subsection{List Manipulation Cost}
\label{sec:pl-vs-elisp-list-manipulation-cost}

The costs of list manipulation were tested with the ``extensive list
manipulations`` code from Alexander Burger:

\begin{wideverbatim}
$ cat > tst.l << .
 (de tst ()
   (mapcar
    (quote (X)
       (cons
        (car X)
        (reverse (delete (car X) (cdr X))) ) )
    '((a b c a b c) (b c d b c d) (c d e c d e) (d e f d e f)) ) )
  (do 1000000 (tst))
 .
\end{wideverbatim}


\begin{wideverbatim}
$ cat > tst.el << .
 (defun tst ()
   (mapcar
    (lambda (X)
       (cons
        (car X)
        (reverse (delete (car X) (cdr X))) ) )
    '((a b c a b c) (b c d b c d) (c d e c d e) (d e f d e f)) ) )
 (dotimes (i 1000000) (tst))
 .
\end{wideverbatim}


\section{Results}
\label{sec:pl-vs-elisp-results}

\subsection{32bit}
\label{sec:pl-vs-elisp-32bit}

\subsubsection{System Information}
\label{sec:pl-vs-elisp-system-information}

\begin{wideverbatim}
$ uname -a
Linux icz 2.6.32-5-686 #1 SMP Mon Jan 16 16:04:25 UTC 2012 i686
GNU/Linux
$ cat /proc/cpuinfo |grep "model name" | cut -d: -f2
 Pentium(R) Dual-Core CPU       T4200  @ 2.00GHz
 Pentium(R) Dual-Core CPU       T4200  @ 2.00GHz
\end{wideverbatim}


\subsubsection{Function Calls}
\label{sec:pl-vs-elisp-function-calls}

These are the results for running \texttt{fibo (N)} with \texttt{N=35}:

\begin{wideverbatim}
| PicoLisp         | 0m5.662s  | 1x      |
| Elisp            | 0m13.854s | ca 2.5x |
| Elisp (compiled) | 0m5.882s  | ca 1x   |
\end{wideverbatim}


PicoLisp is 2.5x faster than interpreted Emacs Lisp and as fast as compiled
Emacs Lisp.

\subsubsection{List Manipulation}
\label{sec:pl-vs-elisp-list-manipulation}

These are the results for running \texttt{tst} with \texttt{(do 1000000 (tst)} or
\texttt{(dotimes (i 1000000) (tst))}:

\begin{wideverbatim}
| PicoLisp         | 0m1.208s | 1x      |
| Elisp            | 0m8.311s | ca 7x   |
| Elisp (compiled) | 0m5.622s | ca 4.5x |
\end{wideverbatim}


PicoLisp is 7x faster than interpreted Emacs Lisp and 4.5x faster than compiled
Emacs Lisp. Looks like the Emacs compiler can't improve much in that function and
it's still 4.6-6.9x slower than PicoLisp.

\subsection{64bit}
\label{sec:pl-vs-elisp-64bit}

\subsubsection{System Information}
\label{sec:pl-vs-elisp-system-information}

\begin{wideverbatim}
$ uname -a
Linux arch 3.3.2-1-ARCH #1 SMP PREEMPT Sat Apr 14 09:48:37 CEST 2012
x86_64 AMD Athlon(tm) 64 X2 Dual Core Processor 5000+ AuthenticAMD
GNU/Linux
$ cat /proc/cpuinfo |grep "model name" | cut -d: -f2
 AMD Athlon(tm) 64 X2 Dual Core Processor 5000+
 AMD Athlon(tm) 64 X2 Dual Core Processor 5000+
\end{wideverbatim}


\subsubsection{Function Calls}
\label{sec:pl-vs-elisp-function-calls}

These are the results for running \texttt{fibo (N)} with \texttt{N=35}:

\begin{wideverbatim}
| PicoLisp         | 0m3.191s  | 1x    |
| Elisp            | 0m12.731s | ca 4x |
| Elisp (compiled) | 0m6.635s  | ca 2x |
\end{wideverbatim}


PicoLisp is 4x faster than interpreted Emacs Lisp and 2x faster than compiled
Emacs Lisp.

These are the results for running \texttt{fibo (N)} with \texttt{N=40}:

\begin{wideverbatim}
| PicoLisp         | 0m35.982s | 1x    |
| Elisp            | 2m14.352s | ca 4x |
| Elisp (compiled) | 1m13.304s | ca 2x |
\end{wideverbatim}


Again PicoLisp is ca. 2x faster than compiled Elisp and
4x faster than interpreted Elisp.


\subsubsection{List Manipulation}
\label{sec:pl-vs-elisp-list-manipulation}

These are the results for running \texttt{tst} with \texttt{(do 1000000 (tst)} or
\texttt{(dotimes (i 1000000) (tst))}:

\begin{wideverbatim}
| PicoLisp         | 0m1.635s | 1x      |
| Elisp            | 0m9.582s | ca 6x   |
| Elisp (compiled) | 0m7.129s | ca 4.5x |
\end{wideverbatim}


PicoLisp is 6x faster than interpreted Emacs Lisp and 4.5x faster than
compiled Emacs Lisp.

Just to remind you - PicoLisp is always interpreted, but the
interpreter is designed with the \underline{need for speed}\footnote{http://picolisp.com/5000/!wiki?needforspeed}.


\subsection{32bit vs 64bit}
\label{sec:pl-vs-elisp-32bit-vs-64bit}
All other things equal, 64-bit PicoLisp is usually slower than the 32-bit
version, due to a poorer memory cache performance (the cells are twice as
large size). On the other hand, arithmetics are faster, due to the additional
short number type in pil64.
