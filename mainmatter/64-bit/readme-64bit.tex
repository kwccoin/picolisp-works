\title{README 64-bit}
\author{Alexander Burger}
% Use \authorrunning{Short Title} for an abbreviated version of
% your contribution title if the original one is too long
\institute{\texttt{abu@software-lab.de}}
%
% Use the package "url.sty" to avoid
% problems with special characters
% used in your e-mail or web address
%


\maketitle


\begin{abstract}
  This is the README file from the 64-bit PicoLisp distribution. 
\end{abstract}

\section{64-bit PicoLisp}
\label{sec:64-bit-64-bit-picolisp}


The 64-bit version of PicoLisp is a complete rewrite of the 32-bit
version.

While the 32-bit version was written in C, the 64-bit version is
implemented in a generic assembler, which in turn is written in
PicoLisp. In most respects, the two versions are compatible (see
"Differences" below).


\subsection{Building the Kernel}
\label{sec:64-bit-building-the-kernel}

No C-compiler is needed to build the interpreter kernel, only a 64-bit
version of the GNU assembler for the target architecture.

The kernel sources are the "*.l" files in the "src64/" directory. The
PicoLisp assembler parses them and generates a few "*.s" files, which
the GNU assembler accepts to build the executable binary file. See the
details for bootstrapping the "*.s" files in INSTALL.

The generic assembler is in "src64/lib/asm.l". It is driven by the
script "src64/mkAsm" which is called by "src64/Makefile".

The CPU registers and instruction set of the PicoLisp processor are
described in "doc64/asm", and the internal data structures of the
PicoLisp machine in "doc64/structures".

Currently, x86-64/Linux, x86-64/SunOS and ppc64/Linux are supported. The
platform dependent files are in the "src64/arch/" for the target architecture,
and in "src64/sys/" for the target operating system.


\subsection{Reasons for the Use of Assembly Language}
\label{sec:64-bit-reasons-for-the-use-of-assembly-language}

Contrary to the common expectation: Runtime execution speed was not a
primary design decision factor. In general, pure code efficiency has
not much influence on the overall execution speed of an application
program, as memory bandwidth (and later I/O bandwidth) is the main
bottleneck.

The reasons to choose assembly language (instead of C) were, in
decreasing order of importance:

\begin{enumerate}
\item  Stack manipulations

      Alignment to cell boundaries: To be able to directly express the desired
      stack data structures (see "doc64/structures", e.g. "Apply frame"), a
      better control over the stack (as compared to C) was required.

      Indefinite pushs and pops: A Lisp interpreter operates on list structures
      of unknown length all the time. The C version always required two passes,
      the first to determine the length of the list to allocate the necessary
      stack structures, and then the second to do the actual work. An assembly
      version can simply push as many items as are encountered, and clean up the
      stack with pop's and stack pointer arithmetics.

\item Alignments and memory layout control

      Similar to the stack structures, there are also heap data structures that
      can be directly expressed in assembly declarations (built at assembly
      time), while a C implementation has to defer that to runtime.

      Built-in functions (SUBRs) need to be aligned to to a multiple of 16+2,
      reflecting the data type tag requirements, and thus allow direct jumps to
      the SUBR code without further pointer arithmetic and masking, as is
      necessary in the C version.

\item  Multi-precision arithmetics (Carry-Flag)

      The bignum functions demand an extensive use of CPU flags. Overflow and
      carry/borrow have to emulated in C with awkward comparisons of signed
      numbers.

\item  Register allocation

      A manual assembly implementation can probably handle register allocation
      more flexibly, with minimal context saves and reduced stack space, and
      multiple values can be returned from functions in registers. As mentioned
      above, this has no measurable effect on execution speed, but the binary's
      overall size is significantly reduced.

\item  Return status register flags from functions

      Functions can return condition codes directly. The callee does not need to
      re-check returned values. Again, this has only a negligible impact on
      performance.

\item  Multiple function entry points

      Some things can be handled more flexibly, and existing code may be easier
      to re-use. This is on the same level as wild jumps within functions
      ('goto's), but acceptable in the context of an often-used but rarely
      modified program like a Lisp kernel.

\end{enumerate}



It would indeed be feasible to write only certain parts of the system in
assembly, and the rest in C. But this would be rather unsatisfactory. And it
gives a nice feeling to be independent of a heavy-weight C compiler.


\subsection{Differences to the 32-bit Version}
\label{sec:64-bit-differences-to-the-32-bit-version}

Except for the following seven cases, the 64-bit version should be upward
compatible to the 32-bit version.

\begin{enumerate}
\item  Internal format and printed representation of external symbols

   This is probably the most significant change. External (i.e. database)
   symbols are coded more efficiently internally (occupying only a single cell),
   and have a slightly different printed representation. Existing databases need
   to be converted.

\item  Short numbers are pointer-equal

   As there is now an internal "short number" type, an expression like

\begin{wideverbatim}
      (== 64 64)
\end{wideverbatim}

   will evaluate to 'T' on a 64-bit system, but to 'NIL' on a 32-bit system.

\item  Bit manipulation functions may differ for negative arguments

  Numbers are represented internally in a different format. Bit
  manipulations are not really defined for negative numbers, but (\&
  -15 -6) will give -6 on 32 bits, and 6 on 64 bits.

\item  'do' takes only a 'cnt' argument (not a bignum)

  For the sake of simplicity, a short number (60 bits) is considered
  to be enough for counted loops.

 \item  Calling native functions is different.

   Direct calls using the 'lib:fun' notation is still possible (see
   the 'ext' and 'ht' libraries), but the corresponding functions must
   of course be coded in assembly and not in C. To call C functions,
   the new 'native' function should be used, which can interface to
   native C functions directly, without the need of glue code to
   convert arguments and return values.

\item  New features were added, like coroutines or namespaces.

\item  Bugs (in the implementation, or in this list ;-)

\end{enumerate}

