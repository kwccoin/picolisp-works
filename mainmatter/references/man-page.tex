\title{Manual Page}
\author{Alexander Burger}
% Use \authorrunning{Short Title} for an abbreviated version of
% your contribution title if the original one is too long
\institute{\texttt{abu@software-lab.de}}
%
% Use the package "url.sty" to avoid
% problems with special characters
% used in your e-mail or web address
%


\maketitle

% \section{Manual Page}
% \label{sec:manpage-manual-page}

\begin{abstract}
  This is the PicoLisp Manual Page.
\end{abstract}

\section{NAME}
\label{sec:manpage-name}


pil, picolisp - a fast, lightweight Lisp interpreter

 
\section{SYNOPSIS}
\label{sec:manpage-synopsis}


\textbf{pil} [arguments \ldots{}] [-] [arguments \ldots{}] [+]
 \textbf{/installpath/bin/picolisp} [arguments \ldots{}] [-] [arguments \ldots{}] [+]

 
\section{DESCRIPTION}
\label{sec:manpage-description}

\begin{description}

\item[PicoLisp] is a Lisp interpreter with a small memory footprint, yet
relatively high execution speed. It combines an elegant and powerful
language with built-in database functionality.

\item[pil] is the startup front-end for the interpreter. It takes care of
starting the binary base system and loading a useful runtime
environment.

\item[picolisp] is just the bare interpreter binary. It is usually called in
stand-alone scripts, using the she-bang notation in the first line,
passing the minimal environment in \emph{lib.l} and loading additional files
as needed:

\end{description}

\begin{wideverbatim}
(load ``@ext.l'' ``myfiles/lib.l'' ``myfiles/foo.l'')

(do \ldots{} something \ldots{})

(bye)
\end{wideverbatim}

 
\section{INVOCATION}
\label{sec:manpage-invocation}


\textbf{PicoLisp} has no pre-defined command line flags; applications are free
to define their own. Any built-in or user-level Lisp function can be
invoked from the command line by prefixing it with a hyphen. Examples
for built-in functions useful in this context are \textbf{version} (print the
version number) or \textbf{bye} (exit the interpreter). Therefore, a minimal
call to print the version number and then immediately exit the
interpreter would be:

\begin{wideverbatim}
$ pil -version -bye
\end{wideverbatim}

Any other argument (not starting with a hyphen) should be the name of a
file to be loaded. If the first character of a path or file name is an
at-mark, it will be substituted with the path to the installation
directory.

All arguments are evaluated from left to right, then an interactive
\emph{read-eval-print} loop is entered (with a colon as prompt).

A single hyphen stops the evaluation of the rest of the command line, so
that the remaining arguments may be processed under program control.

If the very last command line argument is a single plus character,
debugging mode is switched on at interpreter startup, before evaluating
any of the command line arguments. A minimal interactive session is
started with:

\begin{wideverbatim}
$ pil +
\end{wideverbatim}


Here you can access the reference manual

\begin{wideverbatim}
 (doc)
\end{wideverbatim}

and the online documentation for most functions,

\begin{wideverbatim}
 (doc 'vi)
\end{wideverbatim}

or directly inspect their sources:

\begin{wideverbatim}
 (vi 'doc)
\end{wideverbatim}

The interpreter can be terminated with

\begin{wideverbatim}
 (bye)
\end{wideverbatim}

or by typing Ctrl-D.

 
\section{FILES}
\label{sec:manpage-files}


Runtime files are maintained in the \~{}/.pil directory:

\begin{wideverbatim}
~{}/.pil/tmp/<pid>/
\end{wideverbatim}


Process-local temporary directories

\begin{wideverbatim}
~{}/.pil/history
\end{wideverbatim}

The line editor's history file

 
\section{BUGS}
\label{sec:manpage-bugs}


\textbf{PicoLisp} doesn't try to protect you from every possible programming
error (``You asked for it, you got it'').

 
\section{AUTHOR}
\label{sec:manpage-author}


Alexander Burger \href{mailto:abu@software-lab.de}{abu@software-lab.de}

 
\section{RESOURCES}
\label{sec:manpage-resources}


\textbf{Home page:}
\href{http://home.picolisp.com}{http://home.picolisp.com}

\textbf{Download:}
\href{http://www.software-lab.de/down.html}{http://www.software-lab.de/down.html}
