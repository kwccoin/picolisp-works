\title{Coroutines}
\author{Alexander Burger}
% Use \authorrunning{Short Title} for an abbreviated version of
% your contribution title if the original one is too long
\institute{\texttt{abu@software-lab.de}}
%
% Use the package "url.sty" to avoid
% problems with special characters
% used in your e-mail or web address
%


\maketitle

% 16jul12
% Alexander Burger

% \documentclass[10pt,a4paper]{article}
% \usepackage{graphicx}

% \textwidth 1.4\textwidth
% \textheight 1.125\textheight
% \oddsidemargin 0em
% \evensidemargin 0em
% \headsep 0em
% \parindent 0em
% \parskip 6pt

% \title{Coroutines}
% \author{Alexander Burger\\abu@software-lab.de}
% \date{2011-07-29}

% \begin{document}
% \maketitle


\begin{abstract}
This article introduces application examples for \emph{coroutines},
comparing their use and relative efficiency with a possible solution
via \emph{generators}.     
\end{abstract}

\section{Introduction}
\label{sec:coroutines-introduction}

With picoLisp-3.0.3, the 64-bit version of PicoLisp has support for
\underline{coroutines}\footnote{http://software-lab.de/doc/ref.html\#coroutines}. This article tries
to show their basic usage.

Assume we need all Pythagorean triples (i.e. all numbers A, B and C, such that
% A\char942 + B\char942 = C\char942)
$A + B = C$  with elements between 1 and N. A straightforward way to print
them is:
\begin{wideverbatim}
   (de pythag (N)
      (for A N
         (for B (range A N)
            (for C (range B N)
               (when (= (+ (* A A) (* B B)) (* C C))
                  (println (list A B C)) ) ) ) ) )
\end{wideverbatim}

We get:
\begin{wideverbatim}
   : (pythag 20)
   (3 4 5)
   (5 12 13)
   (6 8 10)
   (8 15 17)
   (9 12 15)
   (12 16 20)
\end{wideverbatim}

However, just printing the triples is not very useful if they were needed in
various situations for further processing. The lispy way for a general tool is
passing a \textit{function} to \texttt{pythag} and decide later what to do with the data:
\begin{wideverbatim}
   (de pythag (N Fun)
      (for A N
         (for B (range A N)
            (for C (range B N)
               (when (= (+ (* A A) (* B B)) (* C C))
                  (Fun (list A B C)) ) ) ) ) )
\end{wideverbatim}

(note that for a truly general tool, we would write \texttt{Fun} and the local
variables as transient symbols to avoid conflicts)

Now we can print them again
\begin{wideverbatim}
   : (pythag 20 println)
   (3 4 5)
   (5 12 13)
   (6 8 10)
   (8 15 17)
   (9 12 15)
   (12 16 20)
\end{wideverbatim}

collect them into a list if we like
\begin{wideverbatim}
   : (make (pythag 20 link))
   -> ((3 4 5) (5 12 13) (6 8 10) (8 15 17) (9 12 15) (12 16 20))
\end{wideverbatim}

or do with them whatever we like.

Still, this method has its limits. What if we need to pass a very large number
for \texttt{N}, and we want to access the values one by one, perhaps in the course of
some other involved calculation? Pre-generating the list of all values might not
be feasible.

\section{Using a Generator}
\label{sec:coroutines-using-a-generator}

One way to solve this problem is a generator. This function returns the next
value each time it is called:
\begin{wideverbatim}
   (de pythag (N)
      (job '((A . 1) (B . 1) (C . 0))
         (loop
            (when (> (inc 'C) N)
               (when (> (inc 'B) N)
                  (setq B (inc 'A)) )
               (setq C B) )
            (T (> A N))
            (T (= (+ (* A A) (* B B)) (* C C))
               (list A B C) ) ) ) )

   : (pythag 20)
   -> (3 4 5)
   : (pythag 20)
   -> (5 12 13)
   : (pythag 20)
   -> (6 8 10)
   ...
\end{wideverbatim}

Now we can call it whenever we need a new value. The function encapsulates the
state of its local variables in a \underline{job}\footnote{http://software-lab.de/doc/refJ.html\#job} environment.

A major disadvantage, however, is that it does not reflect the flow of control.
The three nested \texttt{for} loops above had to be unfolded and programmed manually.
This is hard to read, even for this simple case, and may be difficult or
impossible to program in more complicated cases.

\section{Using a Coroutine}
\label{sec:coroutines-using-a-coroutine}

A coroutine preserves the local environment, as well as the state of control of
a function. It may have multiple exit points, and continue execution where it
left off the last time.

This is done via two new functions: \underline{co}\footnote{http://software-lab.de/doc/refC.html\#co} and \underline{yield}\footnote{http://software-lab.de/doc/refY.html\#yield}.
\begin{wideverbatim}
   (de pythag (N)
      (co 'pythag
         (for A N
            (for B (range A N)
               (for C (range B N)
                  (when (= (+ (* A A) (* B B)) (* C C))
                     (yield (list A B C)) ) ) ) ) ) )

   : (pythag 20)
   -> (3 4 5)
   : (pythag 20)
   -> (5 12 13)
   : (pythag 20)
   -> (6 8 10)
   ...
\end{wideverbatim}

So this is a generator equivalent to the one above, but with cleanly nested
\texttt{for} loops.

The function \texttt{co} is called with a \textbf{tag} argument, and an executable \textbf{body}
(a \textit{prg}), similar to \underline{catch}\footnote{http://software-lab.de/doc/refC.html\#catch}. When
called the first time, a new coroutine for that tag is created, and the body
gets executed. If called again later, and an existing coroutine for that tag is
found, execution will continue at the point where it left off last time with
\texttt{yield}.

\texttt{yield} stops executing the coroutine's body, and immediately returns to the
caller (or to some other coroutine if desired). When a coroutine is resumed with
\texttt{co}, it will continue at the point of the last call to \texttt{yield}. If it is
resumed by a call to \texttt{yield} in another coroutine, the return value of the
first \texttt{yield} is the value given as an argument to the second \texttt{yield}.

\section{Efficiency}
\label{sec:coroutines-efficiency}

In terms of memory usage, coroutines are rather expensive, because each of them
requires its own stack segment. For that reason (and also due to internal
structures) the maximum number of coroutines in the system is limited to 64.

To my surprise, however, coroutines are quite efficient in terms of runtime
overhead. Measuring the context switch to and from an empty coroutine (an
endless loop with just a (yield))
\begin{wideverbatim}
   : (bench (do 1000000 (co 'bench (loop (yield)))))
   0.380 sec
   -> NIL
\end{wideverbatim}

shows that it needs just 0.38 / 2 = 0.19 microseconds per switch operation.

This is in the same order of magnitude of a normal function call:
\begin{wideverbatim}
   : (bench (do 1000000 ((quote (X Y) (+ X Y)) 3 4)))
   0.162 sec
   -> 7
\end{wideverbatim}

Comparing the implementations of \texttt{pythag} above - as a generator using \texttt{job}
and a coroutine - shows that the coroutine version is about 10 percent faster.

\section{Inspecting and Stopping Coroutines}
\label{sec:coroutines-inspecting-and-stopping-coroutines}

The function \underline{stack}\footnote{http://software-lab.de/doc/refS.html\#stack} can be used to
see which coroutines are currently running (in addition to its primary task of
setting or returning the current stack segment size).

Let's say we called \texttt{pythag} like in the example above, and then started two
more coroutines as
\begin{wideverbatim}
   : (co "routine1" (yield 1))
   -> 1
   : (co "routine2" (yield 2))
   -> 2
\end{wideverbatim}

Now there must be three coroutines running. \texttt{stack} returns them:
\begin{wideverbatim}
   : (stack)
   -> ("routine2" "routine1" pythag . 4)
\end{wideverbatim}


When \texttt{co} is called with only a \texttt{tag}, but without a body, the corresponding
coroutine is stopped.
\begin{wideverbatim}
   : (co "routine1")
   -> T
   : (co "routine2")
   -> T
   : (co 'pythag)
   -> T
   : (stack)
   -> 4
\end{wideverbatim}

Now all coroutines are gone, and \texttt{stack} returns only the remaining segment
size (in megabytes).

\section{A Tree Example}
\label{sec:coroutines-a-tree-example}

A typical example of a situation where a coroutine can simplify things a lot, is
a recursive algorithm.

For testing, let's generate a balanced binary tree:
\begin{wideverbatim}
   (balance '*Tree (range 1 15))
\end{wideverbatim}

As you may know, you can display it with the
\underline{view}\footnote{http://software-lab.de/doc/refV.html\#view} function
\begin{wideverbatim}
   : (view *Tree T)
            15
         14
            13
      12
            11
         10
            9
   8
            7
         6
            5
      4
            3
         2
            1
\end{wideverbatim}

It is easy to traverse such a tree, e.g. to print its nodes:
\begin{wideverbatim}
   : (de printNodes (Tree)
      (when Tree
         (printNodes (cadr Tree))
         (println (car Tree))
         (printNodes (cddr Tree)) ) )
   -> printNodes

   : (printNodes *Tree)
   1
   2
   3
   4
   5
   6
   7
   8
   9
   10
   11
   12
   13
   14
   15
\end{wideverbatim}

But what if you want to get the nodes returned, one by one, as in a generator
function? For example, to compare the nodes of two trees and see of they are
equal? If you think about it, you'll see that it is not trivial.

With a coroutine, it is straightforward. We can write a function that returns
another node upon each call:
\begin{wideverbatim}
   (de nextLeaf (Rt Tree)
      (co Rt
         (recur (Tree)
            (when Tree
               (recurse (cadr Tree))
               (yield (car Tree))
               (recurse (cddr Tree)) ) ) ) )
\end{wideverbatim}

With that, we can write a function to compare two trees
\begin{wideverbatim}
   (de cmpTrees (Tree1 Tree2)
      (prog1
         (use (Node1 Node2)
            (loop
               (setq
                  Node1 (nextLeaf "rt1" Tree1)
                  Node2 (nextLeaf "rt2" Tree2) )
               (T (nor Node1 Node2) T)
               (NIL (= Node1 Node2)) ) )
         (co "rt1")
         (co "rt2") ) )
\end{wideverbatim}

The last two calls to \texttt{co} stop the two coroutines, independent of the result.

