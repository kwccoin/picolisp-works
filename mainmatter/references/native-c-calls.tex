\title{Native C Calls}
\author{Alexander Burger}
% Use \authorrunning{Short Title} for an abbreviated version of
% your contribution title if the original one is too long
\institute{\texttt{abu@software-lab.de}}
%
% Use the package "url.sty" to avoid
% problems with special characters
% used in your e-mail or web address
%


\maketitle



\begin{abstract}
This article describes how to call C functions in shared object files
(libraries) from PicoLisp, using the built-in \texttt{native} function
-- possibly with the help of the \texttt{struct} and \texttt{lisp}
functions. It applies only to the 64-bit version of PicoLisp.
\end{abstract}


\section{Overview}
\label{sec:native-overview}

\texttt{native} calls a C function in a shared library. It tries to

\begin{enumerate}
\item
  find a library by name
\item
  find a function by name in the library
\item
  convert the function's argument(s) from Lisp to C structures
\item
  call the function's C code
\item
  convert the function's return value(s) from C to Lisp structures
\end{enumerate}

The direct return value of \texttt{native} is the Lisp representation of
the C function's return value. Further values, returned by reference
from the C function, are available in Lisp variables (symbol values).

\texttt{struct} is a helper function, which can be used to manipulate C
data structures in memory. It may take a scalar (a numeric
representation of a C value) to convert it to a Lisp item, or (more
typically) a pointer to a memory area to build and extract data
structures. \texttt{lisp} allows you to install callback functions,
callable from C code, written in Lisp.

In combination, these three functions can interface PicoLisp to almost
any C function.

The above steps are fully dynamic; \texttt{native} doesn't have (and
doesn't require) a priory knowledge about the library, the function or
the involved data. No need to write any glue code, interfaces or include
files. All functions can even be called interactively from the REPL.

% \begin{center}\rule{3in}{0.4pt}\end{center}

\section{Syntax}
\label{sec:native-syntax}

The arguments to \texttt{native} are

\begin{enumerate}
\item
  a library
\item
  a function
\item
  a return value specification
\item
  optional arguments
\end{enumerate}

The simplest form is a call to a function without return value and
without arguments. If we assume a library ``lib.so'', containing a
function with the prototype

\begin{wideverbatim}
void fun(void);
\end{wideverbatim}

then we can call it as

\begin{wideverbatim}
(native "lib.so" "fun")
\end{wideverbatim}

% \begin{center}\rule{3in}{0.4pt}\end{center}

\subsection{Libraries}
\label{sec:native-syntax}

The first argument to \texttt{native} specifies the library. It is
either the \emph{name} of a library (a symbol), or the \emph{handle} of
a previously found library (a number).

As a special case, a transient symbol \texttt{"@"} can be passed for the
library name. It then refers to the current main program (instead of an
external library), and can be used for standard functions like
\texttt{"malloc"} or \texttt{"printf"}.

\texttt{native} uses \texttt{dlopen(3)} internally to find and open the
library, and to obtain the handle. If the name contains a slash ('/'),
then it is interpreted as a (relative or absolute) pathname. Otherwise,
the dynamic linker searches for the library according to the system's
environment and directories. See the man page of \texttt{dlopen(3)} for
further details.

If called with a symbolic argument, \texttt{native} automatically
caches the handle of the found library in the value of that symbol.
The most natural way is to pass the library name as a \emph{transient}
symbol (\texttt{"lib.so"} above): The initial value of a transient
symbol is that symbol itself, so that \texttt{native} receives the
library name upon the first call. After successfully finding and
opening the library, \texttt{native} stores the handle of that library
in the value of the passed symbol (\texttt{"lib.so"}). As
\texttt{native} evaluates its arguments in the normal way, subsequent
calls within the same transient scope will receive the numeric value
(the handle), and don't need to open and search the library again.

% \begin{center}\rule{3in}{0.4pt}\end{center}

\subsection{Functions}
\label{sec:native-functions}

The same rules applies to the second argument, the function. When called
with a symbol, \texttt{native} stores the function pointer in its value,
so that subsequent calls evaluate to that pointer, and \texttt{native}
can directly jump to the function.

\texttt{native} uses \texttt{dlsym(3)} internally to obtain the function
pointer. See the man page of \texttt{dlsym(3)} for further details.

In most cases a program will call more than one function from a given
library. If we keep the code within the same transient scope (i.e. in
the same source file, and not separated by the \texttt{====} function),
each library will be opened -- and each function searched -- only once.

\begin{wideverbatim}
(native "lib.so" "fun1")
(native "lib.so" "fun2")
(native "lib.so" "fun3")
\end{wideverbatim}

After \texttt{"fun1"} was called, \texttt{"lib.so"} will be open, and
won't be re-opened for \texttt{"fun2"} and \texttt{"fun3"}. Consider the
definition of helper functions:

\begin{wideverbatim}
(de fun1 ()
   (native "lib.so" "fun1") )

(de fun2 ()
   (native "lib.so" "fun2") )

(de fun3 ()
   (native "lib.so" "fun3") )
\end{wideverbatim}

After any one of \texttt{fun1}, \texttt{fun2} or \texttt{fun3} was
called, the symbol \texttt{"lib.so"} will hold the library handle. And
each function function \texttt{"fun1"}, \texttt{"fun2"} and
\texttt{"fun3"} will be searched only when called the first time.

Warning: It should be avoided to put more than one library into a single
transient scope if there is a chance that two different functions with
the same name will be called in two different libraries. Because of the
function pointer caching, the second call would otherwise (wrongly) go
to the first function.

% \begin{center}\rule{3in}{0.4pt}\end{center}

\subsection{Return Value}
\label{sec:native-return-value}

The (optional) third argument to \texttt{native} specifies the return
value. A C function can return many types of values, like integer or
floating point numbers, string pointers, or pointers to structures which
in turn consist of those types, and even other structures or pointers to
structures. \texttt{native} tries to cover most of them.

As described in the \emph{result specification},
the third argument should consist of a pattern which tells
\texttt{native} how to extract the proper value.

\subsubsection{Primitive Types}
\label{sec:native-primitive-types}

In the simplest case, the result specification is \texttt{NIL} like in
the examples so far. This means that either the C function returns
\texttt{void}, or that we are not interested in the value. The return
value of \texttt{native} will be \texttt{NIL} in that case.

If the result specification is one of the symbols \texttt{B}, \texttt{I}
or \texttt{N}, an integer number is returned, by interpreting the result
as a \texttt{char} (8 bit unsigned byte), \texttt{int} (32 bit signed
integer), or \texttt{long} number (64 bit signed integer), respectively.
Other (signed or unsigned numbers, and of different sizes) can be
produced from these types with logical and arithmetic operations if
necessary.

If the result specification is the symbol \texttt{C}, the result is
interpreted as a 16 bit number, and a single-char transient symbol
(string) is returned.

A specification of \texttt{S} tells \texttt{native} to interpret the
result as a pointer to a C string (null terminated), and to return a
transient symbol (string).

If the result specification is a number, it will be used as a scale to
convert a returned \texttt{double} (if the number is positive) or
\texttt{float} (if the number is negative) to a scaled fixpoint number.

Examples for function calls, with their corresponding C prototypes:

\begin{wideverbatim}
(native "lib.so" "fun" 'I)             # int fun(void);
(native "lib.so" "fun" 'N)             # long fun(void);
(native "lib.so" "fun" 'N)             # void *fun(void);
(native "lib.so" "fun" 'S)             # char *fun(void);
(native "lib.so" "fun" 1.0)            # double fun(void);
\end{wideverbatim}

\subsubsection{Arrays and Structures}
\label{sec:native-arrays-and-structures}

If the result specification is a list, it means that the C function
returned a pointer to an array, or an arbitrary memory structure. The
specification list should then consist of either the above primitive
specifications (symbols or numbers), or of cons pairs of a primitive
specification and a repeat count, to denote arrays of the given type.

Examples for function calls, with their corresponding pseudo C
prototypes:

\begin{wideverbatim}
(native "lib.so" "fun" '(I . 8))       # int *fun(void);  // 8 integers
(native "lib.so" "fun" '(B . 16))      # unsigned char *fun(void);  // 16 bytes

(native "lib.so" "fun" '(I I))         # struct {int i; int j;} *fun(void);
(native "lib.so" "fun" '(I . 4))       # struct {int i[4];} *fun(void);

(native "lib.so" "fun" '(I (B . 4)))   # struct {
                                       #    int i;
                                       #    unsigned char c[4];
                                       # } *fun(void);

(native "lib.so" "fun"                 # struct {
   '(((B . 4) I) (S . 12) (N . 8)) )   #    struct {unsigned char c[4]; int i;}
                                       #    char *names[12];
                                       #    long num[8];
                                       # } *fun(void);
\end{wideverbatim}

If a returned structure has an element which is a \emph{pointer} to some
other structure (i.e. not an embedded structure like in the last example
above), this pointer must be first obtained with a \texttt{N} pattern,
which can then be passed to \texttt{struct} for further extraction.

% \begin{center}\rule{3in}{0.4pt}\end{center}

\subsection{Arguments}
\label{sec:native-arguments}

The (optional) fourth and following arguments to \texttt{native} specify
the arguments to the C function.

\subsubsection{Primitive Types}
\label{sec:native-primitive-types}

Integer arguments (up to 64 bits, signed or unsigned \texttt{char},
\texttt{short}, \texttt{int} or \texttt{long}) can be passed as they
are: As numbers.

\begin{wideverbatim}
(native "lib.so" "fun" NIL 123)        # void fun(int);
(native "lib.so" "fun" NIL 1 2 3)      # void fun(int, long, short);
\end{wideverbatim}

String arguments can be specified as symbols. \texttt{native} allocates
memory for each string (with \texttt{strdup(3)}), passes the pointer to
the C function, and releases the memory (with \texttt{free(3)}) when
done.

\begin{wideverbatim}
(native "lib.so" "fun" NIL "abc")      # void fun(char*);
(native "lib.so" "fun" NIL 3 "def")    # void fun(int, char*);
\end{wideverbatim}

Note that the allocated string memory is released \emph{after} the
return value is extracted. This allows a C function to return the
argument string pointer, perhaps after modifying the data in-place, and
receive the new string as the return value (with the \texttt{S}
specification).

\begin{wideverbatim}
(native "lib.so" "fun" 'S "abc")       # char *fun(char*);
\end{wideverbatim}

Also note that specifying \texttt{NIL} as an argument passes an empty
string (``'', which also reads as \texttt{NIL} in PicoLisp) to the C
function. Physically, this is a pointer to a NULL-byte, and is not a
NULL-pointer. Be sure to pass \texttt{0} (the number zero) if a
NULL-pointer is desired.

Floating point arguments are specified as cons pairs, where the value is
in the CAR, and the CDR holds the fixpoint scale. If the scale is
positive, the number is passed as a \texttt{double}, otherwise as a
\texttt{float}.

\begin{wideverbatim}
(native "lib.so" "fun" NIL             # void fun(double, float);
   (12.3 . 1.0) (4.56 . -1.0) )
\end{wideverbatim}

\subsubsection{Arrays and Structures}
\label{sec:native-arrays-and-structures}

Composite arguments are specified as nested list structures.
\texttt{native} allocates memory for each array or structure (with
\texttt{malloc(3)}), passes the pointer to the C function, and releases
the memory (with \texttt{free(3)}) when done.

This implies that such an argument can be both an input and an output
value to a C function (pass by reference).

The CAR of the argument specification can be \texttt{NIL} (then it is an
input-only argument). Otherwise, it should be a variable which receives
the returned structure data.

The CADR of the argument specification must be a cons pair with the
total size of the structure in its CAR. The CDR is ignored for
input-only arguments, and should contain a
\emph{result specification} for the output value
to be stored in the variable.

For example, a minimal case is a function that takes an integer
reference, and stores the number `123' in that location:

\begin{wideverbatim}
void fun(int *i) {
   *i = 123;
}
\end{wideverbatim}

We call \texttt{native} with a variable \texttt{X} in the CAR of the
argument specification, a size of 4 (i.e. \texttt{sizeof(int)}), and
\texttt{I} for the result specification. The stored value is then
available in the variable \texttt{X}:

\begin{wideverbatim}
: (native "lib.so" "fun" NIL '(X (4 . I)))
-> NIL
: X
-> 123
\end{wideverbatim}

The rest (CDDR) of the argument specification may contain initialization
data, if the C function expects input values in the structure. It should
be a list of \emph{initialization items}, optionally
with a fill-byte value in the CDR of the last cell.

If there are \emph{no} initialization items and just the final fill-bye,
then the whole buffer is filled with that byte. For example, to pass a
buffer of 20 bytes, initialized to zero:

\begin{wideverbatim}
: (native "lib.so" "fun" NIL '(NIL (20) . 0))
\end{wideverbatim}

A buffer of 20 bytes, with the first 4 bytes initialized to 1, 2, 3, and
4, and the rest filled with zero:

\begin{wideverbatim}
: (native "lib.so" "fun" NIL '(NIL (20) 1 2 3 4 . 0))
\end{wideverbatim}

and the same, where the buffer contents are returned as a list of bytes
in the variable \texttt{X}:

\begin{wideverbatim}
: (native "lib.so" "fun" NIL '(X (20 B . 20) 1 2 3 4 . 0))
\end{wideverbatim}

For a more extensive example, let's use the following definitions:

\begin{wideverbatim}
typedef struct value {
   int x, y;
   double a, b, c;
   int z;
   char nm[4];
} value;

void fun(value *val) {
   printf("%d %d\n", val->x, val->y);
   val->x = 3;
   val->y = 4;
   strcpy(val->nm, "OK");
}
\end{wideverbatim}

We call this function with a structure of 40 bytes, requesting the
returned data in \texttt{V}, with two integers \texttt{(I . 2)}, three
doubles \texttt{(100 . 3)} with a scale of 2 (1.0 = 100), another
integer \texttt{I} and four characters \texttt{(C . 2)}. If the
structure gets initialized with two integers 7 and 6, three doubles
0.11, 0.22 and 0.33, and another integer 5 while the rest of the 40
bytes is cleared to zero

\begin{wideverbatim}
: (native "lib.so" "fun" NIL
   '(V (40 (I . 2) (100 . 3) I (C . 4)) -7 -6 (100 11 22 33) -5 . 0) )
\end{wideverbatim}

then it will print the integers 7 and 6, and \texttt{V} will contain the
returned list

\begin{wideverbatim}
((3 4) (11 22 33) 5 ("O" "K" NIL NIL))
\end{wideverbatim}

i.e. the original integer values 7 and 6 replaced with 3 and 4.

Note that the allocated structure memory is released \emph{after} the
return value is extracted. This allows a C function to return the
argument structure pointer, perhaps after modifying the data in-place,
and receive the new structure as the return value -- instead of (or even
in addition to) to the direct return via the argument reference.

% \begin{center}\rule{3in}{0.4pt}\end{center}

\section{Memory Management}
\label{sec:native-memory-management}

The preceding \emph{Arguments} section mentions that
\texttt{native} implicitly allocates and releases memory for strings,
arrays and structures.

Technically, this mimics \emph{automatic variables} in C.

For a simple example, let's assume that we want to call \texttt{read(2)}
directly, to fetch a 4-byte integer from a given file descriptor. This
could be done with the following C function:

\begin{wideverbatim}
int read4bytes(int fd) {
   char buf[4];

   read(fd, buf, 4);
   return *(int*)buf;
}
\end{wideverbatim}

\texttt{buf} is an automatic variable, allocated on the stack, which
disappears when the function returns. A corresponding \texttt{native}
call would be:

\begin{wideverbatim}
(native "@" "read" 'N Fd '(Buf (4 . I)) 4)
\end{wideverbatim}

The structure argument \texttt{(Buf (4 . I))} says that a space of 4
bytes should be allocated and passed to \texttt{read}, then an integer
\texttt{I} returned in the variable \texttt{Buf} (the return value of
\texttt{native} itself is the number returned by \texttt{read}). The
memory space is released after that.

(Note that we use \texttt{"@"} for the library here, as \texttt{read}
resides in the main program.)

Instead of a single integer, we might want a list of four bytes to be
returned from \texttt{native}:

\begin{wideverbatim}
(native "@" "read" 'N Fd '(Buf (4 B . 4)) 4)
\end{wideverbatim}

The difference is that we wrote \texttt{(B . 4)} (a list of 4 bytes)
instead of \texttt{I} (a single integer) for the \emph{result
  specification} (see the \emph{Arrays and Structures} section).

Let's see what happens if we extend this example. We'll write the four
bytes to another file descriptor, after reading them from the first one:

\begin{wideverbatim}
void copy4bytes(int fd1, int fd2) {
   char buf[4];

   read(fd1, buf, 4);
   write(fd2, buf, 4);
}
\end{wideverbatim}

Again, \texttt{buf} is an automatic variable. It is passed to both
\texttt{read} and \texttt{write}. A direct translation would be:

\begin{wideverbatim}
(native "@" "read" 'N Fd '(Buf (4 B . 4)) 4)
(native "@" "write" 'N Fd (cons NIL (4) Buf) 4)
\end{wideverbatim}

This work as expected. \texttt{read} returns a list of four bytes in
\texttt{Buf}. The call to \texttt{cons} builds the structure

\begin{wideverbatim}
(NIL (4) 1 2 3 4)
\end{wideverbatim}

i.e. no return variable, a four-byte memory area, filled with the four
bytes (assuming that \texttt{read} returned 1, 2, 3 and 4). Then this
structure is passed to \texttt{write}.

But: This solution induces quite some overhead. The four-byte buffer is
allocated before the call to \texttt{read} and released after that, then
allocated and released again for \texttt{write}. Also, the bytes are
converted to a list to be stored in \texttt{Buf}, then that list is
extended for the structure argument to \texttt{write}, and converted
again back to the raw byte array. The data in the list itself are never
used.

If the above operation is to be used more than once, it is better to
allocate the buffer manually, use it for both reading and writing, and
then release it. This also avoids all intermediate list conversions.

\begin{wideverbatim}
(let Buf (native "@" "malloc" 'N 4) # Allocate memory
   (native "@" "read" 'N Fd Buf 4)  # (Possibly repeat this several times)
   (native "@" "write" 'N Fd Buf 4)
   (native "@" "free" NIL Buf) )    # Release memory
\end{wideverbatim}

\subsection{Fast Fourier Transform}
\label{sec:native-fast-fourier-transform}

For a more typical example, we might call the Fast Fourier Transform
using the library from the \href{http://fftw.org}{FFTW} package. With
the example code for calculating Complex One-Dimensional DFTs:

\begin{wideverbatim}
#include <fftw3.h>
...
{
   fftw_complex *in, *out;
   fftw_plan p;
   ...
   in = (fftw_complex*) fftw_malloc(sizeof(fftw_complex) * N);
   out = (fftw_complex*) fftw_malloc(sizeof(fftw_complex) * N);
   p = fftw_plan_dft_1d(N, in, out, FFTW_FORWARD, FFTW_ESTIMATE);
   ...
   fftw_execute(p); /* repeat as needed */
   ...
   fftw_destroy_plan(p);
   fftw_free(in); fftw_free(out);
}
\end{wideverbatim}

we can build the following equivalent:

\begin{wideverbatim}
(load "@lib/math.l")

(de FFTW_FORWARD . -1)
(de FFTW_ESTIMATE . 64)

(de fft (Lst)
   (let
      (Len (length Lst)
         In (native "libfftw3.so" "fftw_malloc" 'N (* Len 16))
         Out (native "libfftw3.so" "fftw_malloc" 'N (* Len 16))
         P (native "libfftw3.so" "fftw_plan_dft_1d" 'N
            Len In Out FFTW_FORWARD FFTW_ESTIMATE ) )
      (struct In NIL (cons 1.0 (apply append Lst)))
      (native "libfftw3.so" "fftw_execute" NIL P)
      (prog1 (struct Out (make (do Len (link (1.0 . 2)))))
         (native "libfftw3.so" "fftw_destroy_plan" NIL P)
         (native "libfftw3.so" "fftw_free" NIL Out)
         (native "libfftw3.so" "fftw_free" NIL In) ) ) )
\end{wideverbatim}

This assumes that the argument list \texttt{Lst} is passed as a list of
complex numbers, each as a list of two numbers for the real and
imaginary part, like

\begin{wideverbatim}
(fft '((1.0 0) (1.0 0) (1.0 0) (1.0 0) (0 0) (0 0) (0 0) (0 0)))
\end{wideverbatim}

The above translation to Lisp is quite straightforward. After the two
buffers are allocated, and a plan is created, \texttt{struct} is called
to store the argument list in the \texttt{In} structure as a list of
double numbers (according to the \texttt{1.0}
\emph{initialization item}). Then
\texttt{fftw\_execute} is called, and \texttt{struct} is called again to
retrieve the result from \texttt{Out} and return it from \texttt{fft}
via the \texttt{prog1}. Finally, all memory is released.

\subsection{Constant Data}
\label{sec:native-constant-data}

If such allocated data (strings, arrays or structures passed to
\texttt{native}) are constant during the lifetime of a program, it makes
sense to allocate them only once, before their first use. A typical
candidate is the format string of a \texttt{printf} call. Consider a
function which prints a floating point number in scientific notation:

\begin{wideverbatim}
(load "@lib/math.l")

: (de prf (Flt)
   (native "@" "printf" NIL "%e^J" (cons Flt 1.0)) )
-> prf

: (prf (exp 12.3))
2.196960e+05
\end{wideverbatim}

As we know that the format string \texttt{"\%e\^{}J"} will be converted
from a Lisp symbol to a C string with \texttt{strdup} -- and then thrown
away -- on each call to \texttt{prf}, we might as well perform a little
optimization and delegate this conversion to the program load time:

\begin{wideverbatim}
: (de prf (Flt)
   (native "@" "printf" NIL `(native "@" "strdup" 'N "%e^J") (cons Flt 1.0)) )
-> prf

: (prf (exp 12.3))
2.196960e+05
\end{wideverbatim}

If we look at the \texttt{prf} function, we see that it now contains the
pointer to the allocated string memory:

\begin{wideverbatim}
: (pp 'prf)
(de prf (Flt)
   (native "@" "printf" NIL 24662032 (cons Flt 1000000)) )
-> prf
\end{wideverbatim}

This pointer will be used by \texttt{printf} directly, without any
further conversion or memory management.

% \begin{center}\rule{3in}{0.4pt}\end{center}

\section{Callbacks}
\label{sec:native-callbacks}

Sometimes it is necessary to do the reverse: Call Lisp code from C code.
This can be done in two ways -- with certain limitations.

\subsection{Call by Name}
\label{sec:native-callbacks}

The first way is actually not a
\href{http://en.wikipedia.org/wiki/Callback\_(computer\_programming)}{callback}
in the strict sense. It just allows to call a Lisp function with a given
name.

The limitation is that this function can accept only maximally five
numeric arguments, and returns a number.

The prerequisite is, of course, that you have access to the C source
code. To use it from C, insert the following prototype somewhere before
the first call:

\begin{wideverbatim}
long lisp(char*,long,long,long,long,long);
\end{wideverbatim}

Then you can call \texttt{lisp} from C:

\begin{wideverbatim}
long n = lisp("myLispFun", a, b, 0, 0, 0);
\end{wideverbatim}

The first argument should be the name of a Lisp function (built-in, or
defined in Lisp). It is searched for at runtime, so it doesn't need to
exist at the time the C library is compiled or loaded.

Be sure to pass dummy arguments (e.g. zero) if your function expects
less than five arguments, to keep the C compiler happy.

This mechanism can generally be used for any type of argument and return
value (not only \texttt{long}). On the C side, appropriate casts or a
adapted prototype should be used. It is then up to the called Lisp
function to prepare and/or extract the proper data with \texttt{struct}
and memory management operations.

\subsection{Function Pointer}
\label{sec:native-function-pointer}

This is a true
\href{http://en.wikipedia.org/wiki/Callback\_(computer\_programming)}{callback}
mechanism. It uses the Lisp-level function \texttt{lisp} (not to
confuse with the C-level function with the same name in the previous
section). No C source code access is required.

\texttt{lisp} returns a function pointer, which can be passed to C
functions via \texttt{native}. When this function pointer is
dereferenced and called from the C code, the corresponding Lisp function
is invoked. Here, too, only five numeric arguments and a numeric return
value can be used, and other data types must be handled by the Lisp
function with \texttt{struct} and memory management operations.

Callbacks are often used in user interface libraries, to handle key-,
mouse- and other events. Examples can be found in
\texttt{"@lib/openGl.l"}. The following function \texttt{mouseFunc}
takes a Lisp function, installs it under the tag \texttt{mouseFunc} (any
other tag would be all right too) as a callback, and passes the
resulting function pointer to the OpenGL \texttt{glutMouseFunc()}
function, to set it as a callback for the current window:

\begin{wideverbatim}
(de mouseFunc (Fun)
   (native `*GlutLib "glutMouseFunc" NIL (lisp 'mouseFunc Fun)) )
\end{wideverbatim}

(The global \texttt{*GlutLib} holds the library
\texttt{"/usr/lib/libglut.so"}. The backquote (\texttt{`}) is important
here, so that the transient symbol with the library name (and not the
global \texttt{*GlutLib}) is evaluated by \texttt{native}, resulting in
the proper library handle at runtime).

A program using OpenGL may then use \texttt{mouseFunc} to install a
function

\begin{wideverbatim}
(mouseFunc
   '((Btn State X Y)
      (do-something-with Btn State X Y) ) )
\end{wideverbatim}

so that future clicks into the window will pass the button, state and
coordinates to that function.

