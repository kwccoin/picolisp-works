\title{README Ersatz-PicoLisp}
\author{Alexander Burger}
% Use \authorrunning{Short Title} for an abbreviated version of
% your contribution title if the original one is too long
\institute{\texttt{abu@software-lab.de}}
%
% Use the package "url.sty" to avoid
% problems with special characters
% used in your e-mail or web address
%


\maketitle

\begin{abstract}
  This is the README file from the Ersatz PicoLisp distribution. 
\end{abstract}

\section{Ersatz PicoLisp}
\label{sec:ersatz-picolisp}

Ersatz PicoLisp is a version of PicoLisp completely written in Java.
It requires a 1.6 Java Runtime Environment.

It should be the last resort when there is no other way to run a
"real" PicoLisp. Also, it may be used to bootstrap the 64-bit version,
which requires a running PicoLisp to build from the sources.

Unfortunately, ErsatzLisp lacks everything which makes up "true"
PicoLisp: Speed, small memory footprint, and simple internal
structures.

Performance is rather poor. It is 5 to 10 times slower, allocates a
huge amount of memory at startup (600 MB vs. 3 MB), and needs 2.5 to 4
times the space for runtime Lisp data. But efficiency was not a major
goal. Instead, performance was often sacrificed in favor of simpler or
more modular structures.

There is no support for

\begin{itemize}
\item raw console input ('key') and line editing
\item child processes ('fork')
\item interprocess communication ('tell', 'hear', 'ipc', 'udp' etc.)
\item databases (external symbols)
\item signal handling
\end{itemize}


\subsection{Invocation}
\label{sec:invocation}

Ersatz PicoLisp can be started - analog to 'pil' - as

\begin{wideverbatim}
   $ ersatz/pil
\end{wideverbatim}

This includes slighly simplfied versions of the standard libraries as loaded by
the "real" 'pil' (without database, but with Pilog and XML support).

To start it in debug mode, use

\begin{wideverbatim}
   $ ersatz/pil +
\end{wideverbatim}

On non-Unix systems, you might start 'java' directly, e.g.:

\begin{wideverbatim}
   java -DPID=42 -cp .;tmp;picolisp.jar PicoLisp lib.l
\end{wideverbatim}

Instead of '42' some other number may be passed. It is used to simulate a
"process ID", so it should be different for every running instance of Ersatz
PicoLisp.


\subsection{Building the JAR file}
\label{sec:building-the-jar-file}

The actual source files are

\begin{wideverbatim}
   sys.src  # The system
   fun.src  # Function definitions
\end{wideverbatim}

The PicoLisp script "mkJar" will read them, generate the Java source
file "PicoLisp.java", compile that with 'javac', and pack the result
into a JAR (Java Archive) file. "mkJar" expects to be run in the
"ersatz/" directory, e.g.:

\begin{wideverbatim}
   $ (cd ersatz; ./mkJar)
\end{wideverbatim}

