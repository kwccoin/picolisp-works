\title{A PicoLisp Tutorial}
\author{Alexander Burger}
% Use \authorrunning{Short Title} for an abbreviated version of
% your contribution title if the original one is too long
\institute{\texttt{abu@software-lab.de}}
%
% Use the package "url.sty" to avoid
% problems with special characters
% used in your e-mail or web address
%


\maketitle


\begin{abstract}
  This article demonstrates some aspects of the PicoLisp system in
  detail and example. For a general description of the PicoLisp kernel
  please look at the \emph{PicoLisp Reference}.

  This is not a Lisp tutorial, as it assumes some basic knowledge of
  programming, Lisp, and even PicoLisp. Please read these sections in
  the \emph{PicoLisp Reference} before coming back here:
  \emph{Introduction} and \emph{The PicoLisp Machine}. This tutorial
  concentrates on the specificities of PicoLisp, and its differences
  with other Lisp dialects.
\end{abstract}

\section{Now let's start}
\label{sec:tut-now-lets-start}

If not stated otherwise, all examples assume that PicoLisp was started
from a global installation (see \href{ref.html\#inst}{Installation})
from the shell prompt as

\begin{verbatim}
$ pil +
:
\end{verbatim}

It loads the PicoLisp base system and the debugging environment, and
waits for you to enter input lines at the interpreter prompt
(\texttt{:}). You can terminate the interpreter and return to the shell
at any time, by either hitting the \texttt{Ctrl-D} key, or by executing
the function \texttt{(bye)}.

Please note that special handling is done during character input. This
one is incompatible with \texttt{rlwrap} for example but is more
powerful.

\begin{itemize}
\item
  \texttt{vi}-like command-line editing (typos fixes and history with
  ESC, \texttt{h}, \texttt{j}, \texttt{k} and \texttt{l} but not
  arrows),
\item
  auto-formating (underlined) of double-quoted strings (don't try and
  struggle to make \texttt{"} appear).
\end{itemize}

If you prefer to use Emacs, please use the picolisp-mode bundled in
``@lib/el''.

If you feel that you absolutely have to use an IDE, \texttt{rlwrap} or
another input front-end, please remove the entry ``@lib/led.l'' from
``lib.l'' and ``dbg.l''. Note that in this case, however, you will not
have the TAB symbol completion feature available during command line
editing.

\section{Command Line Editing}
\label{sec:tut-command-line-editing}

PicoLisp permanently reads input from the current input channel (i.e.
the console in interactive mode), evaluates it, and prints the result to
the current output channel. This is called a ``read-eval-print-loop''
(REPL).

\subsection{VI-like editing}
\label{sec:tut-vi-like-editing}


It is very helpful - though not absolutely necessary - when you know how
to use the \texttt{vi} editor.

To alleviate the task of manual line input, a command line editor is
provided which is similar to (though much simpler than) the \texttt{readline}
feature of the \texttt{bash} shell. Only a subset of the \texttt{vi} mode is
supported, which is restricted to single-key commands (the ``real'' \texttt{vi}
supports multi-key commands and the modification of most commands with
count prefixes). It is loaded at startup in debug mode, you find its
source in ``lib/led.l''.

You can enter lines in the normal way, correcting mistypes with the
BACKSPACE key, and terminating them with the ENTER key. This is the
\emph{Insert Mode}.

If you hit ESC, you get into \emph{Command Mode}. Now you can navigate
horizontally in the current input line, or vertically in the history of
previously entered lines, with key commands borrowed from the \texttt{vi}
editor (only \texttt{h}, \texttt{j}, \texttt{k} and \texttt{l} and not arrows). Note, however, that
there is always only a single line visible.

Let's say you did some calculation


\begin{wideverbatim}
: (* (+ 2 3) (- 7 2))
-> 25
:
\end{wideverbatim}

If you want to repeat a modified version of this command, using \texttt{8}
instead of \texttt{7}, you don't have to re-type the whole command, but type

\begin{itemize}
\item ESC to get into \emph{Command Mode}
\item \texttt{k} to get one line ``up''
\item \texttt{f} and \texttt{7} to ``find'' the character \texttt{7}
\item \texttt{r} and \texttt{8} to ``replace'' with \texttt{8}
\end{itemize}

Then you hit ENTER to execute the modified line. Instead of jumping to
the \texttt{7} with the ``find'' command, you may also type \texttt{l} (move ``right'')
repeatedly till you reach the correct position.

The key commands in the \emph{Command Mode} are listed below. Some commands
change the mode back to \emph{Insert Mode} as indicated in parentheses.
Deleting or changing a ``word'' take either the current atom (number or
symbol), or a whole expression when the cursor is at a left parenthesis.

\begin{itemize}
\item \texttt{k} - Go up one line
\item \texttt{j} - Go down one line
\item \texttt{l} - Go right one character
\item \texttt{h} - Go left one character
\item \texttt{w} - Go right one word
\item \texttt{b} - Go back (left) one word
\item \texttt{0} - Go to the beginning of the line
\item \texttt{\$} - Go to the end of the line
\item \texttt{i} - Enter \emph{Insert Mode} at the cursor position
\item \texttt{a} - Append (\emph{Insert Mode}) after the cursor position
\item \texttt{A} - Append (\emph{Insert Mode}) at the end of the line
\item \texttt{I} - Insert (\emph{Insert Mode}) at the beginning of the line
\item \texttt{x} - Delete the character at the cursor position
\item \texttt{X} - Delete the character left of the cursor position
\item \texttt{r} - Replace the character at the cursor position with the next key
\item \texttt{s} - Substitute the character at the cursor position (\emph{Insert Mode})
\item \texttt{S} - Substitute the whole line (\emph{Insert Mode})
\item \texttt{d} - Delete the word at the cursor position (\emph{Insert Mode})
\item \texttt{D} - Delete the rest of the line
\item \texttt{c} - Change the word at the cursor position (\emph{Insert Mode})
\item \texttt{C} - Change the rest of the line (\emph{Insert Mode})
\item \texttt{f} - Find next key in the rest of the current line
\item \texttt{p} - Paste data deleted with \texttt{x}, \texttt{X}, \texttt{d} or \texttt{D} after the cursor
   position
\item \texttt{P} - Paste data deleted with \texttt{x}, \texttt{X}, \texttt{d} or \texttt{D} before the cursor
   position
\item \texttt{/} - Accept an input pattern and search the history for it
\item \texttt{n} - Search for next occurrence of pattern (as entered with \texttt{/})
\item \texttt{N} - Search for previous occurrence of pattern
\item \texttt{\%} - Go to matching parenthesis
\item \texttt{\textasciitilde{}} - Convert character to opposite (lower or upper) case and move
   right
\item \texttt{u} - Undo the last change (one level only)
\item \texttt{U} - Undo all changes of the current line
\item \texttt{g} - Display current contents of cut buffer (not in \texttt{vi})
\end{itemize}

Notes:

\begin{itemize}
\item The \texttt{d} command corresponds to the \texttt{dw} command of the \texttt{vi} editor,
   and \texttt{c} corresponds to \texttt{cw}.
\item Search patterns may contain ` \texttt{@} ' characters as wildcards.
\item Lines shorter than 3 characters, lines beginning with a space
   character, or duplicate lines are not entered into the history.
\item The history is stored in the file ``.pil/history'' in the user's home
   directory. The length of the history is limited to 1000 lines.
\end{itemize}

The following two key-combinations work both in Insert and Command Mode:

\begin{itemize}
\item \texttt{Ctrl-D} will immediately terminate the current process.
\item \texttt{Ctrl-X} discards all input, abandons further processing, and returns
   to the interpreter's top level (equivalent to invoking \texttt{quit}). This
   is also useful when the program stopped at a breakpoint (see
   single-stepping \emph{Debugging}), or after program execution was
   interrupted with \texttt{Ctrl-C}.
\end{itemize}

Besides these two keys, in \emph{Insert Mode} only the following keys have a
special meaning:

\begin{itemize}
\item BACKSPACE (\texttt{Ctrl-H}) and DEL erase the character to the left
\item \texttt{Ctrl-V} inserts the next key literally
\item TAB performs symbol and/or path completion: When a symbol (or path)
   name is entered partially and TAB is pressed subsequently, all
   internal symbols (and/or path names in the file system) matching the
   partial input are shown in sequence.
\item ESC terminates \emph{Input Mode} and enters \emph{Command Mode}
\end{itemize}

 
\subsection{Conclusion}
\label{sec:tut-conclusion}


Please take some time to experiment and to get used to command line
editing. It will make life much easier in the future :-)


 
\section{Browsing}
\label{sec:tut-browsing}


PicoLisp provides some functionality for inspecting pieces of data and
code within the running system.

 
\subsection{Basic tools}
\label{sec:tut-basic-tools}


The really basic tools are of course available and their name alone is
enough to know: \texttt{print}, \texttt{size} \ldots{}

But you will appreciate some more powerful tools like:

\begin{itemize}
\item \texttt{match}, a predicate which compares S-expressions with bindable
   wildcards when matching,
\end{itemize}

 
\subsection{Inspect a symbol with \emph{show}}
\label{sec:tut-inspect-a-symbol-with-show}


The most commonly used tool is probably the \texttt{show} function. It takes a
symbolic argument, and shows the symbol's name (if any), followed by its
value cell, and then the contents of the property list on the following
lines (assignment of such things to a symbol can be done with \texttt{set},
\texttt{setq}, and \texttt{put}).


\begin{wideverbatim}
: (setq A '(This is the value))  # Set the value cell of 'A'
-> (This is the value)
: (put 'A 'key1 'val1)           # Store property 'key1'
-> val1
: (put 'A 'key2 'val2)           # and 'key2'
-> val2
: (show 'A)                      # Now 'show' the symbol 'A'
A (This is the value)
   key2 val2
   key1 val1
-> A
\end{wideverbatim}

\texttt{show} accepts an arbitrary number of arguments which are processed
according to the rules of \texttt{get}, resulting in a symbol which is showed
then.


\begin{wideverbatim}
: (put 'B 'a 'A)        # Put 'A' under the 'a'-property of 'B'
-> A
: (setq Lst '(A B C))   # Create a list with 'B' as second argument
-> (A B C)
: (show Lst 2 'a)       # Show the property 'a of the 2nd element of 'Lst'
A (This is the value)   # (which is 'A' again)
   key2 val2
   key1 val1
-> A
\end{wideverbatim}

 
\subsection{Inspect and edit with \emph{edit}}
\label{sec:tut-inspect-and-edit-with-edit}


Similar to \texttt{show} is \texttt{edit}. It takes an arbitrary number of symbolic
arguments, writes them to a temporary file in a format similar to
\texttt{show}, and starts the \texttt{vim} editor with that file.


\begin{wideverbatim}
: (edit 'A 'B)
\end{wideverbatim}

The \texttt{vim} window will look like


\begin{wideverbatim}
A (This is the value)
key1 val1
key2 val2

(********)

B NIL
a A  # (This is the value)

(********)
\end{wideverbatim}

Now you can modify values or properties. You should not touch the
parenthesized asterisks, as they serve as delimiters. If you position
the cursor on the first character of a symbol name and type  \texttt{K} 
(``Keyword lookup''), the editor will be restarted with that symbol added
to the editor window.  \texttt{Q}  (for ``Quit'') will bring you back to the
previous view.

\texttt{edit} is also very useful to browse in a database. You can follow the
links between objects with  \texttt{K} , and even - e.g. for low-level repairs
\begin{itemize}
\item modify the data (but only if you are really sure about what you are
\end{itemize}
doing, and don't forget to \texttt{commit} when you are done).

 
\subsection{Built-in pretty print with \emph{pp}}
\label{sec:tut-built-in-pretty-print-with-pp}


The \emph{pretty-print} function \texttt{pp} takes a symbol that has a function
defined (or two symbols that specify message and class for a method
definition), and displays that definition in a formatted and indented
way.


\begin{wideverbatim}
: (pp 'pretty)
(de pretty (X N . @)
   (setq N (abs (space (or N 0))))
   (while (args) (printsp (next)))
   (if (or (atom X) (>= 12 (size X)))
      (print X)
      (while (== 'quote (car X))
         (prin "'")
         (pop 'X) )
      (let Z X
         (prin "(")
         (cond
            ((memq (print (pop 'X)) *PP)
               (cond
                  ((memq (car Z) *PP1)
                     (if (and (pair (car X)) (pair (cdar X)))
                        (when (>= 12 (size (car X)))
                           (space)
                           (print (pop 'X)) )
                        (space)
                        (print (pop 'X))
                        (when
                           (or
                              (atom (car X))
                              (>= 12 (size (car X))) )
                           (space)
                           (print (pop 'X)) ) ) )
                  ((memq (car Z) *PP2)
                     (inc 'N 3)
                     (loop
                        (prinl)
                        (pretty (cadr X) N (car X))
                        (NIL (setq X (cddr X)) (space)) ) )

\end{wideverbatim}

\begin{wideverbatim}


                  ((or (atom (car X)) (>= 12 (size (car X))))
                     (space)
                     (print (pop 'X)) ) ) )
            ((and (memq (car Z) *PP3) (>= 12 (size (head 2 X))))
               (space)
               (print (pop 'X) (pop 'X)) ) )
         (when X
            (loop
               (T (== Z X) (prin " ."))
               (T (atom X) (prin " . ") (print X))
               (prinl)
               (pretty (pop 'X) (+ 3 N))
               (NIL X) )
            (space) )
         (prin ")") ) ) )
-> pretty
\end{wideverbatim}

The style is the same as we use in source files:

\begin{itemize}
\item The indentation level is three spaces
\item If a list is too long (to be precise: if its \texttt{size} is greater than
   12), pretty-print the CAR on the current line, and each element of
   the CDR recursively on its own line.
\item A closing parenthesis a preceded by a space if the corresponding open
   parenthesis is not on the same line
\end{itemize}

 
\subsection{Inspect elements one by one with \emph{more}}
\label{sec:tut-inspect-elements-one-by-one-with-more}


\texttt{more} is a simple tool that displays the elements of a list one by one.
It stops after each element and waits for input. If you just hit ENTER,
\texttt{more} continues with the next element, otherwise (usually I type a dot
(\texttt{.}) followed by ENTER) it terminates.


\begin{wideverbatim}
: (more (1 2 3 4 5 6))
1                          # Hit ENTER
2.                         # Hit '.' and ENTER
-> T                       # stopped
\end{wideverbatim}

Optionally \texttt{more} takes a function as a second argument and applies that
function to each element (instead of the default \texttt{print}). Here, often
\texttt{show} or \texttt{pp} (see below) is used.


\begin{wideverbatim}
: (more '(A B))            # Step through 'A' and 'B'
A
B
-> NIL
: (more '(A B) show)       # Step through 'A' and 'B' with 'show'
A (This is the value)      # showing 'A'
   key2 val2
   key1 val1
                           # Hit ENTER
B NIL                      # showing 'B'
   a A
-> NIL
\end{wideverbatim}

 
\subsection{Search through available symbols with \emph{what}}
\label{sec:tut-search-through-available-symbols-with-what}


The \texttt{what} function returns a list of all internal symbols in the system
which match a given pattern (with  \texttt{@}  wildcard characters).


\begin{wideverbatim}
: (what "prin@")
-> (prin print prinl print> printsp println)
\end{wideverbatim}

 
\subsection{Search through values or properties of symbols with \emph{who}}
\label{sec:tut-search-through-values-or-properties-of-symbols-with-can}


The function \texttt{can} returns a list which indicates which classes \emph{can}
accept a given message. Again, this list is suitable for iteration with
\texttt{pp}:


\begin{wideverbatim}
: (can 'del>)                                   # Which classes accept 'del>' ?
-> ((del> . +List) (del> . +Entity) (del> . +relation))

: (more (can 'del>) pp)                         # Inspect the methods with 'pp'
(dm (del> . +List) (Obj Old Val)
   (and ((<> Old Val) (delete Val Old)) )

(dm (del> . +Entity) (Var Val)
   (when
      (and
         Val
         (has> (meta This Var) Val (get This Var)) )

\end{wideverbatim}

\begin{wideverbatim}


      (let Old (get This Var)
         (rel>
            (meta This Var)
            This
            Old
            (put This Var (del> (meta This Var) This Old @)) )
         (when (asoq Var (meta This 'Aux))
            (relAux This Var Old (cdr @)) )
         (upd> This Var Old) ) ) )

(dm (del> . +relation) (Obj Old Val)
   (and ((<> Old Val) Val) )
\end{wideverbatim}

 
\subsection{Inspect dependencies with \emph{dep}}
\label{sec:tut-inspect-dependencies-with-dep}


\texttt{dep} shows the dependencies in a class hierarchy. That is, for a given
class it displays the tree of its (super)class(es) above it, and the
tree of its subclasses below it.

To view the complete hierarchy of input fields, we start with the root
class \texttt{+relation}:


\begin{wideverbatim}
: (dep '+relation)
+relation
   +Bag
   +Any
   +Blob
   +Link
      +Joint
   +Bool
   +Symbol
      +String
   +Number
      +Time
      +Date
-> +relation
\end{wideverbatim}

If we are interested in \texttt{+Link}:


\begin{wideverbatim}
: (dep '+Link)
   +relation
+Link
   +Joint
-> +Link
\end{wideverbatim}

This says that \texttt{+Link} is a subclass of \texttt{+relation}, and has a single
subclass (\texttt{+Joint}).


 
\section{Defining Functions}
\label{sec:tut-defining-functions}


Most of the time during programming is spent defining functions (or
methods). In the following we will concentrate on functions, but most
will be true for methods as well except for using \texttt{dm} instead of \texttt{de}.

 
\subsection{Functions with no argument}
\label{sec:tut-functions-with-no-argument}


The notorious ``Hello world'' function must be defined:


\begin{wideverbatim}
: (de hello ()
   (prinl "Hello world") )
-> hello
\end{wideverbatim}

The \texttt{()} in the first line indicates a function without arguments. The
body of the function is in the second line, consisting of a single
statement. The last line is the return value of \texttt{de}, which here is the
defined symbol. From now on we will omit the return values of examples
when they are unimportant.

Now you can call this function this way:


\begin{wideverbatim}
: (hello)
Hello world
\end{wideverbatim}

 
\subsection{Functions with one argument}
\label{sec:tut-functions-with-one-argument}


A function with an argument might be defined this way:


\begin{wideverbatim}
: (de hello (X)
   (prinl "Hello " X) )
# hello redefined
-> hello
\end{wideverbatim}

PicoLisp informs you that you have just redefined the function. This
might be a useful warning in case you forgot that a bound symbol with
that name already existed.


\begin{wideverbatim}
: (hello "world")
Hello world
\end{wideverbatim}


\begin{wideverbatim}
: (hello "Alex")
Hello Alex
\end{wideverbatim}

 
\subsection{Preventing arguments evaluation and variable number of arguments}
\label{sec:tut-preventing-arguments-evaluation-and-variable-number-of-arguments}


Normally, PicoLisp evaluates the arguments before it passes them to a
function:


\begin{wideverbatim}
: (hello (+ 1 2 3))
Hello 6
\end{wideverbatim}


\begin{wideverbatim}
: (setq A 1  B 2)       # Set 'A' to 1 and 'B' to 2
-> 2
: (de foo (X Y)         # 'foo' returns the list of its arguments
   (list X Y) )
-> foo
: (foo A B)             # Now call 'foo' with 'A' and 'B'
-> (1 2)                # -> We get a list of 1 and 2, the values of 'A' and 'B'
\end{wideverbatim}

In some cases you don't want that. For some functions (\texttt{setq} for
example) it is better if the function gets all arguments unevaluated,
and can decide for itself what to do with them.

For such cases you do not define the function with a \emph{list} of
parameters, but give it a \emph{single atomic} parameter instead. PicoLisp
will then bind all (unevaluated) arguments as a list to that parameter.


\begin{wideverbatim}
: (de foo X
   (list (car X) (cadr X)) )        # 'foo' lists the first two arguments

: (foo A B)                         # Now call it again
-> (A B)                            # -> We don't get '(1 2)', but '(A B)'

: (de foo X
   (list (car X) (eval (cadr X))) ) # Now evaluate only the second argument

: (foo A B)
-> (A 2)                            # -> We get '(A 2)'
\end{wideverbatim}

 
\subsection{Mixing evaluated arguments and variable number of unevaluated}
\label{sec:tut-mixing-evaluated-arguments-and-variable-number-of-unevaluated}

arguments

As a logical consequence, you can combine these principles. To define a
function with 2 evaluated and an arbitrary number of unevaluated
arguments:


\begin{wideverbatim}
: (de foo (X Y . Z)     # Evaluate only the first two args
   (list X Y Z) )

: (foo A B C D E)
-> (1 2 (C D E))        # -> Get the value of 'A' and 'B' and the remaining list
\end{wideverbatim}

 
\subsection{Variable number of evaluated arguments}
\label{sec:tut-variable-number-of-evaluated-arguments}


More common, in fact, is the case where you want to pass an arbitrary
number of \emph{evaluated} arguments to a function. For that, PicoLisp
recognizes the symbol \texttt{@} as a single atomic parameter and remembers all
evaluated arguments in an internal frame. This frame can then be
accessed sequentially with the \texttt{args}, \texttt{next}, \texttt{arg} and \texttt{rest}
functions.


\begin{wideverbatim}
: (de foo @
   (list (next) (next)) )     # Get the first two arguments

: (foo A B)
-> (1 2)
\end{wideverbatim}

Again, this can be combined:


\begin{wideverbatim}
: (de foo (X Y . @)
   (list X Y (next) (next)) ) # 'X' and 'Y' are fixed arguments

: (foo A B (+ 3 4) (* 3 4))
-> (1 2 7 12)                 # All arguments are evaluated
\end{wideverbatim}

These examples are not very useful, because the advantage of a variable
number of arguments is not used. A function that prints all its
evaluated numeric arguments, each on a line followed by its squared
value:


\begin{wideverbatim}
: (de foo @
   (while (args)                            # Check if there are some args left
      (println (next) (* (arg) (arg))) ) )  # Call the last arg (next) returned

: (foo (+ 2 3) (- 7 1) 1234 (* 9 9))
5 25
6 36
1234 1522756
81 6561
-> 6561
\end{wideverbatim}

This next example shows the behaviour of \texttt{args} and \texttt{rest}:


\begin{wideverbatim}
: (de foo @
   (while (args)
      (next)
      (println (arg) (args) (rest)) ) )
: (foo 1 2 3)
1 T (2 3)
2 T (3)
3 NIL NIL
\end{wideverbatim}

Finally, it is possible to pass all these evaluated arguments to another
function, using \texttt{pass}:


\begin{wideverbatim}
: (de foo @
   (pass println 9 8 7)       # First print all arguments preceded by 9, 8, 7
   (pass + 9 8 7) )           # Then add all these values

: (foo (+ 2 3) (- 7 1) 1234 (* 9 9))
9 8 7 5 6 1234 81             # Printing ...
-> 1350                       # Return the result
\end{wideverbatim}

 
\subsection{Anonymous functions without the \emph{lambda} keyword}
\label{sec:tut-anonymous-functions-without-the-lambda}


There's no distinction between code and data in PicoLisp, \texttt{quote} will
do what you want (see also \emph{this FAQ entry}).


\begin{wideverbatim}
: ((quote (X) (* X X)) 9)
-> 81
\end{wideverbatim}


\begin{wideverbatim}
: (setq f '((X) (* X X)))  # This is equivalent to (de f (X) (* X X))
-> ((X) (* X X))
: f
-> ((X) (* X X))
: (f 3)
-> 9
\end{wideverbatim}

 
\section{Debugging}
\label{sec:tut-debugging}


There are two major ways to debug functions (and methods) at runtime:
\emph{Tracing} and \emph{single-stepping}.

In this section we will use the REPL to explore the debugging
facilities, but in the \emph{Scripting} section, you will learn
how to launch PicoLisp scripts with some selected functions debugged:


\begin{wideverbatim}
$ pil app/file1.l -"trace 'foo" -main -"debug 'bar" app/file2.l +
\end{wideverbatim}

 
\subsection{Tracing}
\label{sec:tut-tracing}


\emph{Tracing} means letting functions of interest print their name and
arguments when they are entered, and their name again and the return
value when they are exited.

For demonstration, let's define the unavoidable factorial function (or
just \texttt{load} the file ``\texttt{@doc/fun.l}''):


\begin{wideverbatim}
(de fact (N)
   (if (=0 N)
      1
      (* N (fact (dec N))) ) )
\end{wideverbatim}

With \texttt{trace} we can put it in trace mode:


\begin{wideverbatim}
: (trace 'fact)
-> fact
\end{wideverbatim}

Calling \texttt{fact} now will display its execution trace.


\begin{wideverbatim}
: (fact 3)
 fact : 3
  fact : 2
   fact : 1
    fact : 0
    fact = 1
   fact = 1
  fact = 2
 fact = 6
-> 6
\end{wideverbatim}

As can be seen here, each level of function call will indent by an
additional space. Upon function entry, the name is separated from the
arguments with a colon (\texttt{:}), and upon function exit with an equals sign
 \texttt{=}  from the return value.

\texttt{trace} works by modifying the function body, so generally it works only
for functions defined as lists (lambda expressions, see
\emph{Evaluation}). Tracing a C-function is possible, however,
when it is a function that evaluates all its arguments.

So let's trace the functions \texttt{=0} and \texttt{*}:


\begin{wideverbatim}
: (trace '=0)
-> =0
: (trace '*)
-> *
\end{wideverbatim}

If we call \texttt{fact} again, we see the additional output:


\begin{wideverbatim}
: (fact 3)
 fact : 3
  =0 : 3
  =0 = NIL
  fact : 2
   =0 : 2
   =0 = NIL
   fact : 1
    =0 : 1
    =0 = NIL
    fact : 0
     =0 : 0
     =0 = 0
    fact = 1
    * : 1 1
    * = 1
   fact = 1
   * : 2 1
   * = 2
  fact = 2
  * : 3 2
  * = 6
 fact = 6
-> 6
\end{wideverbatim}

To reset a function to its untraced state, call \texttt{untrace}:


\begin{wideverbatim}
: (untrace 'fact)
-> fact
: (untrace '=0)
-> =0
: (untrace '*)
-> *
\end{wideverbatim}

or simply use \texttt{mapc}:

\begin{wideverbatim}
: (mapc untrace '(fact =0 *))
-> *
\end{wideverbatim}

 
\subsection{Single-stepping}
\label{sec:tut-single-stepping}


\emph{Single-stepping} means to execute a function step by step, giving the
programmer an opportunity to look more closely at what is happening. The
function \texttt{debug} inserts a breakpoint into each top-level expression of
a function. When the function is called, it stops at each breakpoint,
displays the expression it is about to execute next (this expression is
also stored into the global variable \texttt{\textasciicircum{}}) and enters a read-eval-loop.
The programmer can then

\begin{itemize}
\item inspect the current environment by typing variable names or calling
   functions
\item execute \texttt{(d)} to recursively debug the next expression (looping
   through subexpressions of this expression)
\item execute \texttt{(e)} to evaluate the next expression, to see what will
   happen without actually advancing on
\item type ENTER (that is, enter an empty line) to leave the read-eval loop
   and continue with the next expression
\end{itemize}

Thus, in the simplest case, single-stepping consists of just hitting
ENTER repeatedly to step through the function.

To try it out, let's look at the \texttt{stamp} system function. You may need
to have a look at

\begin{itemize}
\item \texttt{=T} (T test),
\item \texttt{date} and \texttt{time} (grab system date and time)
\item \texttt{default} (conditional assignments)
\item \texttt{pack} (kind of concatenation), and
\item \texttt{dat\$} and \texttt{tim\$} (date and time formats)
\end{itemize}

to understand this definition.


\begin{wideverbatim}
: (pp 'stamp)
(de stamp (Dat Tim)
   (and (=T Dat) (setq Dat (date T)))
   (default Dat (date) Tim (time T))
   (pack (dat$ Dat "-") " " (tim$ Tim T)) )
-> stamp
\end{wideverbatim}


\begin{wideverbatim}
: (debug 'stamp)                       # Debug it
-> T
: (stamp)                              # Call it again
(and (=T Dat) (setq Dat (date T)))     # stopped at first expression
!                                      # ENTER
(default Dat (date) Tim (time T))      # second expression
!                                      # ENTER
(pack (dat$ Dat "-") " " (tim$ ...     # third expression
! Tim                                  # inspect 'Tim' variable
-> 41908
! (time Tim)                           # convert it
-> (11 38 28)
!                                      # ENTER
-> "2004-10-29 11:38:28"               # done, as there are only 3 expressions
\end{wideverbatim}

Now we execute it again, but this time we want to look at what's
happening inside the second expression.


\begin{wideverbatim}
: (stamp)                              # Call it again
(and (=T Dat) (setq Dat (date T)))
!                                      # ENTER
(default Dat (date) Tim (time T))
!                                      # ENTER
(pack (dat$ Dat "-") " " (tim$ ...     # here we want to look closer
! (d)                                  # debug this expression
-> T
!                                      # ENTER
(dat$ Dat "-")                         # stopped at first subexpression
! (e)                                  # evaluate it
-> "2004-10-29"
!                                      # ENTER
(tim$ Tim T)                           # stopped at second subexpression
! (e)                                  # evaluate it
-> "11:40:44"
!                                      # ENTER
-> "2004-10-29 11:40:44"               # done
\end{wideverbatim}

The breakpoints still remain in the function body. We can see them when
we pretty-print it:


\begin{wideverbatim}
: (pp 'stamp)
(de stamp (Dat Tim)
   (! and (=T Dat) (setq Dat (date T)))
   (! default Dat (date) Tim (time T))
   (! pack
      (! dat$ Dat "-")
      " "
      (! tim$ Tim T) ) )
-> stamp
\end{wideverbatim}

To reset the function to its normal state, call \texttt{unbug}:


\begin{wideverbatim}
: (unbug 'stamp)
\end{wideverbatim}

Often, you will not want to single-step a whole function. Just use
\texttt{edit} (see above) to insert a single breakpoint (the exclamation mark
followed by a space) as CAR of an expression, and run your program.
Execution will then stop there as described above; you can inspect the
environment and continue execution with ENTER when you are done.

 
\section{Functional I/O}
\label{sec:tut-functional-i/o}


Input and output in PicoLisp is functional, in the sense that there are
not variables assigned to file descriptors, which need then to be passed
to I/O functions for reading, writing and closing. Instead, these
functions operate on implicit input and output channels, which are
created and maintained as dynamic environments.

Standard input and standard output are the default channels. Try reading
a single expression:


\begin{wideverbatim}
: (read)
(a b c)        # Console input
-> (a b c)
\end{wideverbatim}

To read from a file, we redirect the input with \texttt{in}. Note that comments
and whitespace are automatically skipped by \texttt{read}:


\begin{wideverbatim}
: (in "@doc/fun.l" (read))
-> (de fact (N) (if (=0 N) 1 (* N (fact (dec N)))))
\end{wideverbatim}

The \texttt{skip} function can also be used directly. To get the first
non-white character in the file with \texttt{char}:


\begin{wideverbatim}
: (in "@doc/fun.l" (skip "#") (char))
-> "("
\end{wideverbatim}

\texttt{from} searches through the input stream for given patterns. Typically,
this is not done with Lisp source files (there are better ways), but for
a simple example let's extract all items immediately following \texttt{fact} in
the file,


\begin{wideverbatim}
: (in "@doc/fun.l" (while (from "fact ") (println (read))))
(N)
(dec N)
\end{wideverbatim}

or the word following ``(de'' with \texttt{till}:


\begin{wideverbatim}
: (in "@doc/fun.l" (from "(de ") (till " " T)))
-> "fact"
\end{wideverbatim}

With \texttt{line}, a line of characters is read, either into a single
\emph{transient} symbol (the type used by PicoLisp
for strings),


\begin{wideverbatim}
: (in "@doc/tut.html" (line T))
-> "<!DOCTYPE HTML PUBLIC "-//W3C//DTD HTML 4.0 Transitional//EN" "http://..."
\end{wideverbatim}

or into a list of symbols (characters):


\begin{wideverbatim}
: (in "@doc/tut.html" (line))
-> ("<" "!" "D" "O" "C" "T" "Y" "P" "E" " " "H" "T" "M" "L" ...
\end{wideverbatim}

\texttt{line} is typically used to read tabular data from a file. Additional
arguments can split the line into fixed-width fields, as described in
the \texttt{reference manual}. If, however, the data are of variable width,
delimited by some special character, the \texttt{split} function can be used to
extract the fields. A typical way to import the contents of such a file
is:


\begin{wideverbatim}
(load "@lib/import.l")

(in '("bin/utf2" "importFile.txt")              # Pipe: Convert to UTF-8
   (until (eof)                                 # Process whole file
      (let L (split (line) "^I")                # TAB-delimited data
         ... use 'getStr', 'getNum' etc ...     # process them
\end{wideverbatim}

Some more examples with \texttt{echo}:


\begin{wideverbatim}
(in "a"                                         # Copy the first 40 Bytes
   (out "b"                                     # from file "a" to file "b"
      (echo 40) ) )

(in "@doc/tut.html"                             # Show the HTTP-header
   (line)
   (echo "<body>") )

(out "file.mac"                                 # Convert to Macintosh
   (in "file.txt"                               # from Unix or DOS format:
      (while (char)
         (prin
            (case @
               ("^M" NIL)                       # ignore CR
               ("^J" "^M")                      # convert CR to LF
               (T @) ) ) ) ) )                  # otherwise no change

(out "c"                                        # Merge the contents of "a"
   (in "b"                                      # and "b" into "c"
      (in "a"
         (while (read)                          # Read an item from "a",
            (println @ (in -1 (read))) ) ) ) )  # print it with an item from "b"
\end{wideverbatim}

 
\section{Scripting}
\label{sec:tut-scripting}


There are two possibilities to get the PicoLisp interpreter into doing
useful work: via command line arguments, or as a stand-alone script.

 
\subsection{Command line arguments for the PicoLisp interpreter}
\label{sec:tut-command-line-arguments-for-the-picolisp-interpreter}


The command line can specify either files for execution, or arbitrary
Lisp expressions for direct evaluation (see
\emph{Invocation}): if an argument starts with a hyphen, it
is evaluated, otherwise it is \texttt{load} d as a file. A typical invocation
might look like:


\begin{wideverbatim}
$ pil app/file1.l -main app/file2.l +
\end{wideverbatim}

It loads the debugging environment, an application source file, calls
the main function, and then loads another application source. In a
typical development and debugging session, this line is often modified
using the shell's history mechanisms, e.g. by inserting debugging
statements:


\begin{wideverbatim}
$ pil app/file1.l -"trace 'foo" -main -"debug 'bar" app/file2.l +
\end{wideverbatim}

Another convenience during debugging and testing is to put things into
the command line (shell history) which would otherwise have to be done
each time in the application's user interface:


\begin{wideverbatim}
$ pil app/file1.l -main app/file2.l -go -'login "name" "password"' +
\end{wideverbatim}

The final production release of an application usually includes a shell
script, which initializes the environment, does some bookkeeping and
cleanup, and calls the application with a proper command line. It is no
problem if the command line is long and complicated.

For small utility programs, however, this is overkill. Enter full
PicoLisp scripts.

 
\subsection{PicoLisp scripts}
\label{sec:tut-picolisp-scripts}


It is better to write a single executable file using the mechanisms of
``interpreter files''. If the first two characters in an executable file
are ` \texttt{\#!} ', the operating system kernel will pass this file to an
interpreter program whose pathname is given in the first line
(optionally followed by a single argument). This is fast and efficient,
because the overhead of a subshell is avoided.

Let's assume you installed PicoLisp in the directory
``/home/foo/picolisp/'', and put links to the executable and the
installation directory as:


\begin{wideverbatim}
$ ln -s /home/foo/picolisp /usr/lib/picolisp
$ ln -s /usr/lib/picolisp/bin/picolisp /usr/bin/picolisp
\end{wideverbatim}

Then a simple hello-world script might look like:


\begin{wideverbatim}
#!/usr/bin/picolisp /usr/lib/picolisp/lib.l
(prinl "Hello world!")
(bye)
\end{wideverbatim}

If you write this into a text file, and use \texttt{chmod} to set it to
``executable'', it can be executed like any other command. Note that
(because \texttt{\#} is the comment character in PicoLisp) the first line will
not be interpreted, and you can still use that file as a normal command
line argument to PicoLisp (useful during debugging).

 
\subsection{Grab command line arguments from PicoLisp scripts}
\label{sec:tut-grab-command-line-arguments-from-picolisp-scripts}


The fact that a hyphen causes evaluation of command line arguments can
be used to simulate something like command line options. The following
script defines two functions \texttt{a} and \texttt{f}, and then calls \texttt{(load T)} to
process the rest of the command line (which otherwise would be ignored
because of the \texttt{(bye)} statement):


\begin{wideverbatim}
#!/usr/bin/picolisp /usr/lib/picolisp/lib.l

(de a ()
   (println '-a '-> (opt)) )

(de f ()
   (println '-f '-> (opt)) )

(load T)
(bye)
\end{wideverbatim}

(\texttt{opt} retrieves the next command line option)

Calling this script (let's say we named it ``testOpts'') gives:


\begin{wideverbatim}
$ ./testOpts -f abc
-f -> "abc"
$ ./testOpts -a xxx  -f yyy
-a -> "xxx"
-f -> "yyy"
\end{wideverbatim}

We have to be aware of the fact, however, that the aggregation of
arguments like


\begin{wideverbatim}
$ ./testOpts -axxx  -fyyy
\end{wideverbatim}

or


\begin{wideverbatim}
$ ./testOpts -af yyy
\end{wideverbatim}

cannot be achieved with this simple and general mechanism of command
line processing.

 
\subsection{Run scripts from arbitrary places on the host file system}
\label{sec:tut-run-scripts-from-arbitrary-places-on-the-host-file-system}


Utilities are typically used outside the context of the PicoLisp
environment. All examples above assumed that the current working
directory is the PicoLisp installation directory, which is usually all
right for applications developed in that environment. Command line file
arguments like ``app/file1.l'' will be properly found.

To allow utilities to run in arbitrary places on the host file system,
the concept of \emph{home directory substitution} was introduced. The
interpreter remembers internally at start-up the pathname of its first
argument (usually ``lib.l''), and substitutes any leading ` \texttt{@} ' character
in subsequent file names with that pathname. Thus, to run the above
example in some other place, simply write:


\begin{wideverbatim}
$ /home/foo/picolisp/dbg @app/file1.l -main @app/file2.l
\end{wideverbatim}

that is, supply a full path name to the initial command (here `p'), or
put it into your \texttt{PATH} variable, and prefix each file which has to be
loaded from the PicoLisp home directory with a \texttt{@} character. ``Normal''
files (not prefixed by \texttt{@}) will be opened or created relative to the
current working directory as usual.

Stand-alone scripts will often want to load additional modules from the
PicoLisp environment, beyond the ``lib.l'' we provided in the first line
of the hello-world script. Typically, at least a call to


\begin{wideverbatim}
(load "@lib/misc.l")
\end{wideverbatim}

(note the home directory substitution) will be included near the
beginning of the script.

As a more complete example, here is a script which extracts the date,
name and size of the latest official PicoLisp release version from the
download web site, and prints it to standard output:


\begin{wideverbatim}
#!/usr/bin/picolisp /usr/lib/picolisp/lib.l

(load "@lib/misc.l" "@lib/http.l")

(use (@Date @Name @Size)
   (when
      (match
         '(@Date " " "-" " " @Name " " "(" @Size ")")
         (client "software-lab.de" 80 "down.html"
            (from "Release Archive")
            (from ".tgz">")
            (till "") ) )
      (prinl @Name)
      (prinl @Date " -- " @Size) ) )

(bye)
\end{wideverbatim}

 
\subsection{Editing scripts}
\label{sec:tut-editing-scripts}


We recommend that you have a terminal window open, and try the examples
by yourself. You may either type them in, directly to the PicoLisp
interpreter, or edit a separate source file (e.g. \texttt{''@doc/fun.l''}) in a
second terminal window and load it into PicoLisp with


\begin{wideverbatim}
: (load "@doc/fun.l")
\end{wideverbatim}

each time you have modified and saved it.

 
\subsection{Editing scripts with vi}
\label{sec:tut-editing-scripts-with-vi}


Once a function is loaded from a source file, you can call `vim'
directly on that function with


\begin{wideverbatim}
: (vi 'fact)
\end{wideverbatim}

The function `vi' opens the appropriate source file, and jumps to the
right line where `fact' is defined. When you modify it, you can simply
call `ld' to (re)load that source file


\begin{wideverbatim}
: (ld)
\end{wideverbatim}

 
\section{Objects and Classes}
\label{sec:tut-objects-and-classes}


The PicoLisp object model is very simple, yet flexible and powerful.
Objects as well as classes are both implemented as symbols. In fact,
there is no formal difference between objects and classes; classes are
more a conceptual design consideration in the head of the programmer
than a physical reality.

Having said this, we declare that normally:

\begin{enumerate}
\item A Class
\begin{itemize}
\item Has a name (interned symbol)
\item Has method definitions and superclass(es) in the value cell
\item May have class variables (attributes) in the property list
\end{itemize}
\item An Object
\begin{itemize}
\item Has no name (anonymous symbol) or is an external symbol
\item Has class(es) (and optionally method definitions) in the value
      cell
\item Has instance variables (attributes) in the property list
\end{itemize}
\end{enumerate}

So the main difference between classes and objects is that the former
ones usually are internal symbols. By convention, their names start with
a  \texttt{+} . Sometimes it makes sense, however, to create named objects (as
global singletons, for example), or even anonymous classes.

Both classes and objects have a list in their value cell, consisting of
method definitions (often empty for objects) and (super)class(es). And
both classes and objects have local data in their property lists (often
empty for classes). This implies, that any given object (as an instance
of a class) may have private (object-local) methods defined.

It is rather difficult to contrive a simple OOP example. We constructed
a hierarchy of geometric shapes, with a base class \texttt{+Shape} and two
subclasses \texttt{+Rectangle} and \texttt{+Circle}.

The source code is included as ``\texttt{@doc/shape.l}'' in the PicoLisp
distribution, so you don't have to type it in. Just \texttt{load} the file, or
start it from the shell as:


\begin{wideverbatim}
$ pil @doc/shape.l +
\end{wideverbatim}

Let's look at it piece by piece. Here's the base class:


\begin{wideverbatim}
(class +Shape)
# x y

(dm T (X Y)
   (=: x X)
   (=: y Y) )

(dm move> (DX DY)
   (inc (:: x) DX)
   (inc (:: y) DY) )
\end{wideverbatim}

The first line `\texttt{(class +Shape)}' defines the symbol
\texttt{+Shape} as a class without superclasses. The following method
definitions will go to that class.

The comment  \texttt{\# x y}  in the second line is just a convention, to
indicate what instance variables (properties) that class uses. As
PicoLisp is a dynamic language, a class can be extended at runtime with
any number of properties, and there is nothing like a fixed object size
or structure. This comment is a hint of what the programmer thinks to be
essential and typical for that class. In the case of \texttt{+Shape}, \texttt{x} and
\texttt{y} are the coordinates of the shape's origin.

Then we have two method definitions, using the keyword \texttt{dm} for ``define
method''. The first method is special, in that its name is \texttt{T}. Each time
a new object is created, and a method with that name is found in its
class hierarchy, that method will be executed. Though this looks like a
``constructor'' in other programming languages, it should probably better
be called ``initializer''. The \texttt{T} method of \texttt{+Shape} takes two arguments
\texttt{X} and \texttt{Y}, and stores them in the object's property list.

The second method \texttt{move>} changes the object's origin by adding the
offset values \texttt{DX} and \texttt{DY} to the object's origin.

Now to the first derived class:


\begin{wideverbatim}
(class +Rectangle +Shape)
# dx dy

(dm T (X Y DX DY)
   (super X Y)
   (=: dx DX)
   (=: dy DY) )

(dm area> ()
   (* (: dx) (: dy)) )

(dm perimeter> ()
   (* 2 (+ (: dx) (: dy))) )

(dm draw> ()
   (drawRect (: x) (: y) (: dx) (: dy)) )
\end{wideverbatim}

\texttt{+Rectangle} is defined as a subclass of \texttt{+Shape}. The comment
 \texttt{\# dx dy}  indicates that \texttt{+Rectangle} has a width and a height in
addition to the origin coordinates inherited from \texttt{+Shape}.

The \texttt{T} method passes the origin coordinates \texttt{X} and \texttt{Y} to the \texttt{T}
method of the superclass (\texttt{+Shape}), then stores the width and height
parameters into \texttt{dx} and \texttt{dy}.

Next we define the methods \texttt{area>} and \texttt{perimeter>} which do some
obvious calculations, and a method \texttt{draw>} which is supposed to draw the
shape on the screen by calling some hypothetical function \texttt{drawRect}.

Finally, we define a \texttt{+Circle} class in an analog way, postulating the
hypothetical function \texttt{drawCircle}:


\begin{wideverbatim}
(class +Circle +Shape)
# r

(dm T (X Y R)
   (super X Y)
   (=: r R) )

(dm area> ()
   (*/ (: r) (: r) 31415927 10000000) )

(dm perimeter> ()
   (*/ 2 (: r) 31415927 10000000) )

(dm draw> ()
   (drawCircle (: x) (: y) (: r)) )
\end{wideverbatim}

Now we can experiment with geometrical shapes. We create a rectangle at
point (0,0) with a width of 30 and a height of 20, and keep it in the
variable \texttt{R}:


\begin{wideverbatim}
: (setq R (new '(+Rectangle) 0 0 30 20))  # New rectangle
-> $134432824                             # returned anonymous symbol
: (show R)
$134432824 (+Rectangle)                   # Show the rectangle
   dy 20
   dx 30
   y 0
   x 0
\end{wideverbatim}

We see that the symbol \texttt{\$134432824} has a list of classes
`\texttt{(+Rectangle)}' in its value cell, and the coordinates, width
and height in is property list.

Sending messages to that object


\begin{wideverbatim}
: (area> R)                               # Calculate area
-> 600
: (perimeter> R)                          # and perimeter
-> 100
\end{wideverbatim}

will return the values for area and perimeter, respectively.

Then we move the object's origin:


\begin{wideverbatim}
: (move> R 10 5)                          # Move 10 right and 5 down
-> 5
: (show R)
$134432824 (+Rectangle)
   y 5                                    # Origin changed (0,0) -> (10,5)
   x 10
   dy 20
   dx 30
\end{wideverbatim}

Though a method \texttt{move>} wasn't defined for the \texttt{+Rectangle} class, it is
inherited from the \texttt{+Shape} superclass.

Similarly, we create and use a circle object:


\begin{wideverbatim}
: (setq C (new '(+Circle) 10 10 30))      # New circle
-> $134432607                             # returned anonymous symbol
: (show C)
$134432607 (+Circle)                      # Show the circle
   r 30
   y 10
   x 10
-> $134432607
: (area> C)                               # Calculate area
-> 2827
: (perimeter> C)                          # and perimeter
-> 188
: (move> C 10 5)                          # Move 10 right and 5 down
-> 15
: (show C)
$134432607 (+Circle)                      # Origin changed (10,10) -> (20,15)
   y 15
   x 20
   r 30
\end{wideverbatim}

It is also easy to send messages to objects in a list:


\begin{wideverbatim}
: (mapcar 'area> (list R C))              # Get list of areas
-> (600 2827)
: (mapc
   '((Shape) (move> Shape 10 10))         # Move all 10 right and down
   (list R C) )
-> 25
: (show R)
$134431493 (+Rectangle)
   y 15
   x 20
   dy 20
   dx 30
-> $134431493
: (show C)
$134431523 (+Circle)
   y 25
   x 30
   r 30
\end{wideverbatim}

Assume that we want to extend our shape system. From time to time, we
need shapes that behave exactly like the ones above, but are tied to a
fixed position. That is, they do not change their position even if they
receive a \texttt{move>} message.

One solution would be to modify the \texttt{move>} method in the \texttt{+Shape} class
to a no-operation. But this would require to duplicate the whole shape
hierarchy (e.g. by defining \texttt{+FixedShape}, \texttt{+FixedRectangle} and
\texttt{+FixedCircle} classes).

The PicoLisp Way is the use of Prefix Classes through multiple
inheritance. It uses the fact that searching for method definitions is a
depth-first, left-to-right search of the class tree. We define a prefix
class:


\begin{wideverbatim}
: (class +Fixed)

(dm move> (DX DY))  # A do-nothing method
\end{wideverbatim}

We can now create a fixed rectangle, and try to move it:


\begin{wideverbatim}
: (setq R (new '(+Fixed +Rectangle) 0 0 30 20))    # '+Fixed' prefix class
-> $134432881
: (move> R 10 5)                                   # Send 'move>' message
-> NIL
: (show R)
$134432881 (+Fixed +Rectangle)
   dy 20
   dx 30
   y 0                                             # Did not move!
   x 0
\end{wideverbatim}

We see, prefix classes can surgically change the inheritance tree for
selected objects or classes.

Alternatively, if fixed rectangles are needed often, it might make sense
to define a new class \texttt{+FixRect}:


\begin{wideverbatim}
: (class +FixRect +Fixed +Rectangle)
-> +FixRect
\end{wideverbatim}

and then use it directly:


\begin{wideverbatim}
: (setq R (new '(+FixRect) 0 0 30 20))
-> $13455710
\end{wideverbatim}

 
\section{Persistence (External Symbols)}
\label{sec:tut-persistence-(external-symbols)}


PicoLisp has persistent objects built-in as a first class data type.
With ``first class'' we mean not just the ability of being passed around,
or returned from functions (that's a matter of course), but that they
are a primary data type with their own interpreter tag bits. They are,
in fact, a special type of symbolic atoms (called
``\emph{External Symbols}''), that happen to be read from
pool file(s) when accessed, and written back automatically when
modified.

In all other aspects they are normal symbols. They have a value cell, a
property list and a name.

The name cannot be directly controlled by the programmer, as it is
assigned when the symbol is created. It is an encoded index of the
symbol's location in its database file. In its visual representation
(output by the \texttt{print} functions and input by the \texttt{read} functions) it
is surrounded by braces.

To make use of external symbols, you need to open a database first:


\begin{wideverbatim}
: (pool "test.db")
\end{wideverbatim}

If a file with that name did not exist, it got created now. Also created
at the same moment was \texttt{\{1\}}, the very first symbol in the file. This
symbol is of great importance, and is handled especially by PicoLisp.
Therefore a global constant \texttt{*DB} exists, which points to that symbol
\texttt{\{1\}}, which should be used exclusively to access the symbol \texttt{\{1\}}, and
which should never be modified by the programmer.


\begin{wideverbatim}
: *DB                   # The value of '*DB'
-> {1}                  # is '{1}'
: (show *DB)
{1} NIL                 # Value of '{1}' is NIL, property list empty
\end{wideverbatim}

Now let's put something into the value cell and property list of \texttt{\{1\}}.


\begin{wideverbatim}
: (set *DB "Hello world")  # Set value of '{1}' to a transient symbol (string)
-> "Hello world"
: (put *DB 'a 1)           # Property 'a' to 1
-> 1
: (put *DB 'b 2)           # Property 'b' to 2
-> 2
: (show *DB)               # Now show the symbol '{1}'
{1} "Hello world"
   b 2
   a 1
\end{wideverbatim}

Note that instead of `\texttt{(set *DB "Hello world")}', we might also
have written `\texttt{(setq {1} "Hello world")}', and instead of
`\texttt{(put *DB 'a 1)}' we might have written `\texttt{(put '{1} 'a
  1)}'. This would have the same effect, but as a rule external
symbols should never be be accessed literally in application programs,
because the garbage collector might not be able to free these symbols
and all symbols connected to them (and that might well be the whole
database). It is all right, however, to access external symbols
literally during interactive debugging.

Now we can create our first own external symbol. This can be done with
\texttt{new} when a \texttt{T} argument is supplied:


\begin{wideverbatim}
: (new T)
-> {2}               # Got a new symbol
\end{wideverbatim}

We store it in the database root \texttt{\{1\}}:


\begin{wideverbatim}
: (put *DB 'newSym '{2})   # Literal '{2}' (ok during debugging)
-> {2}
: (show *DB)
{1} "Hello world"
   newSym {2}              # '{2}' is now stored in '{1}'
   b 2
   a 1
\end{wideverbatim}

Put some property value into `\{2\}'


\begin{wideverbatim}
: (put *DB 'newSym 'x 777) # Put 777 as 'x'-property of '{2}'
-> 777
: (show *DB 'newSym)       # Show '{2}' (indirectly)
{2} NIL
   x 777
-> {2}
: (show '{2})              # Show '{2}' (directly)
{2} NIL
   x 777
\end{wideverbatim}

All modifications to - and creations of - external symbols done so far
are not written to the database yet. We could call \texttt{rollback} (or simply
exit PicoLisp) to undo all the changes. But as we want to keep them:


\begin{wideverbatim}
: (commit)           # Commit all changes
-> T
: (bye)              # Exit picolisp
$                    # back to the shell
\end{wideverbatim}

So, the next time when ..


\begin{wideverbatim}
$ pil +                 # .. we start PicoLisp
: (pool "test.db")      # and open the database file,
-> T
: (show *DB)            # our two symbols are there again
{1} "Hello world"
   newSym {2}
   b 2
   a 1
-> {1}
: (show *DB 'newSym)
{2} NIL
   x 777
-> {2}
\end{wideverbatim}

 
\section{Database Programming}
\label{sec:tut-database-programming}


To a database, there is more than just persistence. PicoLisp includes an
entity/relation class framework (see also \emph{Database})
which allows a close mapping of the application data structure to the
database.

We provided a simple yet complete database and GUI demo application in
\texttt{@doc/family.tgz} and \texttt{@doc/family64.tgz}. Please unpack the first one
if you use a 32-bit system, and the second one on a 64-bit system. Both
contain the sources in \texttt{@doc/family.l}, and an initial database in the
``family/'' subdirectory.

To use it, please unpack it first in your current working directory,
then start it up in the following way:


\begin{wideverbatim}
$ pil family.l -main +
:
\end{wideverbatim}

This loads the source file, initializes the database by calling the
\texttt{main} function, and prompts for user input.

The data model is small and simple. We define a class \texttt{+Person} and two
subclasses \texttt{+Man} and \texttt{+Woman}.


\begin{wideverbatim}
(class +Person +Entity)
\end{wideverbatim}

\texttt{+Person} is a subclass of the \texttt{+Entity} system class. Usually all
objects in a database are of a direct or indirect subclass of \texttt{+Entity}.
We can then define the relations to other data with the \texttt{rel} function.


\begin{wideverbatim}
(rel nm (+Need +Sn +Idx +String))      # Name
\end{wideverbatim}

This defines the name property (\texttt{nm}) of a person. The first argument to
\texttt{rel} is always a list of relation classes (subclasses of \texttt{+relation}),
optionally followed by further arguments, causing relation daemon
objects be created and stored in the class definition. These daemon
objects control the entity's behavior later at runtime.

Relation daemons are a kind of \emph{metadata}, controlling the interactions
between entities, and maintaining database integrity. Like other
classes, relation classes can be extended and refined, and in
combination with proper prefix classes a fine-grained description of the
application's structure can be produced.

Besides primitive relation classes, like \texttt{+Number}, \texttt{+String} or
\texttt{+Date}, there are

\begin{itemize}
\item relations between entities, like \texttt{+Link} (unidirectional link),
   \texttt{+Joint} (bidirectional link) or \texttt{+Hook} (object-local index trees)
\item relations that bundle other relations into a single unit (\texttt{+Bag})
\item a \texttt{+List} prefix class
\item a \texttt{+Blob} class for ``binary large objects''
\item prefix classes that maintain index trees, like \texttt{+Key} (unique index),
   \texttt{+Ref} (non-unique index) or \texttt{+Idx} (full text index)
\item prefix classes which in turn modify index class behavior, like \texttt{+Sn}
   (modified soundex algorithm~\cite{knuth73} for tolerant
   searches)
\item a \texttt{+Need} prefix class, for existence checks
\item a \texttt{+Dep} prefix class controlling dependencies between other
   relations
\end{itemize}

In the case of the person's name (\texttt{nm}) above, the relation object is of
type \texttt{(+Need +Sn +Idx +String)}. Thus, the name of each person in this
demo database is a mandatory attribute (\texttt{+Need}), searchable with the
soundex algorithm (\texttt{+Sn}) and a full index (\texttt{+Idx}) of type \texttt{+String}.


\begin{wideverbatim}
(rel pa (+Joint) kids (+Man))          # Father
(rel ma (+Joint) kids (+Woman))        # Mother
(rel mate (+Joint) mate (+Person))     # Partner
\end{wideverbatim}

The attributes for \emph{father} (\texttt{pa}), \emph{Mother}
(\texttt{ma}) and \emph{partner} (\texttt{mate}) are all defined as
\texttt{+Joint}s. A \texttt{+Joint} is probably the most powerful
relation mechanism in PicoLisp; it establishes a bidirectional link
between two objects.

The above declarations say that the \emph{father} (\texttt{pa}) attribute points to
an object of type \texttt{+Man}, and is joined with that object's \texttt{kids}
attribute (which is a list of joints back to all his children).

The consistency of \texttt{+Joint}  is maintained automatically by the relation
daemons. These become active whenever a value is stored to a person's
\texttt{pa}, \texttt{ma}, \texttt{mate} or \texttt{kids} property.

For example, interesting things happen when a person's \texttt{mate} is changed
to a new value. Then the \texttt{mate} property of the old mate's object is
cleared (she has no mate after that). Now when the person pointed to by
the new value already has a mate, then that mate's \texttt{mate} property gets
cleared, and the happy new two mates now get their joints both set
correctly.

The programmer doesn't have to care about all that. He just declares
these relations as \texttt{+Joint} .

The last four attributes of person objects are just static data:


\begin{wideverbatim}
(rel job (+Ref +String))               # Occupation
(rel dat (+Ref +Date))                 # Date of birth
(rel fin (+Ref +Date))                 # Date of death
(rel txt (+String))                    # Info
\end{wideverbatim}

They are all searchable via a non-unique index (\texttt{+Ref}). Date values in
PicoLisp are just numbers, representing the day number (starting first
of March of the year zero).

A method \texttt{url>} is defined:


\begin{wideverbatim}
(dm url> ()
   (list "!person" '*ID This) )
\end{wideverbatim}

It is needed later in the GUI, to cause a click on a link to switch to
that object.

The classes \texttt{+Man} and \texttt{+Woman} are subclasses of \texttt{+Person}:


\begin{wideverbatim}
(class +Man +Person)
(rel kids (+List +Joint) pa (+Person)) # Children

(class +Woman +Person)
(rel kids (+List +Joint) ma (+Person)) # Children
\end{wideverbatim}

They inherit everything from \texttt{+Person}, except for the \texttt{kids} attribute.
This attribute joins with the \texttt{pa} or \texttt{ma} attribute of the child,
depending on the parent's gender.

That's the whole data model for our demo database application.

It is followed by a call to \texttt{dbs} (``database sizes''). This call is
optional. If it is not present, the whole database will reside in a
single file, with a block size of 256 bytes. If it is given, it should
specify a list of items, each having a number in its CAR, and a list in
its CDR. The CARs taken together will be passed later to
\emph{pool}, causing an individual database file with that
size to be created. The CDRs tell what entity classes (if an item is a
symbol) or index trees (if an item is a list with a class in its CAR and
a list of relations in its CDR) should be placed into that file.

A handful of access functions is provided, that know about database
relationships and thus allows higher-level access modes to the external
symbols in a database.

For one thing, the B-Trees created and maintained by the index daemons
can be used directly. Though this is rarely done in a typical
application, they form the base mechanisms of other access modes and
should be understood first.

The function \texttt{tree} returns the tree structure for a given relation. To
iterate over the whole tree, the functions \texttt{iter} and \texttt{scan} can be
used:


\begin{wideverbatim}
(iter (tree 'dat '+Person) '((P) (println (datStr (get P 'dat)) (get P 'nm))))
"1770-08-03" "Friedrich Wilhelm III"
"1776-03-10" "Luise Augusta of Mecklenburg-Strelitz"
"1797-03-22" "Wilhelm I"
...
\end{wideverbatim}

They take a function as the first argument. It will be applied to all
objects found in the tree (to show only a part of the tree, an optional
begin- and end-value can be supplied), producing a simple kind of
report.

More useful is \texttt{collect}; it returns a list of all objects that fall
into a range of index values:


\begin{wideverbatim}
: (collect 'dat '+Person (date 1982 1 1) (date 1988 12 31))
-> ({2-M} {2-L} {2-E})
\end{wideverbatim}

This returns all persons born between 1982 and 1988. Let's look at them
with \texttt{show}:


\begin{wideverbatim}
: (more (collect 'dat '+Person (date 1982 1 1) (date 1988 12 31)) show)
{2-M} (+Man)
   nm "William"
   dat 724023
   ma {2-K}
   pa {2-J}
   job "Heir to the throne"

{2-L} (+Man)
   nm "Henry"
   dat 724840
   ma {2-K}
   pa {2-J}
   job "Prince"

{2-E} (+Woman)
   nm "Beatrice"
   dat 726263
   ma {2-D}
   job "Princess"
   pa {2-B}
\end{wideverbatim}

If you are only interested in a certain attribute, e.g. the name, you
can return it directly:


\begin{wideverbatim}
: (collect 'dat '+Person (date 1982 1 1) (date 1988 12 31) 'nm)
-> ("William" "Henry" "Beatrice")
\end{wideverbatim}

To find a single object in the database, the function \texttt{db} is used:


\begin{wideverbatim}
: (db 'nm '+Person "Edward")
-> {2-;}
\end{wideverbatim}

If the key is not unique, additional arguments may be supplied:


\begin{wideverbatim}
: (db 'nm '+Person "Edward"  'job "Prince"  'dat (date 1964 3 10))
-> {2-;}
\end{wideverbatim}

The programmer must know which combination of keys will suffice to
specify the object uniquely. The tree search is performed using the
first value (``Edward''), while all other attributes are used for
filtering. Later, in the \emph{Pilog} section, we will show how
more general (and possibly more efficient) searches can be performed.

 
\section{User Interface (GUI) Programming}
\label{sec:tut-user-interface-(gui)-programming}


The only types of GUI supported by the PicoLisp application server
framework is either dynamically generated (but static by nature) HTML,
or an interactive XHTML/CSS framework with the optional use of
JavaScript.

Before we explain the GUI of our demo database application, we present a
minimal example for a plain HTML-GUI in \texttt{@doc/hello.l}. Start the
application server as:


\begin{wideverbatim}
$ pil @lib/http.l -'server 8080 "@doc/hello.l"' -wait
\end{wideverbatim}

Now point your browser to the address `\url{http://localhost:8080}'. You
should see a very simple HTML page. You can come back here with the
browser's BACK button.

You can call the page repeatedly, or concurrently with many clients if
you like. To terminate the server, you have to send it a TERM signal
(e.g. `\texttt{killall pil}'), or type the \texttt{Ctrl-C} key in the console window.

In our demo database application, a single function \texttt{person} is
responsible for the whole GUI. Again, please look at \texttt{@doc/family.l}.

To start the database \emph{and} the application server, call:


\begin{wideverbatim}
$ pil family.l -main -go +
\end{wideverbatim}

As before, the database is opened with \texttt{main}. The function \texttt{go} is also
defined in \texttt{@doc/family.l}:


\begin{wideverbatim}
(de go ()
   (server 8080 "!person") )
\end{wideverbatim}

It starts the HTTP server listening on TCP port 8080 (we did a similar
thing in our minimal GUI example above directly on the command line).
Each connect to that port will cause the function \texttt{person} to be
invoked.

Again, point your browser to the address `\url{http://localhost:8080}'.

You should see a new browser window with an input form created by the
function \texttt{person}. We provided an initial database in ``family/[1--4]''.
You can navigate through it by clicking on the pencil icons besides the
input fields.

The chart with the children data can be scrolled using the down (\texttt{v})
and up (\texttt{\textasciicircum{}}) buttons.

A click on the button ``Select'' below opens a search dialog. You can
scroll through the chart as before. Again, a click on a pencil will jump
to that person. You can abort the dialog with a click on the
``Cancel''-button.

The search fields in the upper part of the dialog allow a conjunctive
search. If you enter ``Edward'' in the ``Name'' field and click ``Search'',
you'll see all persons having the string ``Edward'' in their name. If you
also enter ``Duke'' in the ``Occupation'' field, the result list will reduce
to only two entries.

To create a new person, press the ``New Man'' or ``New Woman'' button. A new
empty form will be displayed. Please type a name into the first field,
and perhaps also an occupation and birth date. Any change of contents
should be followed by a press on the ``Done'' button, though any other
button (also Scroll or Select-buttons) will also do.

To assign a \emph{father} attribute, you can either type a name directly into
the field (if that person already exists in the database and you know
the exact spelling), or use the ``Set''-button (\texttt{->}) to the left of that
field to open the search dialog. If you type in the name directly, your
input must exactly match upper and lower case.

Alternatively, you may create a new person and assign a child in the
``Children'' chart.

On the console where you started PicoLisp, there should a prompt have
appeared just when the browser connected. You can debug the application
interactively while it is running. For example, the global variable
\texttt{*Top} always contains the top level GUI object:


\begin{wideverbatim}
: (show *Top)
\end{wideverbatim}

To take a look at the first field on the form:


\begin{wideverbatim}
: (show *Top 'gui 1)
\end{wideverbatim}

A production application would be started in a slightly different way:


\begin{wideverbatim}
$ pil family.l -main -go -wait
\end{wideverbatim}

In that case, no debug prompt will appear. In both cases, however, two
\texttt{pil} processes will be running now. One is the initial server process
which will continue to run until it is killed. The other is a child
process holding the state of the GUI in the browser. It will terminate
some time after the browser is closed, or when \texttt{(bye)} or a \texttt{Ctrl-D} is
entered at the PicoLisp prompt.

Now back to the explanation of the GUI function \texttt{person}:


\begin{wideverbatim}
(de person ()
   (app)
   (action
      (html 0 (get (default *ID (val *DB)) 'nm) "@lib.css" NIL
         (form NIL
            (<h2> (<id> (: nm)))
\end{wideverbatim}

For an in-depth explanation of that startup code, please refer to the
guide to \emph{PicoLisp Application Development}.

All components like fields and buttons are controlled by \texttt{form}. The
function \texttt{gui} creates a single GUI component and takes the type (a list
of classes) and a variable number of arguments depending on the needs of
these classes.


\begin{wideverbatim}
(gui '(+E/R +TextField) '(nm : home obj) 40 "Name")
\end{wideverbatim}

This creates a \texttt{+TextField} with the label ``Name'' and a length of 40
characters. The \texttt{+E/R} (: Entity/Relation) prefix class connects that
field to a database object, the \texttt{nm} attribute of a person in this case,
so that the person's name is displayed in that text field, and any
changes entered into that field are propagated to the database
automatically.


\begin{wideverbatim}
(gui '(+ClassField) '(: home obj) '(("Male" +Man) ("Female" +Woman)))
\end{wideverbatim}

A \texttt{+ClassField} displays and changes the class of an object, in this
case the person's sex from \texttt{+Man} to \texttt{+Woman} and vice versa.

As you see, there is no place where explicit accesses to the database
have to be programmed, no \texttt{select} or \texttt{update}. This is all encapsulated
in the GUI components, mainly in the \texttt{+E/R} prefix class. The above
function \texttt{person} is fully functional as we present it and allows
creation, modification and deletion of person objects in the database.

The two buttons on the bottom right generate simple reports:

The first one shows all contemporaries of the person that is currently
displayed, i.e. all persons who did not die before, or were not born
after that person. This is a typical PicoLisp report, in that in
addition to the report's HTML page, a temporary file may be generated,
suitable for download (and import into a spread sheet), and from which a
PDF can be produced for print-out.

In PicoLisp, there is not a real difference between a plain HTML-GUI and
a report. Again, the function \texttt{html} is used to generate the page.

The second report is much simpler. It produces a recursive structure of
the family.

In both reports, links to the person objects are created which allow
easy navigation through the database.

 
\section{Pilog --- PicoLisp Prolog}
\label{sec:tut-pilog-picolisp-prolog}


This sections explains some cases of using Pilog in typical application
programming, in combination with persistent objects and databases.
Please refer to the \emph{Pilog} section of the PicoLisp
Reference for the basic usage of Pilog.

Again, we use our demo application \texttt{@doc/family.l} that was introduced
in the \emph{Database Programming} section.

Normally, Pilog is used either interactively to query the database
during debugging, or in applications to generate export data and
reports. In the following examples we use the interactive query
front-end functions \texttt{?} and \texttt{select}. An application will use \texttt{goal} and
\texttt{prove} directly, or use convenience functions like \texttt{pilog} or \texttt{solve}.

All Pilog access to external symbols is done via the two predicates
\texttt{db/3} and \texttt{select/3}.

\begin{itemize}
\item \texttt{db/3} corresponds to the Lisp-level functions \texttt{db} and \texttt{collect}, as
   it derives its data from a single relation. It can be used for simple
   database queries.
\item \texttt{select/3} provides for self-optimizing parallel access to an
   arbitrary number of relations. There is also a Lisp front-end
   function \texttt{select}, for convenient calls to the Pilog \texttt{select}
   predicate.
\end{itemize}

A predicate \texttt{show/1} is pre-defined for debugging purposes (a simple
glue to the Lisp-level function \texttt{show}, see \emph{Browsing}).
Searching with \texttt{db/3} for all persons having the string ``Edward'' in
their name:


\begin{wideverbatim}
: (? (db nm +Person "Edward" @P) (show @P))
{2-;} (+Man)
   nm "Edward"
   ma {2-:}
   pa {2-A}
   dat 717346
   job "Prince"
 @P={2-;}
{2-1B} (+Man)
   nm "Albert Edward"
   kids ({2-1C} {2-1D} {2-1E} {2-1F} {2-1G} {2-1H} {2-1I} {2-g} {2-a})
   job "Prince"
   mate {2-f}
   fin 680370
   dat 664554
 @P={2-1B}
...               # more results
\end{wideverbatim}

To search for all persons with ``Edward'' in their name who are married to
somebody with occupation ``Queen'':


\begin{wideverbatim}
: (? (db nm +Person "Edward" @P) (val "Queen" @P mate job) (show @P))
{2-1B} (+Man)
   mate {2-f}
   nm "Albert Edward"
   kids ({2-1C} {2-1D} {2-1E} {2-1F} {2-1G} {2-1H} {2-1I} {2-g} {2-a})
   job "Prince"
   fin 680370
   dat 664554
 @P={2-1B}
-> NIL            # only one result
\end{wideverbatim}

If you are interested in the names of ``Albert Edward'''s children:


\begin{wideverbatim}
: (? (db nm +Person "Albert Edward" @P) (lst @K @P kids) (val @Kid @K nm))
 @P={2-1B} @K={2-1C} @Kid="Beatrice Mary Victoria"
 @P={2-1B} @K={2-1D} @Kid="Leopold George Duncan"
 @P={2-1B} @K={2-1E} @Kid="Arthur William Patrick"
 @P={2-1B} @K={2-1F} @Kid="Louise Caroline Alberta"
 @P={2-1B} @K={2-1G} @Kid="Helena Augusta Victoria"
 @P={2-1B} @K={2-1H} @Kid="Alfred Ernest Albert"
 @P={2-1B} @K={2-1I} @Kid="Alice Maud Mary"
 @P={2-1B} @K={2-g} @Kid="Victoria Adelaide Mary"
 @P={2-1B} @K={2-a} @Kid="Edward VII"
-> NIL
\end{wideverbatim}

\texttt{db/3} can do a direct index access only for a single attribute (\texttt{nm} of
\texttt{+Person} above). To search for several criteria at the same time,
\texttt{select/3} has to be used:


\begin{wideverbatim}
: (?
   (select (@P)
      ((nm +Person "Edward") (nm +Person "Augusta" pa))  # Generator clauses
      (tolr "Edward" @P nm)                              # Filter clauses
      (tolr "Augusta" @P kids nm) )
   (show @P) )
{2-1B} (+Man)
   kids ({2-1C} {2-1D} {2-1E} {2-1F} {2-1G} {2-1H} {2-1I} {2-g} {2-a})
   mate {2-f}
   nm "Albert Edward"
   job "Prince"
   fin 680370
   dat 664554
 @P={2-1B}
-> NIL
\end{wideverbatim}

\texttt{select/3} takes a list of generator clauses which are used to retrieve
objects from the database, and a number of normal Pilog filter clauses.
In the example above the generators are

\begin{itemize}
\item \texttt{(nm +Person "Edward")} to generate persons with ``Edward'' in their
   names, and
\item \texttt{(nm +Person "Augusta" pa)} to find persons with ``Augusta'' in their
   names and generate persons using the \texttt{pa} (``father'') attribute.
\end{itemize}

All persons generated are possible candidates for our selection. The
\texttt{nm} index tree of \texttt{+Person} is traversed twice in parallel, optimizing
the search in such a way that successful hits get higher priority in the
search, depending on the filter clauses. The process will stop as soon
as any one of the generators is exhausted. Note that this is different
from the standard Prolog search algorithm.

The filter clauses in this example both use the pre-defined predicate
\texttt{tolr/3} for \emph{tolerant} string matches (according either to the soundex
algorithm (see the section \emph{Database Programming}) or to
substring matches), and filter objects that

\begin{itemize}
\item match ``Edward'' in their name: \texttt{(tolr "Edward" @P nm)}, and
\item match ``Augusta'' in one of their kids' names:
   \texttt{(tolr "Augusta" @P kids nm)}
\end{itemize}

A more typical and extensive example for the usage of \texttt{select} can be
found in the \texttt{qPerson} function in \texttt{@doc/family.l}. It is used in the
search dialog of the demo application, and searches for a person with
the name, the parents' and partner's names, the occupation and a time
range for the birth date. The relevant index trees in the database are
searched (actually only those trees where the user entered a search key
in the corresponding dialog field), and a logical AND of the search
attributes is applied to the result.

For example, press the ``Select'' button, enter ``Elizabeth'' into the
``Mother'' search field and ``Phil'' in the ``Partner'' search field, meaning
to look for all persons whose mother's name is like ``Elizabeth'' and
whose partner's name is like ``Phil''. As a result, two persons
(``Elizabeth II'' and ``Anne'') will show up.

In principle, \texttt{db/3} can be seen as a special case of \texttt{select/3}. The
following two queries are equivalent:


\begin{wideverbatim}
: (? (db nm +Person "Edward" @P))
 @P={2-;}
 @P={2-1B}
 @P={2-R}
 @P={2-1K}
 @P={2-a}
 @P={2-T}
-> NIL
: (? (select (@P) ((nm +Person "Edward"))))
 @P={2-;}
 @P={2-1B}
 @P={2-R}
 @P={2-1K}
 @P={2-a}
 @P={2-T}
-> NIL
\end{wideverbatim}

 
\section{Poor Man's SQL}
\label{sec:tut-poor-mans-sql}


 
\subsection{select}
\label{sec:tut-select}


For convenience, a \texttt{select} Lisp glue function is provided as a
front-end to the \texttt{select} predicate. Note that this function does not
evaluate its arguments (it is intended for interactive use), and that it
supports only a subset of the predicate's functionality. The syntax
resembles SELECT in the SQL language, for example:


\begin{wideverbatim}
# SELECT * FROM Person
: (select +Person)  # Step through the whole database
{2-o} (+Man)
   nm "Adalbert Ferdinand Berengar Viktor of Prussia"
   dat 688253
   ma {2-j}
   pa {2-h}
   fin 711698

{2-1B} (+Man)
   nm "Albert Edward"
   dat 664554
   job "Prince"
   mate {2-f}
   kids ({2-1C} {2-1D} {2-1E} {2-1F} {2-1G} {2-1H} {2-1I} {2-g} {2-a})
   fin 680370
...
\end{wideverbatim}


\begin{wideverbatim}
# SELECT * FROM Person WHERE nm LIKE "%Edward%"
: (select +Person nm "Edward")  # Show all Edwards
{2-;} (+Man)
   nm "Edward"
   dat 717346
   job "Prince"
   ma {2-:}
   pa {2-A}

{2-1B} (+Man)
   nm "Albert Edward"
   dat 664554
   job "Prince"
   kids ({2-1C} {2-1D} {2-1E} {2-1F} {2-1G} {2-1H} {2-1I} {2-g} {2-a})
   mate {2-f}
   fin 680370
...

# SELECT nm, dat FROM Person WHERE nm LIKE "%Edward%"
: (select nm dat +Person nm "Edward")
"Edward" "1964-03-10" {2-;}
"Albert Edward" "1819-08-26" {2-1B}
"George Edward" NIL {2-R}
"Edward Augustus Hanover" NIL {2-1K}
...

# SELECT dat, fin, p1.nm, p2.nm
#    FROM Person p1, Person p2
#    WHERE p1.nm LIKE "%Edward%"
#    AND p1.job LIKE "King%"
#    AND p1.mate = p2.mate  -- Actually, in a SQL model we'd need
#                           -- another table here for the join
: (select dat fin nm (mate nm) +Person nm "Edward" job "King")
"1894-06-23" "1972-05-28" "Edward VIII" "Wallace Simpson" {2-T}
"1841-11-09" NIL "Edward VII" "Alexandra of Denmark" {2-a}
-> NIL
\end{wideverbatim}

 
\subsection{update}
\label{sec:tut-update}


In addition (just to stay with the SQL terminology ;-), there is also an
\texttt{update} function. It is a front-end to the \texttt{set!>} and \texttt{put!>}
transaction methods, and should be used when single objects in the
database have to be modified by hand.

In principle, it would also be possible to use the \texttt{edit} function to
modify a database object. This is not recommended, however, because
\texttt{edit} does not know about relations to other objects (like Links,
Joints and index trees) and may easily cause database corruption.

In the most general case, the value of a property in a database object
is changed with the \texttt{put!>} method. Let's look at ``Edward'' from the
previous examples:


\begin{wideverbatim}
: (show '{2-;})
{2R} (+Man)
   job "Prince"
   nm "Edward"
   dat 717346
   ma {2-:}
   pa {20A}
-> {2-;}
\end{wideverbatim}

We might change the name to ``Johnny'' with \texttt{put!>}:


\begin{wideverbatim}
: (put!> '{2-;} 'nm "Johnny")
-> "Johnny"
\end{wideverbatim}

However, an easier and less error-prone prone way - especially when more
than one property has to be changed - is using \texttt{update}. It presents the
value cell (the list of classes) and then each property on its own line,
allowing the user to change it with the \emph{command line editor}.

Just hitting ENTER will leave that property unchanged. To modify it,
you'll typically hit ESC to get into command mode, and move the cursor
to the point of change.

For properties with nested list structures (\texttt{+List +Bag}), \texttt{update} will
recurse into the data structure.


\begin{wideverbatim}
: (update '{2-;})
{2-;} (+Man)      # ENTER
nm "Johnny"       # Modified the name to "Johnny"
ma {2-:}          # ENTER
pa {2-A}          # ENTER
dat 1960-03-10    # Modified the year from "1964" to "1960"
job "Prince"      # ENTER
-> {2-;}
\end{wideverbatim}

All changes are committed immediately, observing the rules of database
synchronization so that any another user looking at the same object will
have his GUI updated correctly.

To abort \texttt{update}, hit \texttt{Ctrl-X}.

If only a single property has to be changed, \texttt{update} can be called
directly for that property:


\begin{wideverbatim}
: (update '{2-;} 'nm)
{2-;} nm "Edward"
...
\end{wideverbatim}

 
% \section{References}
% \label{sec:tut-references}

\begin{thebibliography}{[9]}

\bibitem{knuth73} Donald E. Knuth: ``The Art of Computer Programming'', Vol.3,
Addison-Wesley, 1973, p. 392

\end{thebibliography}



